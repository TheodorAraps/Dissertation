\begin{abstract}
Η ανάγκη για αποδοτική παρακολούθηση και διαχείριση των ενεργειακών εγκαταστάσεων 
καθιστά επιτακτική την ανάπτυξη έξυπνων και αυτόνομων συστημάτων παρακολούθησης 
υποσταθμών. Στην παρούσα διπλωματική εργασία σχεδιάστηκε και υλοποιήθηκε ένα 
ολοκληρωμένο σύστημα απομακρυσμένης παρακολούθησης και ελέγχου ενός ηλεκτρικού 
υποσταθμού, βασισμένο στην τεχνολογία \en{LoRaWAN}.

Το σύστημα περιλαμβάνει δύο τριφασικούς μετρητές, ο καθένας από τους οποίους κατασκευάστηκε με τη 
χρήση της πλακέτας ανάπτυξης \en{The Things Uno} και τρεις αισθητήρες \en{PZEM-004T},  ένας ανά φάση, 
για την καταγραφή μονοφασικών ηλεκτρικών μεγεθών. Η συλλογή των δεδομένων επιτυγχάνεται 
μέσω ενός \en{LoRaWAN gateway}, το οποίο έχει συναρμολογηθεί βασιζόμενο σε ένα \en{Raspberry Pi 4B}, μία μονάδα 
συγκέντρωσης \en{iC880A-SPI} και μία κεραία σχετικά χαμηλού κέρδους των 2dBi. Το λογισμικό περιλαμβάνει 
την εγκατάσταση της στοίβας ανοιχτού κώδικα \en{The Things Stack} και του \en{LoRa Basics Station}, 
ενώ τα δεδομένα αποθηκεύονται και προβάλλονται μέσω διαδικτυακής εφαρμογής υλοποιημένης 
αξιοποιώντας το \en{web framework} \en{Spring Boot} της \en{JAVA} και την βιβλιοθήκη \en{React} της \en{JavaScript}.

Σκοπός της εργασίας είναι η πρακτική διερεύνηση της αξιοπιστίας, της επεκτασιμότητας 
και της χρηστικότητας των \en{LoRaWAN}-βασισμένων συστημάτων σε κρίσιμες εφαρμογές 
εποπτείας και αυτοματισμού υποσταθμών ηλεκτρικής ενέργειας.
\begin{keywords}
\en{LoRaWAN}, υποσταθμός, απομακρυσμένη παρακολούθηση, \en{Raspberry Pi}, 
\en{The Things Stack, LoRa Basics Station, end-device, IoT, Spring Boot, React}
\end{keywords}
\end{abstract}


\begin{abstracteng}
\en{The need for efficient monitoring and management of energy installations 
necessitates the development of intelligent and autonomous substation 
monitoring systems. In this diploma thesis, a complete system for the 
remote monitoring and control of an electrical substation was designed 
and implemented, based on LoRaWAN technology.

The system includes two three-phase meters, each built using the 
The Things Uno development board and three PZEM-004T 
sensors—one per phase—for capturing single-phase electrical measurements. 
Data collection is achieved through a LoRaWAN gateway based on a 
Raspberry Pi 4B, an iC880A-SPI concentrator module, and a 
high-gain antenna. The software stack includes the installation of 
the open-source The Things Stack and the LoRa Basics Station, 
while data is stored and visualized via a web application developed using 
Java Spring Boot and React.

The purpose of this work is to practically evaluate the reliability, scalability, 
and usability of LoRaWAN-based systems in critical applications related to the 
supervision and automation of electrical substations.}
\begin{keywordseng}
\en{LoRaWAN, substation, remote monitoring, Raspberry Pi, The Things Stack, 
LoRa Basics Station, end-device, IoT, Spring Boot, React}
\end{keywordseng}
\end{abstracteng}