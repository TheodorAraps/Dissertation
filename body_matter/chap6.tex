% --------------------------------------------
% Κεφάλαιο 6: Web Εφαρμογή Παρακολούθησης
% --------------------------------------------

\chapter{Ανάπτυξη \en{Web} Εφαρμογής }

\InitialCharacter{Σ}το κεφάλαιο αυτό παρουσιάζεται η \en{web} εφαρμογή που αναπτύχθηκε 
για την ε\-πο\-πτεί\-α του τριφασικού συστήματος μέτρησης, από το επίπεδο του διακομιστή 
(\en{backend}) μέχρι το περιβάλλον χρήστη (\en{frontend}). Η εφαρμογή αναλαμβάνει να 
συλλέγει τα \en{uplink} μηνύματα που δρομολογούνται από το \en{The Things Stack} προς 
ένα \en{HTTP webhook}, να τα αποθηκεύει σε σχεσιακή βάση δεδομένων και να τα 
παρουσιάζει σε γραφική μορφή σε πίνακες και διαγράμματα. Η υλοποίηση χωρίζεται σε δύο 
κύρια υποσυστήματα: μία \en{Spring Boot} εφαρμογή \en{backend} σε \en{Java} (με 
\en{PostgreSQL} και ασφάλεια βασισμένη σε \en{JSON Web Tokens}) και μία \en{React} εφαρμογή 
\en{frontend} σε \en{TypeScript} που «τρέχει» στον φυλλομετρητή και επικοινωνεί με το 
\en{REST API} του \en{backend} μέσω \en{HTTP} \en{JSON} κλήσεων. 
% Ολόκληρος ο κώδικας της πλατφόρμας είναι διαθέσιμος στα αντίστοιχα αποθετήρια \en{GitHub}. 

Στις επόμενες ενότητες περιγράφεται λεπτομερώς η αρχιτεκτονική της εφαρμογής, η δομή 
των βασικών κλάσεων και \en{endpoints}, η ροή των δεδομένων από το \en{The Things Stack} 
μέχρι την βάση δεδομένων και το \en{UI}, καθώς και οι επιλογές ασφάλειας και ανάπτυξης 
στο \en{LoRaWAN gateway} (το \en{Raspberry Pi} της εγκατάστασης).


% --------------------------------------------
% Ενότητα 6.1 Γενική αρχιτεκτονική εφαρμογής
% --------------------------------------------

\section{Γενική αρχιτεκτονική εφαρμογής}

Η \en{web} εφαρμογή έχει σχεδιαστεί με την εξής τριεπίπεδη λογική:

\begin{itemize}
  \item \textbf{Επίπεδο δικτύου \en{LoRaWAN}:} Οι τριφασικοί μετρητές στέλνουν περιοδικά 
  \en{uplinks} μέσω του \en{LoRaWAN gateway} στο \en{The Things Stack}, όπου γίνεται η 
  αποκρυπτογράφηση του \en{LoRaWAN} πακέτου, η εφαρμογή του \en{payload formatter} και η 
  παραγωγή \en{JSON} δομής με τις φυσικές μετρήσεις (τάση, ρεύμα, ισχύς, ενέργεια, 
  συχνότητα, συντελεστής ισχύος ανά φάση), όπως περιγράφηκε στα προηγούμενα κεφάλαια.

  \item \textbf{Επίπεδο \en{backend} (\en{Spring Boot} \en{REST API}):} Ένας \en{HTTP webhook} 
  που εκτίθεται από την εφαρμογή \en{backend} (σε συγκεκριμένη \en{URL}) δέχεται τα 
  \en{JSON} μηνύματα \en{uplink} από το \en{TTS}. Τα μηνύματα αυτά χαρτογραφούνται σε 
  αντικείμενα \en{DTO}, μετατρέπονται σε οντότητες \en{JPA} και αποθηκεύονται στη βάση 
  δεδομένων \en{PostgreSQL}. Παράλληλα, το ίδιο \en{REST API} παρέχει προστατευμένα 
  \en{endpoints} για ανάκτηση ιστορικών δεδομένων (π.χ. ανά μετρητή, ανά χρονικό διάστημα) 
  από την \en{React} εφαρμογή.

  \item \textbf{Επίπεδο \en{frontend} (\en{React} \en{SPA}):} Το \en{frontend} υλοποιείται 
  ως \en{Single Page Application} (\en{SPA}) με \en{React} και \en{TypeScript}. Ο 
  χρήστης εισέρχεται στο σύστημα με όνομα χρήστη και κωδικό (μία διαχειριστική εγγραφή), 
  λαμβάνει \en{JWT} \en{token} και στη συνέχεια μπορεί να βλέπει σε πραγματικό χρόνο 
  (ή με περιοδικά \en{refresh}) τις μετρήσεις των τριφασικών μετρητών σε διαγράμματα 
  γραμμής, ράβδων ή πίτας, καθώς και σε πίνακες με χρονοσήμανση.
\end{itemize}

Η συνολική αρχιτεκτονική είναι πλήρως ανεξάρτητη από το \en{LoRaWAN} υπόστρωμα: το 
\en{backend} βλέπει τα δεδομένα μόνο ως \en{HTTP} \en{JSON} μηνύματα από το \en{TTS}, ενώ 
το \en{frontend} θεωρεί το \en{backend} ως μια τυπική \en{REST} υπηρεσία. 


% ------------------------------------------------------
% Ενότητα 6.2 Backend: Spring Boot REST API και ασφάλεια
% ------------------------------------------------------

\section{\en{Backend}: \en{Spring Boot} \en{REST API} και ασφάλεια}

Το \en{backend} υλοποιείται ως \en{Spring Boot} εφαρμογή σε \en{Java 17}, με 
\en{Maven} για τη διαχείριση των εξαρτήσεων, \en{Spring Web} για το \en{REST API}, 
\en{Spring Data JPA} για την πρόσβαση στη βάση δεδομένων \en{PostgreSQL} και 
\en{Liquibase} για τον έλεγχο των αλλαγών στο σχήμα της βάσης.
Για την ασφάλεια χρησιμοποιείται αυθεντικοποίηση με \en{JWT} \en{filter} που 
παρεμβάλλεται στην \en{Spring Security} αλυσίδα και ελέγχει κάθε εισερχόμενο αίτημα 
προς τα προστατευμένα \en{endpoints}.

\subsection{Δομή πακέτων και βασικών κλάσεων}

Η εφαρμογή ακολουθεί μια κλασική στρωματοποιημένη δομή:

\begin{itemize}
  \item \textbf{\en{controller} πακέτα}: Περιλαμβάνουν τις \en{REST controllers} για τα 
  δεδομένα μετρήσεων (π.χ. ανάγνωση χρονοσειρών για έναν μετρητή) και για την αυθεντικοποίηση 
  (\en{login} και έκδοση \en{JWT}).

  \item \textbf{\en{service} πακέτα}: Υλοποιούν την επιχειρησιακή λογική (\en{business logic}), 
  όπως την επεξεργασία των \en{uplink} μηνυμάτων από το \en{TTS}, τον υπολογισμό ή την 
  φιλτράρισή τους ανά ημερομηνία, καθώς και την διαχείριση χρηστών.

  \item \textbf{\en{repository} πακέτα}: Ορίζουν τις διεπαφές \en{JPA repositories} που 
  διαχειρίζονται οντότητες όπως \en{SensorData}, \en{Meter} και \en{User}, χαρτογραφημένες 
  σε πίνακες της \en{PostgreSQL}.

  \item \textbf{\en{security} πακέτο}: Περιέχει τις κλάσεις \en{SecurityConfig}, \en{JWT} 
  βοηθητικές συναρτήσεις (π.χ. παραγωγή και επαλήθευση \en{tokens}) και το \en{filter} που 
  ελέγχει την επικεφαλίδα \en{Authorization: Bearer ...} σε κάθε αίτημα.
\end{itemize}

Το σχήμα της βάσης δεδομένων ελέγχεται από αρχεία \en{Liquibase changelog}, στα οποία 
περιγράφονται οι πίνακες \en{users} (μοντέλο χρήστη με \en{username} και \en{password}), 
οι πίνακες μετρητών και δεδομένων, καθώς και τυχόν περιορισμοί, \en{indexes} και σχέσεις 
(π.χ. ένα πλήθος εγγραφών δεδομένων ανά μετρητή).:contentReference[oaicite:3]{index=3} 

\subsection{Διαχείριση εισερχόμενων \en{uplinks} μέσω \en{webhook}}

Κεντρικό ρόλο στην πλατφόρμα παίζει το \en{webhook endpoint} στο οποίο το \en{The Things Stack} 
στέλνει τα \en{uplink} μηνύματα. Η διαδικασία έχει ως εξής:

\begin{enumerate}
  \item Στο \en{The Things Stack Console} έχει δημιουργηθεί μία \en{Application} όπου έχουν 
  καταχωρηθεί οι τριφασικοί μετρητές ως \en{end devices}. Για την εφαρμογή αυτή 
  ενεργοποιείται ένας \en{HTTP webhook} που στο πεδίο \en{Base URL} δείχνει προς την 
  \en{URL} του \en{backend} (π.χ. \en{http://192.168.0.100:8080/api/tts/uplink}).
  
  \item Για κάθε \en{uplink} μήνυμα, το \en{TTS} στέλνει ένα \en{HTTP POST} προς το 
  \en{webhook} με σώμα \en{JSON}. Στο \en{body} περιλαμβάνονται μεταδεδομένα 
  (\en{end\_device\_ids}, \en{received\_at}, \en{rx\_metadata} κ.λπ.) και το 
  \en{decoded\_payload} όπως το επιστρέφει το \en{payload formatter} (με πεδία \en{sensor1}, 
  \en{sensor2}, \en{sensor3} κ.ά.).

  \item Στον \en{backend}, ένας \en{controller} δέχεται το \en{POST}, το χαρτογραφεί σε 
  \en{DTO} κλάσεις (π.χ. \en{UplinkMessageDto}, \en{DecodedPayloadDto}) και εξάγει τα 
  απαραίτητα πεδία:
  \begin{itemize}
    \item το αναγνωριστικό του μετρητή από \en{end\_device\_ids.device\_id},
    \item τις τιμές ανά φάση από τα \en{sensor1}, \en{sensor2}, \en{sensor3},
    \item τη χρονοσήμανση \en{received\_at}.
  \end{itemize}

  \item Οι τιμές αυτές μετατρέπονται σε οντότητες \en{SensorData} (ή αντίστοιχη ονομασία), 
  με πεδία όπως \en{sensorId}, \en{voltage}, \en{current}, \en{power}, \en{energy}, 
  \en{frequency}, \en{powerFactor}, \en{timestamp}, και αποθηκεύονται στη βάση χωρίς 
  απώλεια ακρίβειας.

  \item Ο \en{controller} επιστρέφει μία απλή \en{HTTP 2xx} απόκριση στο \en{TTS}, ώστε ο 
  \en{network server} να θεωρήσει το \en{uplink} παραδοθέν και να μην το επανεκπέμψει.
\end{enumerate}

Με αυτόν τον τρόπο, η \en{Spring Boot} εφαρμογή λειτουργεί ως «γέφυρα» ανάμεσα στο 
\en{LoRaWAN} επίπεδο και στην επίμονη αποθήκευση (\en{PostgreSQL}), παρέχοντας μία 
καθαρή και επεκτάσιμη διεπαφή για κατανάλωση των δεδομένων από το \en{frontend}.

\subsection{REST \en{API} προς την \en{React} εφαρμογή}

Πέρα από τον \en{webhook}, το \en{backend} εκθέτει και ένα σύνολο \en{REST endpoints} για 
ανάγνωση δεδομένων από την \en{React} εφαρμογή. Ενδεικτικά:

\begin{itemize}
  \item \en{POST /api/auth/login}: Δέχεται \en{username}/\en{password}, τα ελέγχει 
  έναντι του πίνακα \en{users} και, σε επιτυχία, επιστρέφει \en{JWT token} που ο 
  φυλλομετρητής αποθηκεύει τοπικά.

  \item \en{GET /api/sensors?from=... \& to=... \& sensorId=...}: Επιστρέφει 
  \en{JSON} λίστα με εγγραφές \en{SensorData} σε επιλεγμένο χρονικό διάστημα για 
  συγκεκριμένο μετρητή (ή για όλες τις φάσεις). Οι εγγραφές αυτές τροφοδοτούν τα 
  γραφήματα στο \en{frontend}.

  \item \en{GET /api/health}: Απλό σημείο ελέγχου κατάστασης (\en{health-check}) που 
  χρησιμοποιείται τόσο για δοκιμές, όσο και για παρακολούθηση της διαθεσιμότητας του 
  \en{backend} από το \en{gateway}.
\end{itemize}

Όλα τα σημαντικά \en{endpoints} (πλην \en{login} και \en{health}) προστατεύονται από 
\en{JWT}-βασισμένη αυθεντικοποίηση: ο \en{client} πρέπει να στείλει κεφαλίδα 
\en{Authorization: Bearer <token>} διαφορετικά λαμβάνει \en{HTTP 401}.

\subsection{Ανάπτυξη του \en{backend} στο \en{LoRaWAN gateway}}

Για την παραγωγική λειτουργία, το \en{backend} πακετάρεται σε \en{fat JAR} (\en{Spring Boot 
executable jar}) και τρέχει στο \en{Raspberry Pi} με \en{Java 17}. Η βάση δεδομένων 
\en{PostgreSQL} μπορεί είτε να τρέχει ως \en{Docker container} στον ίδιο \en{host} (με 
δέσμευση μόνο στη τοπική διεπαφή \en{127.0.0.1}), είτε να φιλοξενείται σε ξεχωριστό 
\en{container} ή μηχάνημα, αρκεί να είναι προσβάσιμη από το \en{backend}. Στην εκκίνηση 
της εφαρμογής, το \en{Liquibase} εκτελεί τα \en{changelogs} και διασφαλίζει ότι το 
σχήμα της βάσης βρίσκεται στην κατάλληλη έκδοση.:contentReference[oaicite:4]{index=4} 

Έτσι, στην ίδια συσκευή \en{gateway} συνυπάρχουν: \en{The Things Stack} (\en{Docker}), 
\en{LoRa Basics Station} (\en{Docker}) και η \en{Spring Boot} \en{REST} υπηρεσία, 
συγκροτώντας μία συμπαγή, αλλά ολοκληρωμένη πλατφόρμα από το επίπεδο του αισθητήρα μέχρι 
την εφαρμογή.


% ------------------------------------------------------
% Ενότητα 6.3 Frontend: React/TypeScript διεπαφή χρήστη
% ------------------------------------------------------

\section{\en{Frontend}: \en{React/TypeScript} διεπαφή χρήστη}

Το \en{frontend} της εφαρμογής υλοποιείται με \en{React} και \en{TypeScript} και 
οργανώνεται ως \en{SPA} με διακριτές σελίδες/οθόνες: \en{Login}, \en{Dashboard} και 
συστατικά (\en{components}) για την αναπαράσταση των μετρήσεων. Για τη γραφική απεικόνιση 
χρησιμοποιείται η βιβλιοθήκη \en{react-chartjs-2} μαζί με \en{Chart.js}, ενώ για την 
επικοινωνία με το \en{backend} χρησιμοποιείται \en{Axios} για \en{HTTP} \en{JSON} κλήσεις.:contentReference[oaicite:5]{index=5} 

\subsection{Δομή και βασικά \en{components}}

Η δομή του \en{frontend} περιλαμβάνει ενδεικτικά τα εξής:

\begin{itemize}
  \item \textbf{\en{App.tsx}}: Κεντρικό \en{component} της εφαρμογής, στο οποίο ορίζονται 
  οι διαδρομές (\en{routes}) προς τη σελίδα \en{Login} και προς το \en{Dashboard} (με 
  προστασία μέσω \en{PrivateRoute} ώστε να απαιτείται έγκυρο \en{JWT}).

  \item \textbf{\en{Login} \en{component}}: Απλή φόρμα με πεδία \en{username}/\en{password}, 
  που καλεί το \en{POST /api/auth/login}. Σε επιτυχία αποθηκεύει το \en{token} σε 
  \en{localStorage} ή \en{sessionStorage} και ανακατευθύνει στον \en{Dashboard}.

  \item \textbf{\en{Dashboard} \en{component}}: Βασική οθόνη εποπτείας. Περιλαμβάνει 
  επιλογείς χρονικού διαστήματος (π.χ. ημερομηνία από/έως), επιλογή μετρητή ή φάσης, 
  κουμπί \en{Refresh} που επαναφέρει τα δεδομένα με νέα κλήση στο \en{backend}, καθώς και 
  ένα ή περισσότερα διαγράμματα/πίνακες.

  \item \textbf{\en{SensorChart.tsx}}: \en{Component} για την απεικόνιση των χρονοσειρών 
  των μετρήσεων. Ανάλογα με την επιλογή του χρήστη, εμφανίζει \en{line}, \en{bar} ή 
  \en{pie} \en{chart}, με άξονα χρόνου (ημερομηνία/ώρα \en{timestamp}) στον οριζόντιο 
  άξονα και την επιλεγμένη φυσική ποσότητα (τάση, ρεύμα, ισχύς, ενέργεια, συχνότητα, 
  συντελεστής ισχύος) στον κατακόρυφο.

  \item \textbf{\en{SensorData.tsx}}: Πίνακας με τις ωμές μετρήσεις, για αναλυτικό 
  έλεγχο. Κάθε γραμμή περιλαμβάνει \en{timestamp}, \en{sensorId} (φάση), τάση, ρεύμα, 
  ισχύ, ενέργεια, συχνότητα και συντελεστή ισχύος.

  \item \textbf{\en{Seperator.tsx}} και βοηθητικά \en{components}: Απλά βοηθητικά στοιχεία 
  για τη διάταξη (\en{layout}) και την οπτική ομαδοποίηση του περιεχομένου.
\end{itemize}

Η δομή των \en{TypeScript interfaces} (π.χ. \en{SensorData}) ευθυγραμμίζεται με το 
\en{JSON} που επιστρέφει το \en{backend}, ώστε ο μετασχηματισμός των δεδομένων να είναι 
ευθύγραμμος: τα πεδία \en{current}, \en{voltage}, \en{power} κ.λπ. προέρχονται απευθείας 
από τις οντότητες της βάσης δεδομένων.

\subsection{Διαχείριση \en{JWT} και κλήσεις προς το \en{REST API}}

Κατά την είσοδο του χρήστη, το \en{Login component} στέλνει \en{HTTP POST} στη 
διαδρομή \en{/api/auth/login}. Αν τα διαπιστευτήρια είναι σωστά, το \en{backend} 
επιστρέφει ένα \en{JWT token} (τυπικά στη μορφή \en{header.payload.signature}). Το 
\en{frontend} αποθηκεύει το \en{token} και, σε κάθε επόμενη κλήση προς προστατευμένο 
\en{endpoint}, προσθέτει στην επικεφαλίδα \en{Authorization} την τιμή 
\en{Bearer <token>}.

Η βιβλιοθήκη \en{Axios} ρυθμίζεται έτσι ώστε να περιλαμβάνει αυτόματα την κεφαλίδα 
σε όλες τις κλήσεις προς \en{/api/...}. Αν το \en{backend} επιστρέψει \en{401 Unauthorized} 
(π.χ. λόγω ληγμένου \en{token}), η εφαρμογή μπορεί είτε να ανακατευθύνει τον χρήστη στη 
σελίδα \en{Login}, είτε να εμφανίσει σχετικό μήνυμα λάθους.

\subsection{Γραφική αναπαράσταση και επιλογές χρήστη}

Η οθόνη \en{Dashboard} δίνει τη δυνατότητα στον χρήστη να:

\begin{itemize}
  \item επιλέξει το χρονικό διάστημα προβολής (π.χ. τελευταία ώρα, τελευταία ημέρα, 
  προσαρμοσμένο διάστημα),
  \item φιλτράρει τα δεδομένα ανά μετρητή ή ανά φάση (π.χ. μόνο \en{sensor1} ή όλες οι 
  φάσεις ταυτόχρονα),
  \item επιλέξει ποια φυσική ποσότητα θα απεικονιστεί (τάση, ρεύμα, ισχύς, ενέργεια, 
  συχνότητα, \en{power factor}),
  \item αλλάξει τύπο διαγράμματος (\en{line}, \en{bar}, \en{pie}) ανάλογα με την 
  πληροφορία που θέλει να αναδείξει,
  \item επαναφορτώσει τα δεδομένα με κουμπί \en{Refresh}, ώστε να εμφανιστούν τα πιο 
  πρόσφατα \en{uplinks}.
\end{itemize}

Στις επόμενες υποενότητες (που ακολουθούν στο υπόλοιπο του κεφαλαίου) παρατίθενται 
στιγμιότυπα (\en{screenshots}) του \en{UI} της εφαρμογής (σε μορφή Εικόνων), όπου 
επεξηγούνται αναλυτικά οι επιλογές, οι αλληλεπιδράσεις του χρήστη και παραδείγματα 
πραγματικών δεδομένων από το σύστημα τριφασικής μέτρησης.
