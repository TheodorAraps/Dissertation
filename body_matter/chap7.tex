\chapter{Συμπεράσματα και Μελλοντικές Κατευθύνσεις}

\section{Συμπεράσματα}
Η παρούσα εργασία υλοποίησε ένα ολοκληρωμένο σύστημα τριφασικής μέτρησης και απομακρυσμένης εποπτείας ενέργειας, βασισμένο στο οικοσύστημα \en{LoRa/LoRaWAN}, με \en{gateway} τύπου \en{Raspberry Pi 4 Model B} και συγκεντρωτή \en{iC880A-SPI}, ενώ ως \en{Network Server} αξιοποιήθηκε το \en{The Things Stack} και ως \en{packet forwarder} το \en{LoRa Basics Station}. Στο ανώτερο επίπεδο της λύσης αναπτύχθηκε πλήρης \en{web} εφαρμογή με \en{Spring Boot}/\en{PostgreSQL} για συλλογή και επίμονη αποθήκευση μετρήσεων και \en{React} \en{frontend} για αλληλεπιδραστική οπτικοποίηση. Η αρχική ερευνητική στοχοθεσία αφορούσε την πρακτική διερεύνηση της αξιοπιστίας, της επεκτασιμότητας και της χρηστικότητας του \en{LoRaWAN} σε περιβάλλοντα υποσταθμών. Τα αποτελέσματα επιβεβαιώνουν ότι, με προσεκτική αρχιτεκτονική και ορθή παραμετροποίηση, το πρωτόκολλο και τα συνοδευτικά εργαλεία ανταποκρίνονται επαρκώς στις απαιτήσεις που τέθηκαν.

Καθοριστική συνιστώσα της επιτυχίας υπήρξε η ενοποίηση όλης της στοίβας στο ίδιο \en{gateway}. Η ενορχήστρωση \en{The Things Stack}, \en{LoRa Basics Station} και της εφαρμογής παρακολούθησης μέσω \en{Docker/Compose}, πλαισιωμένη από \en{systemd} \en{units} για αυτόματη εκκίνηση, οδήγησε σε μια λειτουργική «\en{one-touch}» υλοποίηση: το σύστημα εκκινεί, συγχρονίζει τις απαραίτητες υπηρεσίες, δέχεται \en{uplinks}, τα δρομολογεί στο \en{webhook} του \en{backend} και προσφέρει άμεσα διαθέσιμα διαγράμματα στο \en{dashboard}. Η προσέγγιση αυτή απλοποιεί τη διάθεση (\en{deployment}) σε απομακρυσμένα σημεία, μειώνει τον λειτουργικό φόρτο και καθιστά την εγκατάσταση επαναλήψιμη με ελάχιστα βήματα.

Επιπλέον, η ροή δεδομένων \textit{άκρου→δικτύου→εφαρμογής} αποδείχθηκε αξιόπιστη. Η διασύνδεση \en{HTTP webhook} επέτρεψε καθαρή αποσύνδεση της δικτυακής στοίβας από τη λογική της εφαρμογής, προσφέροντας ιχνηλασιμότητα ανά \en{uplink}, σαφή χειρισμό σφαλμάτων και απλή επαλήθευση της παραλαβής. Η επίμονη αποθήκευση σε \en{PostgreSQL} με κατάλληλο σχήμα και η \en{REST} διεπαφή ανάκτησης δεδομένων διευκόλυναν την παραγωγή διαδραστικών γραφημάτων (π.χ. ρεύμα, τάση, ισχύς, ενέργεια, συχνότητα) με φίλτρα σε μετρητή, μέγεθος και χρονικό διάστημα. Η συνολική εμπειρία χρήσης στο \en{frontend} κατέδειξε ότι, ακόμη και σε μη ιδανικές συνθήκες δικτύου, η λύση παρέχει συνεκτική εικόνα λειτουργίας του συστήματος σε σχεδόν πραγματικό χρόνο.

Ιδιαίτερη αξία είχαν οι εργαστηριακές δοκιμές με ωμικά φορτία (λαμπτήρες πυρακτώσεως), όπου καταγράφηκαν συντελεστές ισχύος κοντά στη μονάδα και τιμές ισχύος που συμφωνούν με τη θεωρητική προσέγγιση 
\(P \approx V \cdot I\), λαμβάνοντας υπόψη τις μικρές αποκλίσεις τάσης και τις ανοχές του εξοπλισμού. Οι μετρήσεις ενέργειας επιβεβαίωσαν επίσης τη συνέπεια του συστήματος ως προς τη χρονική συσσώρευση (\en{kWh}). Μέσα από αυτήν τη διαδικασία, αξιολογήθηκε πρακτικά η ακρίβεια του μετρητικού υποσυστήματος και η ακεραιότητα του πακεταρίσματος/μεταφοράς των δεδομένων μέχρι την τελική απεικόνιση.

Η διαδικασία εγκατάστασης τεκμηριώθηκε πλήρως: από τη δημιουργία \en{headless} εικόνας του \en{Raspberry Pi OS}, την ενεργοποίηση της διεπαφής \en{SPI} και τη συνδεσμολογία με τον συγκεντρωτή, μέχρι την κλωνοποίηση των αποθετηρίων, την προσαρμογή \en{docker-compose.yml} και τη ρύθμιση των \en{systemd} \en{units} για εκκίνηση/τερματισμό του συνόλου. Η τεκμηρίωση αυτή αποτελεί πρακτικό οδηγό αναπαραγωγής της λύσης και μειώνει το «γνωσιακό χρέος» για μελλοντικούς συντηρητές ή επεκτάτες.

Ωστόσο, υπάρχουν και ορισμένοι περιορισμοί που πρέπει να αναγνωριστούν. Η θεμελιώδης ανταλλαγή στο \en{LoRaWAN} ανάμεσα σε ρυθμό μετάδοσης, εμβέλεια και \en{duty-cycle} συνεπάγεται ότι οι παράμετροι \en{SF/DR} πρέπει να επιλέγονται με προσοχή ανάλογα με το περιβάλλον, την πυκνότητα κόμβων και τις απαιτήσεις \en{latency}. Οι δοκιμές πραγματοποιήθηκαν με περιορισμένο αριθμό μετρητών και με ωμικά φορτία, άρα το φάσμα παρεμβολών και ο θόρυβος ενός «βαρέος» βιομηχανικού περιβάλλοντος δεν διερευνήθηκαν πλήρως. Τέλος, η συνειδητή επιλογή συγκέντρωσης όλων των υπηρεσιών στον ίδιο κόμβο (\en{single-node hosting}) διευκολύνει την εγκατάσταση αλλά δεν αποτυπώνει σενάρια υψηλής διαθεσιμότητας ή οριζόντιας κλιμάκωσης.

Συνοψίζοντας, η εργασία αποδεικνύει ότι μια στοχευμένη, ανοιχτού κώδικα στοίβα \en{LoRaWAN} μπορεί να υλοποιήσει μια πρακτική λύση «άκρο–σε–άκρο» για παρακολούθηση ενεργειακών μεγεθών, με χαμηλό κόστος, υψηλή επαναληψιμότητα εγκατάστασης και επαρκή αξιοπιστία για τις ανάγκες εποπτείας υποσταθμών και παρόμοιων βιομηχανικών εφαρμογών.

\section{Μελλοντικές κατευθύνσεις}
Με βάση τα παραπάνω ευρήματα, διαφαίνονται πολλαπλές επεκτάσεις που μπορούν να ωριμάσουν περαιτέρω τη λύση. Πρώτη προτεραιότητα είναι η κλιμάκωση και η ανθεκτικότητα. Η μεταφορά του \en{The Things Stack} και της βάσης δεδομένων σε ξεχωριστούς κόμβους, η υιοθέτηση \en{Docker Swarm} ή \en{Kubernetes} και η εισαγωγή αντιγράφων (\en{replicas}) για την \en{PostgreSQL} με στρατηγικές \en{backup/restore} θα επιτρέψουν ορισμό στόχων \en{RPO/RTO} και επιχειρησιακή συνέχεια σε περιπτώσεις αστοχίας. Παράλληλα, η ενσωμάτωση παρακολούθησης πόρων και εφαρμογών με \en{Prometheus/Grafana} θα προσφέρει ορατότητα, έγκαιρα προειδοποιητικά σήματα και μετρήσιμη βάση για βελτιστοποίηση.

Σε επίπεδο ραδιοζεύξης, αξίζει μια συστηματικότερη μελέτη ποιότητας: συλλογή και ανάλυση \en{RSSI}/\en{SNR} σε ποικίλες γεωμετρίες εγκατάστασης, δοκιμές με διαφορετικά μήκη/τύπους ομοαξονικών καλωδίων και κεραιών, καθώς και πειράματα με υψομετρικές διαφοροποιήσεις. Η αξιοποίηση του \en{ADR} για δυναμική προσαρμογή του ρυθμού δεδομένων μπορεί να βελτιώσει τον λόγο επιτυχών \en{uplinks} σε περιβάλλοντα με άνισο ραδιο-/φορτίο. Στο άκρο της συσκευής, η χρήση τεχνικών συμπίεσης ή αποδοτικότερων σχημάτων κωδικοποίησης τηλεμετρίας θα μείωνε τον χρόνο αέρα (\en{airtime}) και τη συνολική κατανάλωση.

Ως προς την ασφάλεια, η λύση μπορεί να σκληρυνθεί «από άκρο σε άκρο». Η χρήση έγκυρων δημόσιων πιστοποιητικών \en{TLS} και η αυστηρή ρύθμιση \en{HTTPS/WSS} σε όλες τις διεπαφές, οι πολιτικές ελαχιστοποίησης δικαιωμάτων (π.χ. ξεχωριστοί χρήστες/ρόλοι στη βάση, περιορισμένα \en{scopes} στο \en{TTS}), καθώς και η τακτική περιστροφή/ανανέωση μυστικών αυξάνουν την εμπιστοσύνη του συστήματος. Στο \en{web} επίπεδο, πρακτικές όπως \en{HSTS}, προστασία από \en{CSRF} και έλεγχος ισχυρών \en{CORS} πολιτικών θωρακίζουν την πρόσβαση, ενώ ένα ελαφρύ \en{WAF} θα βοηθούσε στην αποτροπή κοινών επιθέσεων.

Σε ό,τι αφορά την αναλυτική αξιοποίηση των δεδομένων, η ενσωμάτωση καναλιών επεξεργασίας (\en{pipelines}) για ανίχνευση ανωμαλιών και πρόβλεψη ζήτησης δύναται να μετατρέψει το σύστημα από απλή εποπτεία σε εργαλείο προληπτικής συντήρησης. Μέθοδοι όπως αποσύνθεση χρονοσειρών (\en{STL}) ή απλά μοντέλα πρόβλεψης, σε συνδυασμό με κανόνες ενεργοποίησης ειδοποιήσεων, μπορούν να εντοπίζουν εγκαίρως ασυνήθιστες συμπεριφορές, διακυμάνσεις συχνότητας ή απότομες αλλαγές συντελεστή ισχύος. Η επέκταση του σχήματος δεδομένων ώστε να συνενώνει μετρήσεις από περισσότερους αισθητήρες (π.χ. θερμοκρασίας, κραδασμών) διευρύνει τον ορίζοντα χρήσης σε περιβάλλοντα όπου απαιτείται πολυπαραμετρική εποπτεία.

Τέλος, σε επίπεδο συμμόρφωσης και εναρμόνισης με πρότυπα, μελλοντική εργασία μπορεί να συμπεριλάβει δοκιμές σύμφωνα με οδηγίες \en{EMC}/ασφάλειας και διερεύνηση χαρακτηριστικών \en{LoRaWAN 1.1/1.0.4} που σχετίζονται με \en{roaming} ή ιδιωτικές διασυνδέσεις (\en{peering}) δικτύων. Η διερεύνηση αυτών των δυνατοτήτων θα καταστήσει τη λύση πιο «βιομηχανικά ώριμη» και έτοιμη για υιοθέτηση σε μεγαλύτερες εγκαταστάσεις, με απαιτήσεις διαλειτουργικότητας και αυστηρά \en{SLA}.

Συνολικά, η εργασία απέδειξε τη βιωσιμότητα μιας οικονομικής, ανοιχτού κώδικα λύσης \en{LoRaWAN} για μέτρηση και εποπτεία ισχύος. Η προτεινόμενη πορεία εξέλιξης περιλαμβάνει επιχειρησιακή ωρίμανση (κλιμάκωση, ανθεκτικότητα, ασφάλεια), τεχνική βελτίωση της ζεύξης (προσαρμοστικότητα και μετρήσεις ποιότητας) και εμπλουτισμό του πληροφοριακού επιπέδου (ανάλυση/πρόβλεψη, ειδοποιήσεις), έτσι ώστε το σύστημα να μεταβεί από πιλότο επίδειξης σε παραγωγική υποδομή μετρήσεων σε πραγματικές συνθήκες.