\chapter{Υλικό και Σχεδιασμός Συσκευών του Συστήματος}
\InitialCharacter{Σ}ε αυτό το κεφάλαιο περιγράφεται η υλική υποδομή που 
κατασκευάστηκε για την υλοποίηση του συστήματος: το \textbf{\en{LoRaWAN gateway}}  
και οι δύο \textbf{τριφασικοί μετρητές}. Δίνεται έμφαση στο σχεδιασμό, στα υλικά και τα 
χαρακτηριστικά τους, καθώς και στη διασύνδεση και στις τεχνικές επιλογές.

\section{Γενική Αρχιτεκτονική Συσκευών}

Πριν από την περιγραφή κάθε συσκευής, αξίζει να παρουσιαστεί συνοπτικά η τοπολογία: 
οι αισθητήρες (μετρητές) επικοινωνούν ασύρματα μέσω \en{LoRaWAN} με το \en{gateway}, το 
οποίο συλλέγει τα πακέτα και τα προωθεί (\en{forwarding}) προς τον \en{LoRaWAN Network Server}. 
Από εκεί, τα δεδομένα διατρέχουν το \en{backend} μέχρι την τελική αποθήκευση και παρουσίαση 
στο \en{frontend} της εφαρμογής μας.

\begin{Illustration}[!ht] 
  \centering
	\includegraphics[width=1\textwidth]{figures/My-System-Arcitecture.png} 
  \caption{Τοπολογία του συστήματος.}
  \label{figure4.1}
\end{Illustration} 

\section{\en{LoRaWAN Gateway}}