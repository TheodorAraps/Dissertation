\chapter{Υλικό και Σχεδιασμός Συσκευών του Συστήματος}

\InitialCharacter{Σ}ε αυτό το κεφάλαιο περιγράφεται η υλική υποδομή που 
κατασκευάστηκε για την υλοποίηση του συστήματος, το \textbf{\en{LoRaWAN gateway}}  
και οι δύο \textbf{τριφασικοί μετρητές}. Δίνεται έμφαση στον σχεδιασμό, στα υλικά και στα 
χαρακτηριστικά τους, καθώς και στη διασύνδεση και στις τεχνικές επιλογές.


% ---------------------------------------------
% Ενότητα 4.1: Γενική Αρχιτεκτονική Συστήματος
% ---------------------------------------------



\section{Γενική Αρχιτεκτονική Συστήματος}

Πριν από την περιγραφή κάθε συσκευής, αξίζει να παρουσιαστεί συνοπτικά η τοπολογία: 
οι αισθητήρες (μετρητές) επικοινωνούν ασύρματα μέσω \en{LoRaWAN} με το \en{gateway}, το 
οποίο συλλέγει τα πακέτα και τα προωθεί (\en{forwarding}) προς τον \en{LoRaWAN Network Server}. 
Από εκεί, τα δεδομένα διατρέχουν το \en{backend} μέχρι την τελική αποθήκευση και παρουσίαση 
στο \en{frontend} της εφαρμογής μας.

\begin{Illustration}[!ht] 
  \centering
	\includegraphics[width=1\textwidth]{figures/My-System-Arcitecture.png} 
  \caption{Τοπολογία του συστήματος.}
  \label{figure4.1}
\end{Illustration} 



% -----------------------------
% Ενότητα 4.2: LoRaWAN Gateway
% -----------------------------



\section{\en{LoRaWAN Gateway}}
\label{sec:4.2}


%%%%   Υποενότητα 4.2.1: Υλικά και κατασκευή   %%%%


\subsection{Υλικά και κατασκευή}

Για την κατασκευή του \en{gateway} χρησιμοποιήθηκε ένα \textbf{\en{Raspberry Pi 4 Model B}} σε συνδυασμό με τη μονάδα συγκεντρωτή
(\en{concentrator board}) \textbf{\en{iC880A-SPI}} της \en{IMST}. Σύμφωνα με τον οδηγό της \en{The Things Industries} \cite{TTI_RPiGatewayDocs}, 
αυτές οι συνθέσεις υποστηρίζονται ώστε να κατασκευάζει κανείς ένα προσωπικό \en{gateway} χρησιμοποιώντας τα προαναφερθέντα εξαρτήματα.

Ο ακόλουθος πίνακας συνοψίζει τα βασικά υλικά που χρησιμοποιήθηκαν:

\begin{table}[ht]
\centering
\setlength{\tabcolsep}{10pt}        
\renewcommand{\arraystretch}{1.5}   
\begin{tabular}{| >{\raggedright\arraybackslash}p{0.38\linewidth}
                | >{\raggedright\arraybackslash}p{0.52\linewidth} |}
\hline
\textbf{Συσκευή / Στοιχείο} & \textbf{Σκοπός / Σχόλιο} \\
\hline
1$\times$ \en{Raspberry Pi 4 Model B} & Κεντρικός υπολογιστής και \en{host} λογισμικού \\
\hline
1$\times$ \en{Concentrator board iC880A-SPI} & Διαχείριση πακέτων \en{LoRa} σε πολλαπλά κανάλια \\
\hline
1$\times$ Κεραία 868$MHz$ 2$dBi$ & Εκπομπή / λήψη ραδιοσημάτων \\
\hline
1$\times$ \en{Pigtail} καλώδιο & Σύνδεση κεραίας με \en{iC880A-SPI} \\
\hline
7$\times$ \en{Jumper wires, dual female} & Σύνδεση \en{SPI, reset, power lines} \\
\hline
1$\times$ Κάρτα \en{microSD, 64GB} & Λειτουργικό σύστημα και λογισμικό \\
\hline
1$\times$ Τροφοδοσία (5$V$, $\geq$ 3$A$) & Παροχή, αναγκαία για σταθερή λειτουργία \\
\hline
\end{tabular}
\caption{Βασικά υλικά για το \en{LoRaWAN Gateway}.}
\label{tab:gateway_components}
\end{table}


%%%%   Raspberry Pi 4 Model B   %%%%


\subsubsection{\en{Raspberry Pi 4 Model B}}

Το \en{Raspberry Pi 4 Model B} (βλπ Εικόνα \ref{figure4.2}) αποτελεί μια ευέλικτη και ευρύτατα διαδεδομένη πλατφόρμα μικροϋπολογιστή 
(\en{single-board computer}) που χρησιμοποιείται ευρέως σε έργα \en{IoT}, αυτοματισμού και πρωτότυπης ανάπτυξης. Στηρίζεται σε 
ανοικτό λογισμικό και κοινοτικά πρότυπα, επιτρέποντας τη χρήση του σε πειραματικά ή παραγωγικά έργα με 
ευελιξία και χαμηλό κόστος. Υποστηρίζει πλήρες \en{Linux} οικοσύστημα και συνοδεύεται από μια πολύ ενεργή κοινότητα, γεγονός που το καθιστά 
ιδιαίτερα ελκυστικό για μαθητές και επαγγελματίες που θέλουν να υλοποιήσουν έργα \en{IoT}, 
αυτοματισμού ή γρήγορου \en{prototyping} \cite{RaspberryPi4ProductBrief2025}. Στο πλαίσιο της 
παρούσας υλοποίησης, το \en{Raspberry Pi} 4 χρησιμεύει ως κόμβος που «υποδέχεται» και διαχειρίζεται το λογισμικό 
του \en{gateway}, συμβάλλοντας καθοριστικά στην αξιοπιστία και επεκτασιμότητα του συστήματος. Επιπλέον, αποτελεί και το \en{host} μηχάνημα
όπου θα φιλοξενηθούν τόσο το λογισμικό του \en{LoRaWAN network server} όσο και η τελική εφαρμογή του χρήστη.



%%%%   LoRaWAN Concentrator board iC880A-SPI   %%%%


\subsubsection{\en{LoRaWAN Concentrator board iC880A-SPI}}

Το \en{iC880A-SPI} (βλπ Εικόνα \ref{figure4.2}) είναι συγκεντρωτής \en{LoRaWAN} που χρησιμοποιείται ως το κύριο στοιχείο ενός \en{gateway}, 
δηλαδή ως πλήρες \en{RF frontend} που δέχεται ταυτόχρονα πολλαπλά \en{LoRa} πακέτα σε διαφορετικά κανάλια 
και προωθεί την κίνηση προς τον \en{network server}. Τοποθετείται συνήθως πάνω σε \en{single-board} πλακέτες 
όπως το \en{Raspberry Pi} και έχει καθιερωθεί σε \en{DIY} και εργαστηριακές υλοποιήσεις λόγω της ευρείας 
τεκμηρίωσης και της συμβατότητάς του με σύγχρονα λογισμικά \en{gateway} όπως το \en{LoRa Basics Station}. 
Με αυτό τον τρόπο επιτρέπει την κατασκευή οικονομικών αλλά αξιόπιστων \en{LoRaWAN gateways} που εντάσσονται 
εύκολα σε δίκτυα όπως το \en{The Things Stack} \cite{iC880A_Datasheet_V0_50}.

\begin{table}[H]
\centering
\renewcommand{\arraystretch}{1.2}
\begin{tabular}{|l|p{0.72\linewidth}|}
\hline
\textbf{Τεχνικά Δεδομένα} & \textbf{Περιγραφή} \\
\hline
Επεξεργαστής &
\en{Broadcom BCM2711 quad-core Cortex-A72 (ARM v8) 64-bit SoC @ 1.5GHz} \\
\hline
Μνήμη & \en{2}$GB$ \\
\hline
Συνδεσιμότητα &
\en{2.4}$GHz$ \en{and 5.0}$GHz$ \en{IEEE 802.11b/g/n/ac wireless LAN, Bluetooth 5.0, BLE} \\
& \en{Gigabit Ethernet} \\
& \en{2} $\times$ \en{USB 3.0 ports, 2}$\times$ \en{USB 2.0 ports} \\
\hline
Είσοδοι/ Έξοδοι &
\en{Standard 40-pin GPIO header} \\
\hline
Ήχος και Εικόνα &
\en{2} $\times$ \en{micro HDMI ports (up to 4Kp60 supported)} \\
& \en{2-lane MIPI DSI display port} \\
& \en{2-lane MIPI CSI camera port} \\
& \en{4-pole stereo audio and composite video port} \\
\hline
Πολυμέσα &
\en{H.265 (4Kp60 decode)} \\
& \en{H.264 (1080p60 decode, 1080p30 encode)} \\
& \en{OpenGL ES, 3.0 graphics} \\
\hline
Υποστήριξη κάρτας \en{SD} &
\en{Micro SD card slot for loading operating system and data storage} \\
\hline
Ισχύς Εισόδου &
\en{5}$V$ \en{DC via USB-C connector} (\en{minimum} 3$A$) \\
& \en{5}$V$ \en{DC via GPIO header} (\en{minimum} 3$A$) \\
& \en{Power over Ethernet (PoE)-enabled (requires separate PoE HAT)} \\
\hline
Περιβάλλον λειτουργίας &
\en{Operating temperature 0-50\textdegree C} \\
\hline
\end{tabular}
\caption{Τεχνικά χαρακτηριστικά \en{Raspberry Pi 4 Model B}.}
\cite{RaspberryPi4ProductBrief2025}
\end{table}


\begin{table}[ht]
\centering
\setlength{\tabcolsep}{10pt}
\renewcommand{\arraystretch}{1.25}
\begin{tabular}{| >{\raggedright\arraybackslash}p{0.3\linewidth}
                | >{\raggedright\arraybackslash}p{0.65\linewidth} |}
\hline
\textbf{Τεχνικά Δεδομένα} & \textbf{Περιγραφή} \\
\hline
Βασικό \en{Chipset} & \en{Semtech SX1301 baseband processor} + 2$\times$ \en{SX1257 RF transceivers} \\
\hline
Συχνότητες & \en{EU868} (διαθέσιμη έκδοση και για \en{US915}) \\
\hline
Κανάλια & 8$\times$ \en{LoRa channels} ταυτόχρονα + 1$\times$ \en{FSK channel} \\
\hline
Διεπαφή προς \en{Host} & \en{SPI} (3.3$V$ \en{logic level}) \\
\hline
Χρονισμός & Είσοδος \en{PPS} από \en{GPS} για συγχρονισμό \\
\hline
Σύνδεση Κεραίας & Υποδοχή \en{SMA} 50Ω \\
\hline
Τάση / Ισχύς & 5$V$ μέσω ακροδεκτών (κατανάλωση $\thicksim$ 700$mW$) \\
\hline
Εφαρμογή & Συγκεντρωτής (\en{concentrator}) για \en{LoRaWAN gateways} \\
\hline
Διαστάσεις & 80$mm$ $\times$ 50$mm$ περίπου \\
\hline
\end{tabular}
\caption{Τεχνικά χαρακτηριστικά του \en{iC880A-SPI LoRaWAN Concentrator}.}
\label{tab:ic880a_specs}
\cite{iC880A_Datasheet_V0_50}
\end{table}


%%%%   Υποενότητα 4.2.2: Συνδεσμολογία και λειτουργία   %%%%

\subsection{Συνδεσμολογία και λειτουργία}

Η συνδεσμολογία μεταξύ \en{Raspberry Pi} και συγκεντρωτή \en{IC880A-SPI} ακολουθεί τις οδηγίες που παρέχει η 
\en{The Things Industries}: το \en{board} συνδέεται μέσω \en{SPI pins (MOSI, MISO, SCLK, NSS)} καθώς και 
\en{reset}, 5$V$ και γείωσης (\en{GND}). Η συνδεσμολογία φαίνεται στην Εικόνα \ref{figure4.2} και περιγράφεται 
σε επίπεδο ακίδων (\en{pins}) στον Πίνακα \ref{tab:ic880a_rpi_pins}.

Πριν την ενεργοποίηση, είναι απαραίτητο να συνδεθεί η κεραία στο \en{concentrator}, προκειμένου να αποφευχθεί η 
ανάκλαση ισχύος που θα μπορούσε να βλάψει το \en{hardware}.

Το λογισμικό του \en{gateway} (\en{LoRa Basics Station}) εκτελείται στο 
\en{Raspberry Pi} και επικοινωνεί με τον συγκεντρωτή, στέλνοντας \en{uplink} μηνύματα προς τον 
\en{TTS} και λαμβάνοντας τυχόν \en{downlink}. Στη συνέχεια τα δεδομένα αυτά στέλνονται και παρουσιάζονται στην εφαρμογή μας. 
Οι υπηρεσίες αυτές θα μπορούσαν κάλλιστα να φιλοξενούνται σε διαφορετικές συσκευές (\en{servers}), εντούτοις στη δική μας 
υλοποίηση επιλέχθηκε να φιλοξενηθούν με χρήση του \en{Docker} (βλπ Ενότητα \ref{section3.3}) όλες στο ίδιο μηχάνημα 
(\en{LoRaWAN gateway}) για λόγους α\-πλο\-ποί\-η\-σης και εξοικονόμησης υλικού. \\


\begin{Illustration}[!ht] 
  \centering
	\includegraphics[width=1\textwidth]{figures/1st_Implementation_gateway.png} 
  \caption{Υλοποίηση \en{LoRaWAN Gateway} και σύνδεσμολογία εξαρτημάτων.}
  \label{figure4.2}
\end{Illustration} 

\begin{table}[ht]
\centering
\setlength{\tabcolsep}{6pt}        
\renewcommand{\arraystretch}{1.4}   
\begin{tabular}{| >{\centering\arraybackslash}m{0.22\linewidth}
                | >{\centering\arraybackslash}m{0.22\linewidth}
                | >{\centering\arraybackslash}m{0.22\linewidth} |}
\hline
\textbf{\en{iC880A pin}} & \textbf{\en{Raspberry Pi pin}} & \textbf{\en{Description}} \\
\hline
21 & 2  & 5$V$ \en{power supply} \\
\hline
22 & 6  & \en{GND} \\
\hline
13 & 22 & \en{Reset} \\
\hline
14 & 23 & \en{SPI Clock} \\
\hline
15 & 21 & \en{MISO} \\
\hline
16 & 19 & \en{MOSI} \\
\hline
17 & 24 & \en{NSS} \\
\hline
\end{tabular}
\caption{Συνδεσομολογία ακίδων (\en{pins}) \en{iC880A-SPI} με \en{Raspberry Pi 4 Model B}.}
\label{tab:ic880a_rpi_pins}
\end{table}



% --------------------------------
% Ενότητα 4.3 Τριφασικοί Μετρητές
% --------------------------------




\newpage
\section{Τριφασικοί Μετρητές}


%%%%   Υποενότητα 4.3.1: Υλικά και αρχιτεκτονική   %%%%

\subsection{Υλικά και αρχιτεκτονική}

Οι τριφασικοί μετρητές που κατασκευάστηκαν έχουν ως στόχο την καταγραφή ηλεκτρικών 
μεγεθών ανά φάση και τη μετάδοση των μετρήσεων μέσω \en{LoRaWAN} προς το \en{gateway}. 
Κάθε μετρητής αποτελείται από μία πλακέτα \textbf{\en{The Things Uno}} ως κεντρικό 
μικροελεγκτή και \en{LoRaWAN} πομπό, καθώς και από τρεις αισθητήρες \textbf{\en{PZEM-004T V3.0}} 
(έναν για κάθε φάση). Οι \en{PZEM-004T} μετρούν τάση, ρεύμα, ενεργό ισχύ, ενέργεια, συντελεστή ισχύος 
και συχνότητα, εκθέτοντας τις τιμές μέσω σειριακής διεπαφής \en{TTL/Modbus-RTU}. Οι μετρήσεις συλλέγονται 
από τη \en{The Things Uno} και αποστέλλονται ασύρματα μέσω \en{LoRaWAN} στο δίκτυο.

Ο ακόλουθος πίνακας συνοψίζει τα βασικά υλικά που χρησιμοποιήθηκαν για την κατασκευή ενός τριφασικού μετρητή. 
Για την παρούσα υλοποίηση κατασκευάστηκαν δύο όμοιες μονάδες.

\begin{table}[ht]
\centering
\setlength{\tabcolsep}{8pt}
\renewcommand{\arraystretch}{1.35}
\begin{tabular}{|>{\raggedright\arraybackslash}p{0.35\linewidth}|>{\raggedright\arraybackslash}p{0.55\linewidth}|}
\hline
\textbf{Συσκευή / Στοιχείο} & \textbf{Σκοπός / Σχόλιο} \\
\hline
1 $\times$ \en{The Things Uno} & Κεντρικός μικροελεγκτής \en{ATmega32U4} με \en{LoRaWAN} μονάδα \en{Microchip RN2483/RN2903}. \\
\hline
3 $\times$ \en{PZEM-004T V3.0} & Μετρητική μονάδα \en{AC} ανά φάση με έξοδο \en{TTL/Modbus-RTU}. \\
\hline
3 $\times$ \en{Current Transformers (CT)} & Δακτύλιοι ρεύματος κατάλληλοι για τη σειρά \en{PZEM-004T V3.0} (π.χ. 100$A$). \\
\hline
3 $\times$ Καλωδιώσεις \en{TTL} και ισχύος & Σύνδεση \en{TX/RX/GND/5}$V$ προς τους \en{PZEM} και τροφοδοσία πλακετών. \\
\hline
% Κέλυφος και απομόνωση & Μηχανική προστασία και ασφαλής διέλευση αγωγών φάσεων και \en{CT}. \\
% \hline
\end{tabular}
\caption{Κατάλογος υλικών για έναν τριφασικό μετρητή}
\label{tab:meter_bom}
\end{table}


\subsubsection{\en{PZEM-004T V3.0}}

Ο \en{PZEM-004T V3.0} (βλπ Εικόνα \ref{figure4.3}) είναι ένας πολυλειτουργικός ψηφιακός μετρητής εναλλασσόμενης ενέργειας \en{(Alternating Current, AC)}, σχεδιασμένος για 
τη μέτρηση τάσης, ρεύματος, ενεργού ισχύος, συχνότητας, συντελεστή ισχύος και της συσσωρευμένης κατανάλωσης 
ενέργειας σε μονοφασικά συστήματα εναλλασσόμενου ρεύματος. Εκθέτει τις τιμές μέσω \en{TTL} με 
\en{Modbus-RTU} σε 9600$bps$ εξ ορισμού. Υπάρχουν παραλλαγές για 10$A$ (με εσωτερικό \en{shunt}) και 100$A$ 
(με εξωτερικό \en{CT}) \cite{PZEM004T_Datasheet}. 

\begin{table}[ht]
\centering
\setlength{\tabcolsep}{8pt}
\renewcommand{\arraystretch}{1.3}
\begin{tabular}{|>{\raggedright\arraybackslash}p{0.36\linewidth}|>{\raggedright\arraybackslash}p{0.54\linewidth}|}
\hline
\textbf{Παράμετρος} & \textbf{Ενδεικτικές προδιαγραφές \en{V3.0}} \\
\hline
Εύρος τάσης & 80-260$VAC$ \\
\hline
Εύρος ρεύματος & 0-100$A$ με εξωτερικό \en{CT} \\
\hline
Μετρούμενες ποσότητες & Τάση, ρεύμα, ενεργός ισχύς, ενέργεια, συχνότητα, \en{power factor} \\
\hline
Διεπαφή επικοινωνίας & \en{TTL/Modbus-RTU}, 9600$bps$ προεπιλογή \\
\hline
Τροφοδοσία & Από την πλευρά της τάσης \en{AC} εισόδου \\
\hline
Τυπική ακρίβεια & $\pm$0.5-1\% για βασικές μετρήσεις (ενδεικτικό, ανά τεκμηρίωση \en{V3.0}) \\
\hline
\end{tabular}
\caption{Τεχνικά χαρακτηριστικά \en{PZEM-004T V3.0}}
\label{tab:pzem_specs}
\cite{PZEM004T_Datasheet}
\end{table}

\subsubsection{\en{The Things Uno}}

Η \en{The Things Uno} (βλπ Εικόνα \ref{figure4.3}) της εταιρείας \en{The Things Industries} βασίζεται στο \en{Arduino Leonardo} με μικροελεγκτή \en{ATmega32U4} και ενσωματωμένη μονάδα 
\en{LoRaWAN} \en{Microchip RN2483} (για \en{EU868}) ή \en{RN2903} (για \en{US915}). Είναι πλήρως συμβατή με το 
\en{Arduino IDE} και τις υπάρχουσες \en{Arduino shields}, ενώ στο επίπεδο λογισμικού η επικοινωνία με τη μονάδα 
\en{LoRa} γίνεται μέσω της \en{Serial1}. Η πλατφόρμα προσφέρεται για γρήγορο \en{prototyping} και εύκολη ένταξη 
με το λογισμικό \en{The Things Stack} \cite{TTN_Uno_Docs}.

\begin{table}[ht]
\centering
\setlength{\tabcolsep}{8pt}
\renewcommand{\arraystretch}{1.3}
\begin{tabular}{|>{\raggedright\arraybackslash}p{0.36\linewidth}|>{\raggedright\arraybackslash}p{0.54\linewidth}|}
\hline
\textbf{Στοιχείο} & \textbf{Περιγραφή} \\
\hline
Κεντρικός \en{MCU} & \en{ATmega32U4} (\en{Arduino Leonardo-compatible}) \\
\hline
\en{LoRaWAN} μονάδα & \en{Microchip RN2483} (\en{EU868}) ή \en{RN2903} (\en{US915}) \\
\hline
Προγραμματισμός & \en{Arduino IDE}, \en{TheThingsNetwork} βιβλιοθήκη, \en{Serial1} προς \en{RN2483/RN2903} \\
\hline
Διεπαφές & Ψηφιακές \en{I/O}, \en{UART}, \en{SPI}, \en{I2C}, \en{USB} \\
\hline
Τροφοδοσία & 7-9$V$ \en{DC power adapter} με 2.1$mm$ \en{jack plug} ή 5$V$ \en{micro USB} \\
\hline
Ενδεικτική εμβέλεια & Μέχρι και 10 χιλιόμετρα, ανάλογα με περιβάλλον και κεραία \\
\hline
\end{tabular}
\caption{Τεχνικά χαρακτηριστικά του \en{The Things Uno}}
\label{tab:things_uno_specs}
\end{table}


%%%%   Υποενότητα 4.3.2: Υλικά και αρχιτεκτονική   %%%%

\subsection{Συνδεσμολογία και λειτουργία}

Η \en{The Things Uno} συλλέγει τις τιμές από τις τρεις μονάδες \en{PZEM}, δημιουργεί ένα συνολικό πακέτο δεδομένων (τρεις φάσεις) 
και στέλνει ασύρματα μέσω \en{LoRaWAN} στο \en{gateway}. Κατά την ένταξη, η συσκευή πραγματοποιεί \en{join procedure} με τον 
\en{Join Server} του \en{TTS} και λαμβάνει τα κλειδιά συνεδρίας για ασφαλή διαδρομή δεδομένων. Οι μετρήσεις αποστέλλονται 
περιοδικά με προκαθορισμένο διάστημα.

Η εικόνα παρακάτω δείχνει τη συνδεσμολογία μεταξύ της πλακέτας \en{The Things Uno} και των τριών \en{PZEM modules}:

\begin{Illustration}[ht]
\centering
\includegraphics[width=1\textwidth]{figures/1st_implementation_el_meter.png}
\caption{Υλοποίηση τριφασικού μετρητή - σύνδεση \en{The Things Uno} με 3 \en{PZEM-004T V3.0}}
\label{figure4.3}
\end{Illustration}

