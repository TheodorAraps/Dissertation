\chapter{Εισαγωγή}
\InitialCharacter{Σ}ε μια εποχή σημαδεμένη από διαρκείς τεχνολικές εξελίξεις, 
η ταχύτατη ανάπτυξη του Διαδικτύου των Πραγμάτων \en{Internet of Things - IoT} 
ήταν αναμενόμενη, γεγονός που έχει φέρει στο προσκήνιο πλήθος από νέες τεχνολογίες 
επικοινωνίας, ικανές να συνδέσουν μία τεράστια ποικιλία απομακρυσμένων αισθητήρων και 
συσκευών, με μεγάλο εύρος λειτουργίας και χαμηλό ενεργειακό κόστος \cite{Ubisense2023}. 
Μάλιστα, εκτιμάται ότι μέχρι το 2030 θα υπάρχουν περισσότερες από 30 δισεκατομμύρια 
συσκευές συνδεδεμένες στο διαδίκτυο παγκοσμίως \cite{DemandSage2025}. Η έννοια του 
Διαδικτύου των Πραγμάτων αναφέρεται σε ένα δίκτυο επικοινωνίας, το οποίο συγκροτείται 
από πληθώρα συσκευών εξοπλισμένων με αισθητήρες. Οι συσκευές αυτές συλλέγουν δεδομένα 
από το φυσικό περιβάλλον, τα μεταδίδουν, τα διαμοιράζονται και τα αξιοποιούν, με σκοπό 
την παροχή ποικίλων υπηρεσιών.

Μία από τις πλέον πολλά υποσχόμενες τεχνολογίες στον χώρο των ασύρματων δικτύων ευρείας περιοχής χαμηλής ισχύος (\en{Low Power Wide Area Networks - LPWANs}) 
είναι το \en{LoRaWAN}. Η τεχνολογία αυτή επιτρέπει τη δημιουργία ανθεκτικών και κλιμακούμενων υποδομών επικοινωνίας με ελάχιστες ενεργειακές απαιτήσεις, 
καθιστώντας την ιδανική για εφαρμογές όπου η συνδεσιμότητα και η αυτονομία είναι υψίστης σημασίας.

Στον τομέα της ενέργειας και ειδικότερα στην παρακολούθηση και τον έλεγχο υ\-πο\-στα\-θμών μέσης και χαμηλής τάσης, η ανάγκη για απομακρυσμένη συλλογή μετρήσεων και δεδομένων, καθώς 
και του εξ αποστάσεως ελέγχου, είναι πιο επίκαιρη από ποτέ. Οι «έξυπνοι» μετρητές και τα συστήματα τηλεμετρίας επιτρέπουν την πρόβλεψη, την αποδοτικότερη κατανομή και απαραίτητη εποπτεία 
της κατανάλωσης, τη βελτίωση της ποιότητας της ενέργειας και την έγκαιρη ανίχνευση σφαλμάτων. Συνεπώς, η ενσωμάτωση αυτών των δυνατοτήτων με τεχνολογίες όπως το \en{LoRaWAN} 
αποτελεί ένα σημαντικό βήμα προς την υλοποίηση και εφαρμογή των «έξυπνων» δικτύων ενέργειας (\en{smart grids}).

Η παρούσα διπλωματική εργασία αποσκοπεί στη μελέτη, σχεδίαση και υλοποίηση ενός ολοκληρωμένου συστήματος παρακολούθησης και ελέγχου ηλεκτρικού υποσταθμού, 
με τη χρήση του \en{LoRaWAN}. Το σύστημα που αναπτύχθηκε περιλαμβάνει:
\begin{itemize}
  \item δύο τριφασικούς μετρητές, βασισμένους στην πλακέτα ανάπτυξης \en{The Things Uno} της \en{The Things Industries} και σε τρεις (ανά μετρητή) μονοφασικούς αισθητήρες \en{PZEM-004T} για τη μέτρηση ηλεκτρικών μεγεθών,
  \item έναν \en{LoRaWAN gateway,} κατασκευασμένο με ένα \en{Raspberry Pi 4B} και μία μονάδα συγκέντρωσης \en{iC880A-SPI}, στην οποία συνδέεται μία κεραία χαμηλού σχετικά κέρδους των 2\en{dBi},
  \item ανάπτυξη δυναμικής ιστοσελίδας με χρήση \en{Java Spring Boot} και \en{React (TypeScript)}, με σκοπό την παρουσίαση των μετρήσεων που λαμβάνονται από τους μετρητές.
\end{itemize}

Η υλοποίηση αυτή δεν περιορίζεται μόνο στη θεωρητική διερεύνηση του προαναφερόμενου συστήματος, αλλά αποσκοπεί, επιπλέον, στην πρακτική αξιολόγηση της τεχνολογίας \en{LoRaWAN} στο πεδίο ενός IoT συστήματος παρακολούθησης υποσταθμού, επιδιώκοντας 
να αναδείξει τις δυνατότητες και τους περιορισμούς της, όταν εφαρμόζεται σε ένα πραγματικό περιβάλλον ενεργειακής υποδομής.

\section{Αντικείμενο της διπλωματικής εργασίας}

Αντικείμενο της παρούσας διπλωματικής εργασίας είναι η μελέτη, σχεδίαση και υ\-λο\-ποί\-η\-ση ενός ολοκληρωμένου, 
χαμηλού κόστους και ενεργειακά αποδοτικού συστήματος απομακρυσμένης παρακολούθησης και ελέγχου ενός ηλεκτρικού 
υποσταθμού, αξιοποιώντας τις δυνατότητες του πρωτοκόλλου \en{LoRaWAN}. Το σύστημα αυτό εντάσσεται στο πλαίσιο 
των έξυπνων ενεργειακών υποδομών και της τεχνολογίας \en{Internet of Things (IoT)} και έχει στόχο τη δημιουργία 
μιας επεκτάσιμης και παράλληλα αξιόπιστης αρχιτεκτονικής, η οποία θα μπορεί να υιοθετείται και σε άλλα βιομηχανικά 
ή ενεργειακά περιβάλλοντα.

Ο στόχος της εργασίας είναι διττός:
\begin{enumerate}
  \item \textbf{Πρακτική αξιοποίηση του \en{LoRaWAN}} ως βασικού μέσου επικοινωνίας για την αποστολή δεδομένων 
  από απομακρυσμένες συσκευές μέτρησης προς ένα κεντρικό σύστημα διαχείρισης. Η τεχνολογία αυτή προσφέρει 
  σημαντικά πλεονεκτήματα όπως μεγάλη εμβέλεια, χαμηλή κατανάλωση ενέργειας και υποστήριξη για μεγάλο αριθμό 
  συσκευών, καθιστώντας την, επομένως, ιδανική για απαιτητικά περιβάλλοντα όπως οι ηλεκτρικοί υποσταθμοί.
  
  \item \textbf{Ανάπτυξη ολοκληρωμένης πλατφόρμας εποπτείας και διαχείρισης ενεργειακών μετρήσεων}, η οποία 
  επιτρέπει την οπτικοποίηση και αποθήκευση κρίσιμων ηλεκτρικών παραμέτρων σε πραγματικό χρόνο, όπως η τάση, 
  το ρεύμα, η ισχύς και η κατανάλωση ενέργειας, ανά φάση. Η πλατφόρμα προσφέρει δυνατότητα επεκτασιμότητας και 
  προσαρμοστικότητας, προκειμένου να μπορεί να υποστηρίξει πολλαπλά σημεία μέτρησης, καθώς και μελλοντική ενσωμάτωση 
  επιπρόσθετων λειτουργιών, όπως ειδοποιήσεις ή αυτοματισμούς.
\end{enumerate}

Η υλοποίηση βασίστηκε σε πλήρως παραμετροποιήσιμα εξαρτήματα ανοιχτού κώδικα και λογισμικού, όπως οι συσκευές 
\en{The Things Uno}, οι αισθητήρες \en{PZEM-004T}, ο συγκεντρωτής σήματος \en{iC880A-SPI}, το \en{Raspberry Pi 4B},
 και η πλατφόρμα \en{The Things Stack}. Τα δεδομένα διαχειρίζονται και παρουσιάζονται μέσω διαδικτυακής 
 εφαρμογής υλοποιημένης σε εργαλέια ανοιχτού κώδικα, όπως το \en{JAVA Spring Boot framework} και η \en{front-end} 
 βιβλιοθήκη \en{React} της \en{JavaScript}.

Τέλος, η εργασία αυτή συμβάλλει στην αξιολόγηση των τεχνικών δυνατοτήτων του \en{LoRaWAN}, όταν αυτό χρησιμοποιείται 
σε κρίσιμες εφαρμογές παρακολούθησης, εντοπισμού ανωμαλιών και ενεργειακής διαχείρισης, προσφέροντας έτσι ένα αξιόπιστο 
πρότυπο, το οποίο να μπορεί να αναπαραχθεί ή να εξελιχθεί περαιτέρω σε μεγαλύτερα δίκτυα ή άλλους τύπους βιομηχανικών εγκαταστάσεων.


\section{Οργάνωση του τόμου}
Η εργασία είναι οργανωμένη ως εξής:
\begin{itemize}
  \item Στο Κεφάλαιο 2 παρουσιάζεται το θεωρητικό υπόβαθρο των τεχνολογιών \en{LPWAN}, \en{LoRa} και \en{LoRaWAN}, καθώς και πληροφορίες για τα σύγχρονα συστήματα
  παρακολούθησης και ελέγχου υποσταθμών.
  \item Στο Κεφάλαιο 3 αναλύεται η τεχνολογική στοίβα που χρησιμοποιήθηκε, όπως το \en{The Things Stack}, το \en{LoRa Basics Station}, καθώς και η υποστήριξη 
  από λογισμικά όπως το \en{Docker}.
  \item Στο Κεφάλαιο 4 γίνεται αναλυτική παρουσίαση του εξοπλισμού: \en{gateway}, συσκευές μέτρησης και υπολογιστική υποδομή.
  \item Στο Κεφάλαιο 5 καταγράφεται η υλοποίηση του συστήματος, η παραμετροποίηση του εξοπλισμού, καθώς και η ανάπτυξη της ιστοσελίδας.
  \item Στο Κεφάλαιο 6 παρουσιάζονται τα αποτελέσματα από τις δοκιμές και αξιολογείται η απόδοση του συστήματος.
  \item Τέλος, στο Κεφάλαιο 7 παρατίθενται τα συμπεράσματα και προτείνονται μελλοντικές κατευθύνσεις βελτίωσης και επέκτασης της εργασίας.
\end{itemize}
