\chapter{Εισαγωγή}
\InitialCharacter{Σ}ε μια εποχή σημαδεμένη από διαρκείς τεχνολικές εξελίξεις, η ταχύτατη ανάπτυξη του Διαδικτύου των Πραγμάτων \en{Internet of Things - IoT} ήταν αναμενόμενη, γεγονός 
που έχει φέρει στο προσκήνιο ένα πλήθος από νέες τεχνολογίες επικοινωνίας, ικανές να συνδέσουν μία τεράστια ποικιλία απομακρυσμένων αισθητήρων και συσκευών, 
με μεγάλο εύρος λειτουργείας και χαμηλό ενεργειακό κόστος. Μάλιστα, εκτιμάται ότι μέχρι το 2030 θα υπάρχουν περισσότερες από 30 δισεκατομμύρια συσκευές συνδεδεμένες 
στο διαδίκτυο παγκοσμίως. Η εννόια Διαδίκτυο των Πραγμάτων περιγράφει ουσιαστικά το δίκτυο επικοινωνίας ενός πλήθους από συσκευές, εξοπλισμένα με αισθητήρες, οι οποίες μεταδίδουν, 
διαμοιράζουν και χρησιμοποιούν δεδομένα που λαμβάνουν από το φυσικό περιβάλλον, με σκόπο την παροχή υπηρεσιών.

Μία από τις πλέον πολλά υποσχόμενες τεχνολογίες στον χώρο των ασύρματων δικτύων ευρείας περιοχής χαμηλής ισχύος (\en{Low Power Wide Area Networks – LPWANs}) 
είναι το \en{LoRaWAN}. Η τεχνολογία αυτή επιτρέπει τη δημιουργία ανθεκτικών και κλιμακούμενων υποδομών επικοινωνίας με ελάχιστες ενεργειακές απαιτήσεις, 
καθιστώντας την ιδανική για εφαρμογές όπου η συνδεσιμότητα και η αυτονομία είναι υψίστης σημασίας.

Στον τομέα της ενέργειας και ειδικότερα στην παρακολούθηση και τον έλεγχο υποσταθμών μέσης και χαμηλής τάσης, η ανάγκη για απομακρυσμένη συλλογή μετρήσεων και δεδομένων, καθώς 
και του εξ αποστάσεως ελέγχου, είναι πιο επίκαιρη από ποτέ. Οι «έξυπνοι» μετρητές και τα συστήματα τηλεμετρίας επιτρέπουν την πρόβλεψη, την αποδοτικότερη κατανομή και απαραίτητη εποπτεία 
της κατανάλωσης, τη βελτίωση της ποιότητας της ενέργειας και την έγκαιρη ανίχνευση σφαλμάτων. Συνεπώς, η ενσωμάτωση αυτών των δυνατοτήτων με τεχνολογίες όπως το \en{LoRaWAN} 
αποτελεί ένα σημαντικό βήμα προς την υλοποίηση και εφαρμογή των «έξυπνων» δικτύων ενέργειας (\en{smart grids}).

Η παρούσα διπλωματική εργασία αποσκοπεί στη μελέτη, σχεδίαση και υλοποίηση ενός ολοκληρωμένου συστήματος παρακολούθησης και ελέγχου ηλεκτρικού υποσταθμού, 
με τη χρήση του \en{LoRaWAN}. Το σύστημα που αναπτύχθηκε περιλαμβάνει:
\begin{itemize}
  \item δύο τριφασικούς μετρητές, βασισμένους στην πλακέτα ανάπτυξης \en{The Things Uno} της \en{The Things Industries} και σε τρεις (ανά μετρητή) μονοφασικούς αισθητήρες \en{PZEM-004T} για την μέτρηση ακολούθηση ηλεκτρικών μεγεθών,
  \item έναν \en{LoRaWAN gateway,} κατασκευασμένο με ένα \en{Raspberry Pi 4B} και μία μονάδα συγκέντρωσης \en{iC880A-SPI}, στην οποία συνδέεται μία κεραία υψηλού κέρδους 7.5dBi,
  \item ανάπτυξη δυναμικής ιστοσελίδας με χρήση \en{Java Spring Boot} και \en{React (TypeScript)}, με σκοπό την παρουσίαση των μετρήσεων που λαμβάνονται από τους μετρητές.
\end{itemize}

Η υλοποίηση αυτή δεν περιορίζεται μόνο στη θεωρητική διερεύνηση του προαναφερόμενου συστήματος, αλλά αποσκοπεί στην πρακτική αξιολόγηση της τεχνολογίας \en{LoRaWAN} στο πεδίο ενός IoT συστήματος παρακολούθησης υποσταθμού, επιδιώκοντας 
να αναδείξει τις δυνατότητες και τους περιορισμούς της όταν εφαρμόζεται σε ένα πραγματικό περιβάλλον ενεργειακής υποδομής.

\section{Αντικείμενο της διπλωματικής}
Αντικείμενο της διπλωματικής είναι η ανάπτυξη ενός ολοκληρωμένου και λειτουργικού συστήματος απομακρυσμένης παρακολούθησης και ελέγχου ενός ηλεκτρικού 
υποσταθμού με χρήση του δικτύου \en{LoRaWAN}. Ο στόχος είναι διττός:
\begin{enumerate}
  \item η πρακτική αξιοποίηση του \en{LoRaWAN} ως βασικό μέσο επικοινωνίας σε ένα IoT οικοσύστημα που αφορά κρίσιμες υποδομές,
  \item και η δημιουργία μιας πλατφόρμας εποπτείας και διαχείρισης ενεργειακών μετρήσεων σε πραγματικό χρόνο, με θεμελεια την αξιοπιστία και την επεκτασιμότητα.
\end{enumerate}
Η εργασία ενσωματώνει και αξιοποιεί τεχνολογίες ανοιχτού κώδικα, προσφέροντας ένα πρότυπο εφαρμογής για παρόμοια έργα στον τομέα των «έξυπνων» ενεργειακών υποδομών.

\section{Οργάνωση του τόμου}
Η εργασία είναι οργανωμένη ως εξής:
\begin{itemize}
  \item Στο Κεφάλαιο 2 παρουσιάζεται το θεωρητικό υπόβαθρο των τεχνολογιών \en{LoRa}, \en{LoRaWAN} και \en{LPWAN}.
  \item Στο Κεφάλαιο 3 περιγράφονται οι αρχές λειτουργίας του \en{LoRaWAN}, η αρχιτεκτονική του πρωτοκόλλου και τα χαρακτηριστικά των συσκευών.
  \item Στο Κεφάλαιο 4 αναλύεται η τεχνολογική στοίβα που χρησιμοποιήθηκε, όπως το \en{The Things Stack}, η \en{LoRa Basics Station}, καθώς και η υποστήριξη 
  από λογισμικά όπως το \en{Docker}.
  \item Στο Κεφάλαιο 5 γίνεται αναλυτική παρουσίαση του εξοπλισμού: gateway, συσκευές μέτρησης και υπολογιστική υποδομή.
  \item Στο Κεφάλαιο 6 καταγράφεται η υλοποίηση του συστήματος, η παραμετροποίηση του εξοπλισμού, καθώς και η ανάπτυξη της ιστοσελίδας.
  \item Στο Κεφάλαιο 7 παρουσιάζονται τα αποτελέσματα από τις δοκιμές και αξιολογείται η απόδοση του συστήματος.
  \item Τέλος, στο Κεφάλαιο 8 παρατίθενται τα συμπεράσματα και προτείνονται μελλοντικές κατευθύνσεις βελτίωσης και επέκτασης της εργασίας.
\end{itemize}
