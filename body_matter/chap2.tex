% ========================================================
% Κεφάλαιο 2: Τεχνολογίες LPWAN και το Πρωτόκολλο LoRaWAN
% ========================================================


\chapter{Τεχνολογίες \en{LPWAN} και το Πρωτόκολλο \en{LoRaWAN}}
\label{chap:lpwan}


% -------------------------------
% Ενότητα 2.1: Εισαγωγή στα LPWAN
% -------------------------------


\section{Εισαγωγή στα \en{LPWAN}}

Η διαρκώς αυξανόμενη ανάγκη για απομακρυσμένη και ταυτόχρονα αποδοτική, ως προς την ενέργεια, επικοινωνία 
μεταξύ έξυπνων συσκευών και αισθητήρων έχει οδηγήσει στην εμφάνιση και εξέλιξη μιας νέας γενιάς 
ασύρματων τεχνολογιών, γνωστών ως \en{Low Power Wide Area Networks (LPWAN)}. Οι 
τεχνολογίες \en{LPWAN} επιτρέπουν την αποστολή μικρών σε ποσότητα δεδομένων σε μεγάλες
 αποστάσεις με εξαιρετικά χαμηλή κατανάλωση ενέργειας, καθιστώντας τις ιδανικές για 
 εφαρμογές \en{Internet of Things (IoT)}, όπου η διάρκεια ζωής της μπαταρίας και η 
 αξιοπιστία είναι κρίσιμοι παράγοντες.

Σε αντίθεση με τις τεχνολογίες \en{Wi-Fi} ή \en{Bluetooth}, οι οποίες είναι σχεδιασμένες 
για υψηλούς ρυθμούς μετάδοσης δεδομένων σε μικρές αποστάσεις, τα \en{LPWAN} είναι 
προσανατολισμένα στην υποστήριξη ενός μεγάλου αριθμού συσκευών, με δυνατότητα μετάδοσης 
δεδομένων σε αποστάσεις που υπερβαίνουν τα 10 χιλιόμετρα, σε ανοικτό πεδίο, και σε 
συχνότητες που βρίσκονται στο μη αδειοδοτημένο φάσμα (\en{unlicensed spectrum}). Το 
σημαντικότερο, μάλιστα, όφελος έναντι άλλων τεχνολογιών μετάδωσης πληροφορίας μεγάλου έυρους
(όπως το φάσμα κινητής τηλεφωνίας \en{3G, 4G ή 5G}), είναι η ελάχιστη ενέργεια που απαιτείται 
για την τροφοδοσία των αντίστοιχων συσκευών \cite{ICTexpress}.

Οι πιο διαδεδομένες τεχνολογίες \en{LPWAN} είναι οι εξής:

\begin{itemize}
    \item \textbf{\en{NB-IoT (Narrowband Internet of Things)}} – αποτελεί τεχνολογία ασύρματης 
    επικοινωνίας και χαμηλής ισχύος, βασισμένη στο \en{LTE (Long-Term Evolution)}, η οποία λειτουργεί 
    στο αδειοδοτημένο φάσμα και προσφέρει αξιόπιστη κάλυψη εντός κτιρίων (\en{deep indoor penetration}). 
    Αναπτύχθηκε από το \en{3rd Generation Partnership Project (3GPP)} και υποστηρίζεται από το 
    πρότυπο \en{3GPP Release 13.} Έχει σχεδιαστεί για εφαρμογές με ανάγκες μαζικής συνδεσιμότητας 
    και μικρού όγκου δεδομένων, όπως μετρητές νερού ή αερίου. Η χαμηλή κατανάλωση ενέργειας που 
    απαιτέιται για την λειτουργία των συσκευών έχει ως αποτέλεσμα η διάρκειά λειτουργίας τους να 
    φτάνει έως και 10 χρόνια, με την χρήση μίας μόνο μπαταρίας. \cite{telit2019nbiot}
    
    \item \textbf{\en{LTE-M (LTE Cat-M1)}} – επίσης βασίζεται στο \en{LTE} και προσφέρει 
    υψηλότερους ρυθμούς μετάδοσης δεδομένων από το \en{NB-IoT} (έως και 1 \en{Mbps}), 
    διατηρώντας ωστόσο εξίσου χαμηλή κατανάλωση \cite{zipit2023ltem}. Είναι κατάλληλο για φορητές εφαρμογές 
    που απαιτούν αμφίδρομη επικοινωνία σε πραγματικό χρόνο, όπως η παρακολούθηση οχημάτων, 
    οι φορητές ιατρικές συσκευές και οι φορητοί αισθητήρες. Ένα από τα κύρια πλεονεκτήματά 
    του είναι η υποστήριξη κινητικότητας, επιτρέποντας την απρόσκοπτη μετάβαση μεταξύ κυψελών, 
    καθώς και η δυνατότητα φωνητικής επικοινωνίας μέσω \en{VoLTE}. \cite{hosangadi2019system}
    
    \item \textbf{\en{LoRa} και \en{LoRaWAN}} – πρόκειται για το πιο διαδεδομένο πρωτόκολλο 
    σε μη-αδειοδοτη-μένο φάσμα (π.χ. 868~\en{MHz} στην Ευρώπη), με κύρια πλεονεκτήματα την 
    ευκολία υλοποίησης, τη μεγάλη αυτονομία (έως και 10 έτη), τη χαμηλή κατανάλωση ισχύος 
    και την υψηλή ευελιξία ανάπτυξης μέσω ιδιωτικών ή δημόσιων δικτύων. Η τεχνολογία \en{LoRa} 
    αναπτύχθηκε αρχικά από τη γαλλική \en{Cycleo} και κατοχυρώθηκε από τη \en{Semtech}, 
    ενώ το \en{LoRaWAN} αναπτύσσεται και προτυποποιείται από τη \en{LoRa Alliance}. \cite{semtech_lora_lorawan}
\end{itemize}

Τα δίκτυα \en{LPWAN} ενσωματώνονται όλο και περισσότερο σε κρίσιμες υποδομές, όπως είναι τα συστήματα 
παρακολούθησης ενέργειας, γεωργίας ακριβείας, έξυπνων πόλεων και βιομηχανικής αυτοματοποίησης, 
προσφέροντας λύσεις υψηλής κάλυψης, ανθεκτικότητας και χαμηλού κόστους εγκατάστασης.


% -------------------------------
% Ενότητα 2.2: Σύγκριση Τεχνολογιών LPWAN
% -------------------------------


\section{Σύγκριση Τεχνολογιών \en{LPWAN}}
Παρακάτω γίνεται η σύγκριση των διαφόρων τεχνολογιών \en{LPWAN}:

\begin{Illustration}[!ht] \centering
	\includegraphics[width=0.95\textwidth]{figures/IoT_techs.png} 
    \caption{Σύγκριση τεχνολογιών ασύρματης επικοινωνίας (\en{LPWAN}) ως προς τον ρυθμό μετάδοσης, 
    την κατανάλωση ενέργειας, την εμβέλεια και το κόστος.}
    \label{figure2.1}
    \cite{saft2023iot}
\end{Illustration} 

Οι τεχνολογίες \en{LPWAN} αποτελούν βασικό πυλώνα για την υλοποίηση ενεργειακά αποδοτικών και 
μεγάλης εμβέλειας εφαρμογών \en{IoT}, με διαφορετικές προσεγγίσεις ως προς το φάσμα λειτουργίας, 
την κατανάλωση ισχύος, την κινητικότητα και τη δυνατότητα υποστήριξης ποικίλων τύπων δεδομένων. Ακολούθως 
παρουσιάζονται τα προαναφερθέντα χαρακτηριστικά για τις τρεις πιο διαδεδομένες τεχνολογίες \en{LPWAN:}

\begin{table}[H]
\centering
\renewcommand{\arraystretch}{1.5}
\begin{tabular}{|p{4cm}|p{3.4cm}|p{3.4cm}|p{3.4cm}|}
\hline
\textbf{\textgreek{Παράμετρος}} & \textbf{\en{NB-IoT}} & \textbf{\en{LTE-M}} & \textbf{\en{LoRa}} \\
\hline
\textbf{\textgreek{Τυποποίηση}} & \en{3GPP} & \en{3GPP} & \en{LoRa Alliance} \\
\hline
\textbf{\textgreek{Διαμόρφωση}} & \en{QPSK, 16QAM} & \en{QPSK, 16QAM} & \en{CSS} (\en{Chirp Spread Spectrum}) \\
\hline
\textbf{\textgreek{Φάσμα Συχνοτήτων}} & \en{Licensed} \en{3GPP} (180 \en{kHz}) & \en{Licensed} \en{3GPP} (1.4 \en{MHz}) & \en{Unlicensed ISM} (\en{EU 868 MHz}) \\
\hline
\textbf{\textgreek{Κάλυψη (\en{Link Budget})}} & 151 \en{dB} & 146 \en{dB} & 154 \en{dB} \\
\hline
\textbf{\textgreek{Μέγιστο Φορτίο}} & 1600 \en{bytes} & 1000 \en{bytes} & 242 \en{bytes} \\
\hline
\textbf{\textgreek{Διάρκεια Ζωής Μπαταρίας}} & έως 10 έτη & έως 2 έτη & έως 10 έτη \\
\hline
\textbf{\textgreek{Ταχύτητα Μετάδοσης}} & 200 \en{kbps} & 1 \en{Mbps} & 50 \en{kbps} \\
\hline
\textbf{\textgreek{Αμφίδρομη Επικοινωνία}} & Ναι & Ναι & Ναι \\
\hline
\textbf{\textgreek{Ασφάλεια}} & \en{3GPP (128-256 bit)} & \en{3GPP (128-256 bit)} & \en{AES (128 bit)} \\
\hline
\textbf{\textgreek{Κινητικότητα}} & \en{<100 km/h} & \en{<300 km/h} & Ναι \\
\hline
\textbf{\en{QoS}} & Ναι & Ναι & Όχι \\
\hline
\end{tabular}
\caption{\textgreek{Συγκριτικός πίνακας τεχνολογιών} \en{NB-IoT}, \en{LTE-M} \textgreek{και} \en{LoRa}}
\label{tab:lpwan-comparison}
\cite{ICTexpress}, \cite{adelantado2017understanding}, \cite{zipit2023ltem}, \cite{semtech_lora_lorawan}
\end{table}

Η ανάλυση των επιμέρους χαρακτηριστικών των τριών τεχνολογιών δείχνει πως κάθε μία εξυπηρετεί 
διαφορετικές ανάγκες, ανάλογα με το σενάριο χρήσης και τις απαιτήσεις της εκάστοτε εφαρμογής.

Ξεκινώντας από την κινητικότητα, το \en{LTE-M} υπερέχει με διαφορά, καθώς υποστηρίζει μετακινήσεις 
με ταχύτητες έως και 300 \en{km/h} και δυνατότητα \en{handover} μεταξύ κυψελών, κάτι που καθιστά 
εφικτή την αξιόπιστη σύνδεση σε περιπτώσεις όπως είναι η παρακολούθηση οχημάτων ή \en{drones} εν κινήσει. 
Από την άλλη μεριά, το \en{NB-IoT} παρέχει περιορισμένη κινητικότητα και είναι περισσότερο κατάλληλο 
για στατικές συσκευές, ενώ το \en{LoRa} μπορεί να χρησιμοποιηθεί για κινητές εφαρμογές μόνο 
αν βρίσκεται εντός εμβέλειας ενός διαθέσιμου \en{gateway}, γεγονός που περιορίζει τη χρήση 
του σε δυναμικά περιβάλλοντα.

Στο πεδίο της μετάδοσης δεδομένων, το \en{LTE-M} προσφέρει τους υψηλότερους ρυθμούς 
(1 \en{Mbps}), καθώς και υποστήριξη φωνητικής επικοινωνίας μέσω \en{VoLTE}, 
χαρακτηριστικά που απουσιάζουν από τις άλλες δύο τεχνολογίες. Αντίθετα, το \en{LoRa} 
περιορίζεται σε πολύ χαμηλούς ρυθμούς (50 \en{kbps}) και είναι σχεδιασμένο 
κυρίως για απλές, σποραδικές μεταδόσεις.

Όσον αφορά το φάσμα λειτουργίας, τόσο το \en{NB-IoT} όσο και το \en{LTE-M} αξιοποιούν το
αδειοδοτημένο φάσμα, γεγονός που προσφέρει πιο σταθερή σύνδεση, μικρότερο λανθάνοντα χρόνο 
και καλύτερη ποιότητα υπηρεσίας (\en{QoS}). Αυτά τα χαρακτηριστικά είναι κρίσιμα για εφαρμογές 
όπως \en{POS terminals}, όπου απαιτείται γρήγορη και αξιόπιστη μετάδοση συναλλαγών. Από την άλλη, 
το \en{LoRa} λειτουργεί σε μη αδειοδοτημένο φάσμα, που αν και μειώνει το κόστος, υπόκειται σε 
περιορισμούς όπως το \en{duty cycle} και το \en{fair access policy}, μειώνοντας, έτσι, την αξιοπιστία 
σε περιβάλλοντα όπου υπάρχει υψηλή κίνηση δεδομένων.

Σε όρους ενεργειακής απόδοσης, το \en{LoRa} και το \en{NB-IoT} είναι εμφανώς πιο αποτελεσματικά, 
υποστηρίζοντας διάρκεια μπαταρίας έως και 10 έτη. Το \en{LTE-M}, λόγω της μεγαλύτερης κατανάλωσης 
ισχύος, τείνει να έχει μικρότερη διάρκεια ζωής, συνήθως μεταξύ 1-2 ετών, κάτι που πρέπει να 
ληφθεί υπόψη σε εφαρμογές όπου η συντήρηση των κόμβων δεν είναι εύκολη.

Ως προς την εμπορική απήχηση, οι τεχνολογίες του \en{3GPP} (\en{NB-IoT} και \en{LTE-M}) 
προωθούνται κυρίως μέσω παρόχων κινητής τηλεφωνίας και ενσωματώνονται σε λύσεις ευρείας 
κλίμακας από τη βιομηχανία \cite{gsma2022mobileiot}. Αντίθετα, το \en{LoRaWAN}, μέσω της \en{LoRa Alliance}, διατίθεται 
ευρύτερα για αποκεντρωμένες και ιδιωτικές αναπτύξεις, γεγονός που το έχει καταστήσει ιδιαίτερα 
δημοφιλές σε αγροτικές εφαρμογές, αισθητήρες έξυπνων κτιρίων και περιβαλλοντική παρακολούθηση \cite{loraalliance2023report}.

Συνοψίζοντας, δεν υπάρχει μία «καλύτερη» τεχνολογία για κάθε περίπτωση. Η επιλογή εξαρτάται 
από το εκάστοτε έργο και τους στόχους του: αν προέχει η κινητικότητα και η χαμηλή καθυστέρηση, 
το \en{LTE-M} είναι πιο κατάλληλο· αν ζητούμενο είναι η μεγάλη διάρκεια ζωής και το χαμηλό κόστος, 
το \en{LoRa} αποτελεί ιδανική επιλογή· ενώ το \en{NB-IoT} είναι ενδιάμεση λύση για στατικές 
εφαρμογές με αξιόπιστο σήμα και μεγάλη πυκνότητα κόμβων. Η τελική απόφαση λαμβάνει υπόψη 
τεχνικούς περιορισμούς, απαιτήσεις απόδοσης και το οικονομικό κόστος υλοποίησης.


% -------------------------------
% Ενότητα 2.3: Τεχνολογία LoRa
% -------------------------------

\vspace{3em}
\section{Τεχνολογία \en{LoRa}}


%%%%   Υποενότητα 2.3.1: Γενική Επισκόπηση της Τεχνολογίας \en{LoRa}   %%%%


\subsection{Γενική Επισκόπηση της Τεχνολογίας \en{LoRa}}

Ξεκινώντας με μία μικρή ιστορική αναδρομή, η τεχνολογία \en{LoRa (Long Range)} αναπτύχθηκε αρχικά το 2009 από 
δύο φίλους, τους \en{Nicolas Sornin} και \en{Olivier Seller}, όπου στην συνέχεια συμμετείχε στην ομάδα 
και ένας τρίτος συνεργάτης, ο \en{François Sforza} και όλοι μαζί δημιούργησαν τη γαλλική εταιρεία \en{Cycleo} 
το 2010. Δύο χρόνια μετά (2012), η \en{Cycleo} εξαγοράστηκε από την αμερικανική εταιρεία \en{Semtech} 
\cite{semtech2020lora}. Η τεχνολογία αυτή λειτουργεί αποκλειστικά στο φυσικό επίπεδο 
(\en{Physical Layer, PHY}) του μοντέλου αναφοράς \en{OSI (Open Systems Interconnection model)},
και βασίζεται στη διαμόρφωση εξάπλωσης φάσματος τύπου \en{Chirp Spread Spectrum (CSS)}, 
που επιτρέπει την αξιόπιστη και χαμηλής κατανάλωσης μετάδοση δεδομένων σε μεγάλες αποστάσεις. 
Χρησιμοποιεί το ελεύθερο φάσμα ραδιοσυχνοτήτων \en{ISM (Industrial, Scientific and Medical)}, 
με κύρια μπάντα συχνοτήτων στην Ευρώπη τα \en{868 MHz} \cite{semtech_lora_lorawan}. 

Η τεχνολογία \en{LoRa} παρέχει σημαντική αντοχή σε παρεμβολές, μιας και χρησιμοποιεί προσαρμοστικό ρυθμό 
μετάδοσης \en{(Adaptive Data Rate - ADR)}, ενώ παράλληλα παρουσιάζει και υψηλή ευαισθησία δεκτών, γεγονότα που 
επιτρέπουν την επικοινωνία ακόμα και σε συνθήκες με μεγάλο θόρυβο από το περιβάλλον. Το εύρος ζώνης που 
χρησιμοποιείται (\en{Bandwidth, BW}) είναι συνήθως \en{125 kHz}, \en{250 kHz} ή \en{500 kHz}, ανάλογα με 
τις ανάγκες της εφαρμογής. Παράλληλα, η διαμόρφωση χρησιμοποιεί διαφορετικούς παράγοντες εξάπλωσης 
(\en{Spreading Factors, SF}) από 7 έως 12, που επηρεάζουν τον ρυθμό μετάδοσης δεδομένων και την εμβέλεια 
του σήματος \cite{ttn_lorawan}.


\begin{Illustration}[!ht] 
  \centering
	\includegraphics[width=0.7\textwidth]{figures/OSI_LoRa.png} 
  \caption{Μοντέλο \en{OSI} σε αντιστοίχιση με τα \en{LoRa} και \en{LoRaWAN} επίπεδα.}
  \label{figure2.2}
  \cite{semtech_lora_lorawan}
\end{Illustration}


%%%%   Υποενότητα 2.3.2: Ραδιοφωνική Διάδοση   %%%%


\subsection{Ραδιοφωνική Διάδοση σε αστικό περιβάλλον}

Στο αστικό περιβάλλον, η αξιοπιστία μιας ασύρματης ζεύξης \en{LoRa} επηρεάζεται καθοριστικά από
φαινόμενα ραδιοδιάδοσης όπως η απώλεια διαδρομής \en{(path loss)}, η ζώνη \en{Fresnel}, η
πολυδιαδρομική διάδοση \en{(multipath propagation)} και το φαινόμενο \en{Doppler}. Παρακάτω
παρουσιάζονται αυτά τα φαινόμενα με έμφαση στη φυσική τους ερμηνεία και τη σημασία τους για το
\en{LoRa}, καθώς και ένα αριθμητικό παράδειγμα υπολογισμού του ισοζυγίου ζεύξης \en{(link budget)} σε
αστικές συνθήκες.

\subsubsection{Απώλεια Διαδρομής \en{(Path Loss)}}

Η απώλεια διαδρομής εκφράζει την εξασθένηση της ισχύος του σήματος καθώς αυτό διαδίδεται μέσω
του χώρου. Σε ελεύθερο χώρο (χωρίς εμπόδια), η απώλεια διαδρομής αυξάνεται με την απόσταση και
την συχνότητα σύμφωνα με το θεμελιώδες μοντέλο \en{Friis}. Η εξίσωση της ελεύθερης διαδρομής σε
μορφή λογαριθμικής \en{(dB)} δίνεται από: 
\begin{equation}
L_{FS}(dB) = 32.45 + 20\log_{10}(d_{km}) + 20\log_{10}(f_{MHz})
\end{equation}
όπου $d_{km}$ η απόσταση σε χιλιόμετρα και $f_{MHz}$ η συχνότητα σε \en{MHz}. Για
παράδειγμα, σε συχνότητα $868~MHz$ και απόσταση $2~km$ (σενάριο ζεύξης \en{LoRa} στην
Ευρώπη), η απώλεια ελεύθερου χώρου είναι: 
$$L_{FS} = 32.45 + 20\log_{10}(2) + 20\log_{10}(868) \approx 97.24~dB.$$ 
Αυτό σημαίνει ότι το σήμα εξασθενεί κατά περίπου 97 \en{dB} λόγω διάδοσης σε ελεύθερο χώρο. Σε
πραγματικές αστικές συνθήκες όμως, η απώλεια διαδρομής είναι σημαντικά μεγαλύτερη από την
ιδανική περίπτωση ελεύθερου χώρου. Κτίρια, τοίχοι, και γενικά τα εμπόδια σκιάζουν και διαθλούν το
σήμα, με αποτέλεσμα πρόσθετες να υπάρχουν απώλειες \en{(shadowing, diffraction losses)} \cite{SemtechModulationBasics}. 

Εμπειρικά μοντέλα διάδοσης για πόλεις (όπως το μοντέλο \en{Okumura-Hata}) τυπικά προβλέπουν εκθέτη 
απωλειών μεγαλύτερο από 2 (συχνά 2.7–4 ανάλογα με την πυκνότητα των κτιρίων), γεγονός που συνεπάγεται δεκάδες 
\en{dB} επιπλέον εξασθένησης συγκριτικά με το ελεύθερο πεδίο. Για παράδειγμα, ένα μοντέλο \en{Hata} σε πυκνή αστική
περιοχή μπορεί να δώσει απώλεια διαδρομής περίπου 130–140 \en{dB} στα 2 \en{km}, τιμή αρκετά υψηλότερη από τα
περίπου 97 \en{dB} του ελεύθερου χώρου. Επομένως, το σφάλμα ζεύξης \en{(link margin)} σε αστικά περιβάλλοντα
μειώνεται δραστικά αν δεν υπάρχει καθαρή οπτική επαφή. Είναι ζωτικής σημασίας να λαμβάνεται
υπόψη αυτή η πρόσθετη εξασθένιση κατά τον σχεδιασμό δικτύων \en{LoRa}, ώστε η διαθέσιμη στάθμη
σήματος να παραμένει πάνω από την ευαισθησία του δέκτη για αξιόπιστη επικοινωνία.

\subsubsection{Ζώνη \en{Fresnel} και Διάθλαση}

Η ζώνη \en{Fresnel} περιγράφει μια ελλειψοειδή περιοχή γύρω από την ευθεία της οπτικής επαφής μεταξύ
πομπού-δέκτη, μέσα στην οποία η διάδοση συμβάλλει εποικοδομητικά στην λήψη. Για την πρώτη
ζώνη \en{Fresnel} ($n=1$), η ακτίνα $F_1$ στο ενδιάμεσο της διαδρομής (εκεί όπου η ζώνη είναι μεγαλύτερη)
δίνεται από: 
\begin{equation}
F_1 = \sqrt{\frac{\lambda\, d_1\, d_2}{d_1 + d_2}},
\end{equation}
όπου $\lambda$ το μήκος κύματος του σήματος, και $d_1$, $d_2$ οι αποστάσεις του σημείου από τον
πομπό και το δέκτη αντίστοιχα. Για τη συχνότητα $868~MHz$ ($\lambda\approx0.345~m$)
και συνολική απόσταση $2~km$, η ακτίνα της 1ης ζώνης \en{Fresnel} στο μέσο της διαδρομής (δηλ.
$d_1=d_2=1~km$) είναι: 
$$F_1 \approx \sqrt{\frac{0.345 \times 1000 \times 1000}{2000}} \approx 13~m.$$ 
Η φυσική σημασία αυτής της ζώνης είναι ότι τουλάχιστον το 60\% της πρώτης ζώνης \en{Fresnel} πρέπει
να είναι ελεύθερο από εμπόδια για να μην προκληθεί σημαντική πρόσθετη εξασθένηση \cite{IJASRE2018}. Αν
αντικείμενα (π.χ. κτίρια) παρεμβάλλονται και εισχωρούν βαθιά στη ζώνη \en{Fresnel}, το σήμα θα υποστεί
διάθλαση \en{(diffraction)} γύρω από τα εμπόδια, επιφέροντας μεγάλες απώλειες πέραν της ελεύθερης
διάδοσης. Σε αστικό περιβάλλον, συνήθως η ζεύξη δεν έχει καθαρή οπτική επαφή – η πρώτη ζώνη
\en{Fresnel} συχνά τέμνεται από κτίρια, δέντρα ή άλλες δομές. Αυτό οδηγεί σε μη γραμμική οπτική ζεύξη
\en{(NLoS)}, όπου η λήψη βασίζεται σε διερχόμενα και περιθλασμένα κύματα. Το αποτέλεσμα είναι
το σήμα να μειωθεί σημαντικά. Για τον σχεδιασμό δικτύου \en{LoRa} στην πόλη, συστήνεται η ανύψωση των
κεραιών (π.χ. εγκατάσταση \en{gateway} σε ψηλά κτίρια) ώστε να μεγιστοποιείται η εκκαθάριση της ζώνης
\en{Fresnel} και να ελαχιστοποιούνται οι απώλειες διάθλασης.

\begin{Illustration}[!ht] 
  \centering
	\includegraphics[width=1\textwidth]{figures/Fresnel_zone.png} 
  \caption{Ζώνη \en{Fresnel}.}
  \label{figure2.3}
  \cite{loradocs} 
\end{Illustration}

\subsubsection{Πολυδιαδρομική Διάδοση \en{(Multipath Propagation)}}

Λόγω των ανακλάσεων σε επιφάνειες όπως κτίρια, το έδαφος και άλλα εμπόδια, ένα ασύρματο σήμα
μπορεί να φτάσει στον δέκτη μέσω πολλαπλών διαδρομών. Αυτή η πολυδιαδρομική διάδοση
προκαλεί διαλείψεις \en{(fading)}, μιας και τα σήματα από διαφορετικές διαδρομές μπορεί να φτάσουν με
διαφορετική καθυστέρηση και φάση. Συνεπώς, εάν οι φάσεις τους είναι αντίθετες, ενδέχεται να επέλθει
καταστροφική συμβολή, μειώνοντας σημαντικά την ισχύ του λαμβανόμενου σήματος \en{(deep fade)}. Σε
ένα δυναμικό περιβάλλον, ακόμη και μικρές μετακινήσεις ή αλλαγές μπορούν να μεταβάλουν το
μοτίβο συμβολής, προκαλώντας ταχεία διακύμανση του σήματος \en{(selective fading)}. 

Για τα δίκτυα \en{LoRa}, η πολυδιαδρομή αποτελεί κρίσιμο φαινόμενο σε αστικές περιοχές, ωστόσο η ίδια
η διαμόρφωση \en{LoRa} παρουσιάζει αξιοσημείωτη αντοχή σε \en{multipath} εξασθένιση. Η διαμόρφωση
\en{Chirp Spread Spectrum (CSS)} που χρησιμοποιεί το \en{LoRa} εκπέμπει το σύμβολο ως ένα ευρύ φάσμα
συχνοτήτων \en{(chirp)}, γεγονός που το καθιστά λιγότερο επιρρεπές σε συχνόλεκτες διαλείψεις: το φάσμα
του σήματος είναι σχετικά ευρύ και λειτουργεί αποτελεσματικά σαν ένα είδος διασποράς στο πεδίο
του χρόνου και της συχνότητας \cite{staniec2018}. Σύμφωνα με τεκμηρίωση της \en{Semtech}, το φαρδύ \en{chirp}
προσδίδει στο \en{LoRa} «ανοσία στην πολυδιαδρομή και στο \en{fading}, καθιστώντας το ιδανικό για
αστικά και προαστιακά περιβάλλοντα όπου αυτά τα φαινόμενα κυριαρχούν» \cite{SemtechModulationBasics}. Πειραματικές
μελέτες επιβεβαιώνουν αυτήν την ανθεκτικότητα: ο \en{Staniec} και ο \en{Kowal} (2018) ανέφεραν ότι το \en{LoRa}
παρουσιάζει αξιοσημείωτη ανοχή σε έντονες συνθήκες \en{multipath} και παρεμβολών, ιδίως στα
χαμηλότερα \en{bit-rate} (μεγαλύτερους \en{spreading factors}). Συγκεκριμένα, οι μετρήσεις τους σε
θάλαμο πολλαπλών ανακλάσεων έδειξαν πως για ένα εύρος ρυθμίσεων \en{LoRa} υφίστανται περιοχές
ευαισθησίας: μια «λευκή» περιοχή όπου το σήμα είναι πρακτικά ανεπηρέαστο από \en{multipath}, μια
«ανοιχτή γκρίζα» όπου το σύστημα εξακολουθεί να παρουσιάζει ανοχή στο \en{multipath}, αλλά αρχίζει να επηρεάζεται
από ισχυρές παρεμβολές, και μια «σκούρα γκρίζα» περιοχή όπου υπό ακραίες συνθήκες το \en{LoRa}
γίνεται ευάλωτο και στα δύο φαινόμενα \cite{staniec2018}. 

Παρότι το \en{LoRa} είναι εγγενώς ανθεκτικό, η πολυδιαδρομική διάδοση σε αστικά περιβάλλοντα μπορεί
ακόμη να δημιουργήσει προκλήσεις. Αν οι χρονικές καθυστερήσεις ορισμένων διαδρομών
πλησιάσουν τη διάρκεια συμβόλου \en{LoRa}, μπορεί να προκληθεί παρεμβολή συμβόλων (\en{inter-symbol
interference}). Ωστόσο, δεδομένου ότι οι ρυθμοί μετάδοσης \en{LoRa} είναι χαμηλοί (μεγάλες διάρκειες
συμβόλων ειδικά σε υψηλό \en{SF}), οι περισσότερες ανακλώμενες συνιστώσες καταφθάνουν εντός του
παραθύρου ενός συμβόλου και συγχωνεύονται χωρίς να καταστρέφουν την πληροφορία. Έτσι, στην
πράξη το \en{LoRa} σπανίως υφίσταται ολική απώλεια λόγω \en{multipath}, σε αντίθεση με τεχνολογίες
υψηλότερου ρυθμού όπου το \en{multipath} οδηγεί σε έντονο επιλεκτικό \en{fading}. Παρ’ όλα αυτά, για μέγιστη
αξιοπιστία σε πόλεις, συνιστάται η επιλογή παραμέτρων που μεγιστοποιούν το \en{link margin} (π.χ.
υψηλός \en{SF} που προσφέρει μεγαλύτερη ευαισθησία δέκτη) ώστε ακόμη και τυχόν βαθιές διαλείψεις να
μην ρίχνουν το σήμα κάτω από το όριο λήψης.

\subsubsection{Ισοζύγιο Ζεύξης \en{(Link Budget)}}

Το ισοζύγιο ζεύξης (\en{link budget}) αποτελεί έναν από τους πιο κρίσιμους παράγοντες στον 
σχεδιασμό συστημάτων ασύρματης επικοινωνίας, καθώς εκφράζει το συνολικό εύρος απωλειών που 
μπορεί να αντέξει το σήμα κατά τη μετάδοση, χωρίς να χάσει τη δυνατότητα αξιόπιστης λήψης. 
Πρακτικά, το ισοζύγιο ζεύξης υπολογίζεται ως η διαφορά μεταξύ της ισχύος εκπομπής και της 
ελάχιστης ισχύος που απαιτεί ο δέκτης για να λειτουργήσει αποτελεσματικά:
\begin{equation}
Link\ Budget\ (dB) = P_{TX}(dBm) + G_{TX}(dBi) + G_{RX}(dBi) - Sensitivity_{RX}(dBm) - Losses_{misc}(dB)
\end{equation}

όπου:
\begin{itemize}
  \item $P_{TX}$ είναι η ισχύς εξόδου του πομπού (σε $dBm$),
  \item $G_{TX}, G_{RX}$ είναι τα κέρδη των κεραιών εκπομπού και δέκτη αντίστοιχα (σε $dBi$),
  \item $Sensitivity_{RX}$ είναι η ευαισθησία του δέκτη (σε $dBm$),
  \item $Losses_{misc}$ είναι διάφορες επιπλέον απώλειες (π.χ. λόγω καλωδίων, συνδέσεων ή περιβαλλοντικών συνθηκών).
\end{itemize}

Η υψηλή τιμή του ισοζυγίου ζεύξης υποδηλώνει ότι το σύστημα μπορεί να λειτουργήσει αξιόπιστα 
ακόμα και με πολύ ασθενή σήματα ή σε δύσκολες συνθήκες μετάδοσης (π.χ. απομακρυσμένες συσκευές, 
αστικά περιβάλλοντα με πολλά εμπόδια). Η τεχνολογία \en{LoRa} είναι γνωστή για το ιδιαίτερα υψηλό 
ισοζύγιο ζεύξης της, που τυπικά μπορεί να φτάσει έως και $154 dB$, ανάλογα με τις παραμέτρους της 
διαμόρφωσης (κυρίως τον παράγοντα εξάπλωσης \en{SF} και το εύρος ζώνης \en{BW}).

Η επίτευξη υψηλού ισοζυγίου ζεύξης στο \en{LoRa} προέρχεται από δύο βασικά χαρακτηριστικά:
\begin{itemize}
\item \textbf{Υψηλή Ευαισθησία Δεκτών}: Οι δέκτες \en{LoRa} είναι σχεδιασμένοι ώστε να μπορούν 
να αποκωδικοποιούν σήματα πολύ χαμηλής έντασης, ακόμη και κάτω από το επίπεδο του θορύβου. 
Χαρακτηριστικά, η ευαισθησία των δεκτών \en{LoRa} κυμαίνεται από $-125 dBm$ (για $SF7, BW=125 kHz$) 
έως $-137 dBm$ (για $SF12, BW=125 kHz$), κάτι που επιτρέπει την λήψη εξαιρετικά ασθενών σημάτων 
από πολύ μεγάλες αποστάσεις \cite{SemtechModulationBasics}.
\item \textbf{Χαμηλός Ρυθμός Μετάδοσης και Διαμόρφωση Ευρέως Φάσματος \en{(Spread Spectrum)}}: 
Η διαμόρφωση \en{LoRa} (\en{Chirp Spread Spectrum, CSS}) διαχέει το σήμα σε μεγάλο χρονικό 
διάστημα και εύρος ζώνης, αυξάνοντας την πιθανότητα αξιόπιστης λήψης του σήματος. Με τον 
τρόπο αυτό επιτυγχάνεται ένα κέρδος επεξεργασίας (\en{processing gain}) που δίνεται περίπου 
από τη σχέση:
\begin{equation}
G_{processing} = 10 \cdot \log_{10}\left(\frac{BW}{R_{data}}\right)
\end{equation}

όπου \(BW\) είναι το εύρος ζώνης μετάδοσης και \(R_{data}\) ο ρυθμός δεδομένων. Αυτό το κέρδος 
επεξεργασίας ενισχύει το σήμα σε σχέση με τον θόρυβο, επιτρέποντας την αποκωδικοποίηση ακόμη 
και υπό πολύ χαμηλές τιμές λόγου σήματος προς θόρυβο (\en{SNR}) \cite{SemtechModulationBasics}.
\end{itemize}

Η τεχνολογία \en{LoRa} συγκρινόμενη με την παραδοσιακή διαμόρφωση συχνότητας (\en{Frequency 
Shift Keying - FSK}), η οποία χρησιμοποιείται ευρέως σε άλλες εφαρμογές ασύρματης επικοινωνίας, 
παρουσιάζει σημαντικά υψηλότερη ευαισθησία. Αυτό οφείλεται στο ότι η \en{FSK} απαιτεί ένα 
ελάχιστο θετικό \en{SNR} για αξιόπιστη αποκωδικοποίηση, ενώ η \en{LoRa} μπορεί να λειτουργήσει 
ακόμα και με αρνητικό \en{SNR}, με το σήμα κυριολεκτικά «θαμμένο» μέσα στο θόρυβο, όπως φαίνεται 
και στο Σχήμα~\ref{figure2.3}.

\begin{Illustration}[!ht]
\centering
\includegraphics[width=1\textwidth]{figures/LoRa_vs_FSK.png}
\caption{Σύγκριση ευαισθησίας \en{LoRa} και \en{FSK}, υπογραμμίζοντας την υπεροχή της τεχνολογίας 
\en{LoRa} σε συνθήκες χαμηλού λόγου σήματος προς θόρυβο (\en{SNR}).}
\label{figure2.4}
\cite{SemtechModulationBasics}
\end{Illustration}

Συνοψίζοντας, το υψηλό ισοζύγιο ζεύξης είναι ο κύριος λόγος που η τεχνολογία \en{LoRa} μπορεί 
να επιτύχει επικοινωνία σε πολύ μεγάλες αποστάσεις (έως και δεκάδες χιλιόμετρα σε ανοιχτό πεδίο), 
με πολύ χαμηλή ισχύ εκπομπής, καθιστώντας την ιδανική για εφαρμογές χαμηλής κατανάλωσης σε 
περιβάλλοντα όπου η συντήρηση και η αντικατάσταση μπαταριών είναι δύσκολη ή και αδύνατη.

\subsubsection{Φαινόμενο \en{Doppler}}

Το φαινόμενο \en{Doppler} είναι η μεταβολή της συχνότητας ενός κύματος λόγω σχετικής κίνησης πομπού
ή δέκτη. Στα ασύρματα δίκτυα, εάν ένας κόμβος \en{LoRa} κινείται (π.χ. αισθητήρας σε όχημα) ή αν το
περιβάλλον μεταβάλει αποτελεσματικά τη συχνότητα (π.χ. ανακλώμενη διάδοση από κινούμενα
αντικείμενα), η φέρουσα συχνότητα του σήματος όπως την αντιλαμβάνεται ο δέκτης μετατοπίζεται. Η
μετατόπιση \en{Doppler} $\Delta f$ προσεγγίζεται από τη σχέση: 
\begin{equation}
\Delta f \approx \frac{v}{c} f_c,
\end{equation}
όπου $v$ η σχετική ταχύτητα πομπού-δέκτη, $c$ η ταχύτητα του φωτός και $f_c$ η φερουσα
συχνότητα. Για παράδειγμα, στα $868~MHz$, αν ένας αισθητήρας κινείται με $v=100~km/
h$ ( $\thickapprox 27.8 m/s$), τότε η παρατηρούμενη συχνότητα μετατοπίζεται κατά: 
$$\Delta f \approx \frac{27.8}{3\times10^8} \times 868\times10^6 \approx 80~Hz.$$ 
Μια μετατόπιση περίπου 80 $Hz$ είναι αμελητέα συγκριτικά με το εύρος ζώνης του \en{LoRa} (τυπικά 125 $kHz$), και
συνεπώς δεν υποβαθμίζει την αξιοπιστία του σήματος. Γενικά, το \en{LoRa} είναι εξαιρετικά ανεκτικό στο φαινόμενο 
\en{Doppler} για τις συνήθεις ταχύτητες που συναντώνται σε αστικά σενάρια (π.χ. οχήματα). Η ίδια η
διαμόρφωση με \en{chirp} καθιστά το σήμα ανθεκτικό σε μικρές συχνές αποκλίσεις: μια μόνιμη
μετατόπιση συχνότητας λόγω \en{Doppler} απλώς μεταφράζεται σε μια μικρή χρονική ολίσθηση του
τοπικού χρονοδιαγράμματος αποδιαμόρφωσης, κάτι που ο δέκτης \en{LoRa} μπορεί να αντιμετωπίσει
χωρίς σημαντικές απώλειες απόδοσης. Στην πράξη, αυτό σημαίνει ότι δεν απαιτούνται
κρύσταλλοι υψηλής ακρίβειας για τους ταλαντωτές και ότι το \en{LoRa} λειτουργεί αξιόπιστα ακόμη και
σε κινητικές εφαρμογές, όπως σε αισθητήρες πίεσης ελαστικών, συστήματα διοδίων ή συσκευές σε
μέσα μεταφοράς \cite{SemtechModulationBasics}. 

Φυσικά, σε ακραίες περιπτώσεις πολύ υψηλών ταχυτήτων (πέρα από τα αστικά δεδομένα, π.χ. σε
δορυφορικές ζεύξεις \en{LoRa}) το φαινόμενο \en{Doppler} μπορεί να γίνει υπολογίσιμο, απαιτώντας τεχνικές
διόρθωσης \en{(frequency offset compensation)}. Όμως για επίγειες αστικές επικοινωνίες \en{LoRa}, ακόμη και
ταχύτητες της τάξης των 100–200 $km/h$ μπορούν να εξυπηρετηθούν χωρίς αξιόλογη υποβάθμιση της
ευαισθησίας \cite{SemtechModulationBasics}. Συνοψίζοντας, το φαινόμενο \en{Doppler} δεν αποτελεί κυρίαρχο περιορισμό στην
αξιοπιστία ενός στατικού δικτύου \en{LoRa} ή με αργά κινούμενους κόμβους στην πόλη.

\subsubsection{Παράδειγμα Υπολογισμού Απώλειας Διαδρομής και \en{Link Budget}}

Στην ενότητα αυτή παρουσιάζεται ένα αριθμητικό παράδειγμα που συνδυάζει τον υπολογισμό της
απώλειας διαδρομής και του ισοζυγίου ζεύξης \en{(link budget)} για μια χαρακτηριστική σύνδεση \en{LoRa}
σε αστικό περιβάλλον. Ας θεωρήσουμε ένα σενάριο όπου: 
\begin{itemize}
  \item Συχνότητα λειτουργίας: $f = 868~MHz$ (ζώνη \en{EU863-870}). 
  \item Απόσταση πομπού-δέκτη: $d = 2~km$ (αστικό περιβάλλον, πιθανώς χωρίς καθαρή οπτική επαφή). 
  \item Ισχύς εκπομπής πομπού: $P_{TX} = 14~dBm$ (τυπικό μέγιστο \en{LoRa} επιτρεπόμενο στην ΕΕ). 
  \item Κέρδος κεραίας πομπού/δέκτη: $G_{TX} = 0~dBi$ (μικρή μονοπολική στοναισθητήρα), 
  $G_{RX} = 2~dBi$ (κεραία \en{gateway}). 
  \item Απώλειες καλωδίων/συνδέσεων: $L_{misc} = 2~dB$ (π.χ. απώλεια ομοαξονικού στον σταθμό βάσης). 
  \item Ευαισθησία δέκτη: $S_{\min} \approx -137~dBm$ (υψηλή ευαισθησία, π.χ. \en{LoRa} δέκτης
  σε $SF=12, BW=125 kHz $) \cite{LansitecLoRaRange2024}. 
\end{itemize}

1. \textbf{Υπολογισμός απώλειας διαδρομής}: Χρησιμοποιούμε πρώτα την εξίσωση ελεύθερου χώρου. Για
$d=2 km$, $f=868 MHz$, όπως υπολογίστηκε προηγουμένως, $L_{FS} \approx 97.4 dB$. Σε αστικό
περιβάλλον χωρίς οπτική επαφή, θα προσθέσουμε μια επιπλέον απώλεια λόγω σκίασης/διάθλασης. Ας
υποθέσουμε μια συντηρητική επιπλέον εξασθένηση $L_{urban} = 20~dB$ (λόγω κτιρίων
που μερικώς φράσσουν τη ζώνη \en{Fresnel} και προκαλούν διάθλαση). Έτσι, η συνολική εκτιμώμενη
απώλεια διαδρομής γίνεται: 
$$L_{path} \;=\; L_{FS} + L_{urban} \;\approx\; 97.4 + 20 \;=\; 117.4~dB.$$

2. \textbf{Λήψη και Ισοζύγιο Ζεύξης}: Το ισοζύγιο ζεύξης λαμβάνει υπόψη όλα τα κέρδη και τις απώλειες από
τον πομπό έως τον δέκτη. Η ισχύς που φτάνει στον δέκτη ($P_{RX}$ σε $dBm$) δίνεται από: 
$$P_{RX} = P_{TX} + G_{TX} + G_{RX} - L_{path} - L_{misc}.$$ 
Αντικαθιστώντας τις αριθμητικές τιμές: 
$$P_{RX} = 14~dBm + 0~dB + 2~dB - 117.4~dB - 2~dB.$$
$$P_{RX} \approx -103.4~dBm.$$
Η λαμβανόμενη ισχύς εκτιμάται περίπου $-103.4 dBm$.

3. \textbf{Σύγκριση με ευαισθησία δέκτη}: Δεδομένου ότι ο δέκτης \en{LoRa} (με $SF=12$) μπορεί να ανιχνεύσει
σήματα έως και $S_{\min}\approx -137 dBm$, το συγκεκριμένο σενάριο παρουσιάζει ένα περιθώριο
ζεύξης γύρω στα $33.6 dB$ (διαφορά μεταξύ $-103.4 dBm$ και $-137 dBm$). Αυτό το περιθώριο είναι πολύ
άνετο – υπερκαλύπτει τυχόν επιπλέον απώλειες λόγω πιο έντονου \en{fading} ή παρεμβολών,
εξασφαλίζοντας αξιόπιστη επικοινωνία. Σημειώνεται ότι το υπολογισθέν $L_{path}$ περιείχε
ήδη ένα αποθεματικό 20 $dB$ για αστικό περιβάλλον· στην πράξη, αν η ζεύξη είχε οπτική επαφή \en{(LoS)} το
περιθώριο θα ήταν ακόμη μεγαλύτερο. 

4. \textbf{Επίδραση παραμέτρων \en{LoRa}}: Αξίζει να τονιστεί ότι η ευαισθησία $-137 dBm$ αντιστοιχεί σε
διαμόρφωση \en{LoRa} με τον πιο παρατεταμένο χρόνο συμβόλου ($SF12, BW 125 kHz$) \cite{LansitecLoRaRange2024}. Εάν
χρησιμοποιούνταν μια ταχύτερη διαμόρφωση (π.χ. $SF7$ με ευαισθησία περί τα $-123 dBm$), το
περιθώριο ζεύξης θα μειωνόταν (περίπου $20 dB$ λιγότερο ευαίσθητος δέκτης) αλλά πιθανώς να παρέμενε
επαρκές για $2 km$. Στην περίπτωσή μας, με $SF12$, το πολύ μεγάλο \en{link margin} των $33$+ $dB$ υποδηλώνει
ότι η ζεύξη θα εξακολουθούσε να λειτουργεί ακόμα κι αν η απώλεια διαδρομής ήταν σημαντικά
μεγαλύτερη (π.χ. περίπου μέχρι $150 dB$ συνολικά). Πράγματι, τα σύγχρονα \en{LoRa transceivers} υποστηρίζουν
μέγιστο \en{link budget} της τάξης των $155-170 dB$, ικανό να καλύψει αποστάσεις πολλών δεκάδων
χιλιομέτρων σε ιδανικές συνθήκες. Στο δικό μας σενάριο, μια απώλεια περίπου $117 dB$ είναι αρκετά χαμηλή
συγκριτικά με το διαθέσιμο \en{link budget} (περίπου $154 dB$ με $168 dB$ ανάλογα τον δέκτη), εξηγώντας γιατί οι
ζεύξεις \en{LoRa} μπορούν να επιτύχουν αξιόπιστη επικοινωνία ακόμα και σε αστικό περιβάλλον με
διάφορες προκλήσεις διάδοσης \cite{ttn_lorawan}.

\textbf{Συμπέρασμα του παραδείγματος}: Με τις παραπάνω παραμέτρους, η ζεύξη \en{LoRa} στα $2 km$ όχι μόνο
«κλείνει» (δηλαδή το σήμα υπερβαίνει το κατώφλι ευαισθησίας του δέκτη), αλλά διαθέτει και
σημαντικό περιθώριο αξιοπιστίας. Αυτό το περιθώριο μπορεί να απορροφήσει επιπλέον απώλειες από
φαινόμενα όπως εξασθένιση \en{multipath}, μελλοντική υποβάθμιση σήματος λόγω παρεμβολών, ή μείωση
ισχύος μπαταρίας του πομπού. Δείχνει επίσης τη σπουδαιότητα του \en{link budget}: συνδυάζοντας
υψηλή ευαισθησία δέκτη, επαρκή ισχύ εκπομπής και κεραίες με μικρά έστω κέρδη, το \en{LoRa} πετυχαίνει
μεγάλες εμβέλειες. Σε ακραίες αστικές συνθήκες (π.χ. πολύ πυκνό αστικό τοπίο, εσωτερικό κτιρίων), το
περιθώριο αυτό θα μειωθεί, ωστόσο η εγγενής ανθεκτικότητα του πρωτοκόλλου (λόγω του χαμηλού
ρυθμού μετάδοσης και του \en{spread spectrum}) επιτρέπει στο \en{LoRa} να διατηρεί επικοινωνία ακόμη και
εκεί όπου άλλα συστήματα υψηλότερης συχνότητας ή ταχύτητας αποτυγχάνουν. 


%%%%   Υποενότητα 2.3.3: Chirp Spread Spectrum και η υλοποίησή του στη \en{LoRa}   %%%%


\subsection{\en{Chirp Spread Spectrum} και η υλοποίησή του στη \en{LoRa}}

Η τεχνολογία \en{LoRa} αποτελεί ένα ιδιόκτητο σύστημα διαμόρφωσης ευρέως φάσματος που
βασίζεται στην τεχνική \en{Chirp Spread Spectrum} (\en{CSS}) \cite{SemtechModulationBasics}. Στη
διαμόρφωση \en{CSS}, το αρχικό σήμα διαμορφώνεται μέσω παλμών με γραμμικά μεταβαλλόμενη
συχνότητα – τα λεγόμενα \en{chirps}. Ένα \en{chirp} είναι ουσιαστικά ένα σήμα του οποίου η
συχνότητα μεταβάλλεται προοδευτικά με το χρόνο μέσα σε ένα προκαθορισμένο εύρος ζώνης. Όταν η
συχνότητα αυξάνεται γραμμικά από μια αρχική χαμηλή τιμή σε μια τελική υψηλή, το σήμα ονομάζεται
\en{up-chirp}, ενώ στη περίπτωση φθίνουσας συχνότητας (από υψηλή προς χαμηλή) ονομάζεται
\en{down-chirp}. Η \en{LoRa} αξιοποιεί πλήρως αυτή την τεχνική: κάθε σύμβολο μεταδίδεται ως ένα
\en{chirp} που καλύπτει ολόκληρο το διαθέσιμο εύρος ζώνης (π.χ. $125 kHz$), είτε αυξάνοντας είτε
μειώνοντας τη συχνότητά του \cite{GhoslyaCSS2024}. Η πληροφορία κωδικοποιείται ως μια κυκλική μετατόπιση στη
φάση ή στη συχνότητα εκκίνησης του \en{chirp}. Με άλλα λόγια, η τιμή του κάθε ψηφιακού συμβόλου
καθορίζει το ποιο σημείο του φάσματος θα αντιστοιχεί στην αρχή του \en{chirp} μέσα στο συνολικό
σαρωμένο εύρος, δίνοντας ουσιαστικά διαφορετικό «ολίσθημα» στο μοτίβο συχνότητας.

\begin{Illustration}[!ht]
\centering
\includegraphics[width=1\textwidth]{figures/LoRa_Symbols_chirps.png}
\caption{Φασματογράφημα σήματος \en{LoRa} που παρουσιάζει 8 αρχικά \en{up-chirps} προοιμίου, 2
\en{down-chirps} συγχρονισμού, και ακολουθία 5 \en{chirp} με κωδικοποιημένα δεδομένα (διαφορετική
κυκλική μετατόπιση σε κάθε σύμβολο). Το σήμα σαρώνει πλήρως ένα εύρος ζώνης 125 \en{kHz} με γραμμικά
αυξανόμενη ή μειούμενη συχνότητα σε κάθε \en{chirp}.}
\label{figure2.5}
\cite{GhoslyaCSS2024}
\end{Illustration}

Η μετάδοση ενός πακέτου \en{LoRa} αρχίζει με έναν προκαθορισμένο πρόλογο (\en{preamble}) από
διαδοχικά \en{up-chirps} που επιτρέπουν στον δέκτη να αντιληφθεί την παρουσία σήματος και να
συγχρονίσει τη συχνότητα και το ρολόι του. Τυπικά χρησιμοποιούνται 8 σύμβολα προοιμίου (στην
Ευρώπη), τα οποία ακολουθούνται από 2 ειδικά \en{down-chirps} που σηματοδοτούν το τέλος του
προοιμίου και βοηθούν στον ακριβή συγχρονισμό φάσης του δεκτή . Μετά το προοίμιο,
έπονται τα \en{chirps} που μεταφέρουν τα ωφέλιμα δεδομένα, καθένα εκ των οποίων έχει
τροποποιηθεί κυκλικά ως προς τη φάση ώστε να αντιστοιχεί σε μια συγκεκριμένη ακολουθία \en{bits}. Η
τεχνική \en{CSS} ουσιαστικά εξαπλώνει κάθε σύμβολο σε ένα ευρύ φάσμα συχνοτήτων, παρέχοντας ένα
είδος κέρδους διασποράς (\en{spread gain}) που βελτιώνει την αξιοπιστία: τυχόν παρεμβολές στενού
φάσματος επηρεάζουν μόνο μικρό μέρος του κάθε \en{chirp}, καθιστώντας δυνατή την ανάκτηση του
συμβόλου από τα υπόλοιπα μη διαβρωμένα τμήματα. Επιπλέον, τα σήματα \en{chirp} έχουν σταθερό
πλάτος και λόγω του ότι διατρέχουν όλη τη ζώνη συχνοτήτων, εμφανίζουν ανθεκτικότητα σε
φαινόμενα \en{Doppler}, όπως αναφέρθηκε στην προηγούμενη ενότητα. Η ανοχή σε μετατοπίσεις
συχνότητας επιτρέπει στο \en{LoRa} να χρησιμοποιείται και σε κινητές εφαρμογές (π.χ. αισθητήρες σε
οχήματα) χωρίς σημαντική υποβάθμιση της απόδοσης. Συνολικά, η διαμόρφωση \en{LoRa CSS} 
συνδυάζει μεγάλη εμβέλεια, αντοχή σε θόρυβο/πολυδιαδρομικότητα και χαμηλή κατανάλωση
ισχύος, θυσιάζοντας όμως τον ρυθμό μετάδοσης δεδομένων.




% Edit following for CSS and decide what to keep and what to remove from the above and bellowp

Η τεχνολογία \en{LoRa} χρησιμοποιεί τη διαμόρφωση φάσματος διασποράς με chirp \en{(Chirp 
Spread Spectrum, CSS)}. Σε αυτό το σχήμα διαμόρφωσης, κάθε σύμβολο μεταδίδεται ως ένα 
ημιτονοειδές σήμα whose συχνότητα μεταβάλλεται γραμμικά με τον χρόνο (σήμα chirp). Χάρη 
σε αυτήν την τεχνική διασποράς φάσματος, κάθε σύμβολο καταλαμβάνει ένα πλήρες εύρος 
συχνοτήτων ίσο με το εύρος ζώνης (Bandwidth, \textbf{BW}), γεγονός που προσφέρει αυξημένη 
ανθεκτικότητα σε θόρυβο και παρεμβολές. Επιπλέον, κάθε σύμβολο στο \en{LoRa} αντιστοιχεί 
σε μια διακριτή κυκλική μετατόπιση ενός βασικού chirp, επιτρέποντας την κωδικοποίηση ψηφιακών 
πληροφοριών στην αρχική φάση ή συχνότητα. Ορίζεται ο παράγοντας εξάπλωσης (Spreading Factor, 
\textbf{SF}), ο οποίος καθορίζει τον αριθμό των ``chips'' (μικρών χρονικών βημάτων σήματος) 
που συνθέτουν κάθε σύμβολο. Συγκεκριμένα, ο αριθμός των chips ανά σύμβολο είναι $N = 2^{SF}$. 
Αυτό σημαίνει ότι κάθε σύμβολο αποτελείται από $2^{SF}$ διακριτές μεταβολές συχνότητας 
(chips). Έτσι, ένα σύμβολο μπορεί να αντιστοιχεί σε $2^{SF}$ διαφορετικές τιμές (από 0 έως 
$2^{SF}-1$), κωδικοποιώντας συνολικά $SF$ bits πληροφορίας (αφού $2^{SF}$ διαφορετικοί 
συνδυασμοί αντιστοιχούν σε $SF$ bits). Η διάρκεια κάθε συμβόλου (symbol duration) εξαρτάται 
από τις παραμέτρους $SF$ και $BW$. Δεδομένου ότι το \en{LoRa} μεταδίδει τα chips με ρυθμό 
ίσο με το εύρος ζώνης ($R_c = BW$, σε chips ανά δευτερόλεπτο), ένα σύμβολο που περιέχει 
$2^{SF}$ chips θα έχει διάρκεια: 
\begin{equation}
T_{sym} = \frac{2^{SF}}{BW},,
\end{equation} όπου $T_{sym}$ μετράται σε δευτερόλεπτα. Συνεπώς, η διάρκεια συμβόλου 
αυξάνεται εκθετικά με το $SF$ (αφού κάθε αύξηση του $SF$ κατά 1 διπλασιάζει τον αριθμό 
chips και τον χρόνο μετάδοσης), ενώ μειώνεται αντιστρόφως ανάλογα με το $BW$. Για 
παράδειγμα, με $SF=7$ και $BW=125kHz$ προκύπτει 
$T_{sym} \approx 2^7/125000 = 0.001024s \approx 1.0~\text{ms}$, ενώ με $SF=12$ (ίδιο $BW$) 
προκύπτει $T_{sym} \approx 2^{12}/125000 \approx 0.032768~s \approx 32.8~\text{ms}$, 
δηλαδή πολύ μεγαλύτερη διάρκεια συμβόλου. Το αντίστροφο μέγεθος είναι ο 
\textit{ρυθμός συμβόλων} $R_{sym}$ (σύμβολα ανά δευτερόλεπτο), ο οποίος δίνεται από: 
\begin{equation}
R_{sym} = \frac{1}{T_{sym}} = \frac{BW}{2^{SF}},,
\end{equation} δηλαδή $R_{sym}$ μειώνεται όσο αυξάνεται το $SF$, και αυξάνεται όσο 
αυξάνεται το $BW$ (αντιστρόφως ανάλογη σχέση του $R_{sym}$ με $2^{SF}$ και ανάλογη 
με το $BW$). Επειδή κάθε σύμβολο μπορεί να μεταφέρει $SF$ bits πληροφορίας (στην 
απλούστερη περίπτωση χωρίς διορθωτική κωδικοποίηση), μπορούμε να ορίσουμε τον ωφέλιμο 
\textit{ρυθμό μετάδοσης bit} $R_b$ (bit ανά δευτερόλεπτο) ως το γινόμενο του ρυθμού 
συμβόλων επί τα bits ανά σύμβολο: \begin{equation}
R_b = R_{sym} \cdot SF = SF \cdot \frac{BW}{2^{SF}},,
\end{equation} εκφρασμένο σε bits/second. Όπως αναμένεται, μεγαλύτερο $SF$ (ή μικρότερο $BW$) 
οδηγεί σε χαμηλότερο $R_b$, ενώ μικρότερο $SF$ (ή μεγαλύτερο $BW$) αυξάνει το $R_b$. Στην 
πράξη, η διαμόρφωση \en{LoRa} μπορεί να χρησιμοποιεί πρόσθετη διορθωτική κωδικοποίηση 
(Forward Error Correction, FEC) με λόγο κωδικοποίησης (π.χ. 4/5, 4/6, 4/7, 4/8), γεγονός 
που μειώνει τον καθαρό ρυθμό δεδομένων. Για παράδειγμα, με λόγο 4/5, μόνο το 80\% των 
μεταδιδόμενων bit αντιστοιχεί σε ωφέλιμη πληροφορία, επομένως ο καθαρός ρυθμός bit γίνεται 
$0.8,R_b$. Γενικά, αν ο λόγος κωδικοποίησης εκφράζεται ως $r_c = 4/(4+\delta)$ (όπου 
$\delta=1,2,3,4$ για 4/5 έως 4/8 αντίστοιχα), ο καθαρός ρυθμός δεδομένων δίνεται από 
$R_b^{(\text{net})} = r_c \cdot R_b$ – δηλαδή μεγαλύτερη τιμή του $\delta$ (περισσότερα 
πλεονάζοντα bits FEC) μειώνει ανάλογα το καθαρό bitrate, προσφέροντας όμως αυξημένη προστασία 
σφαλμάτων. Ο ρυθμός chips $R_c$ είναι μια σημαντική έννοια στη διαμόρφωση CSS. Όπως αναφέρθηκε, 
αριθμητικά $R_c = BW$ (chips ανά δευτερόλεπτο), που σημαίνει ότι το σήμα εκπέμπει $R_c$ chips 
κάθε δευτερόλεπτο. Η διάρκεια ενός chip ισούται με $T_{\text{chip}} = 1/R_c = 1/BW$ δευτερόλεπτα. 
Για παράδειγμα, με $BW = 125kHz$, κάθε chip έχει διάρκεια $8\mu s$ περίπου. Κάθε chip αντιστοιχεί 
σε μία μικρή μεταβολή στη συχνότητα του chirp. Μπορούμε να θεωρήσουμε ότι το διαθέσιμο φάσμα 
χωρίζεται σε $2^{SF}$ διακριτά βήματα συχνότητας, καθένα πλάτους $\Delta f = BW/2^{SF}$. Σε 
κάθε διαδοχικό chip, η συχνότητα του σήματος αυξάνεται κατά $\Delta f$, με αποτέλεσμα το σήμα 
να σαρώνει ολόκληρη τη ζώνη συχνοτήτων στη διάρκεια ενός συμβόλου. Η αντιστοιχία $R_c = BW$ 
και ο ορισμός του $SF$ εξασφαλίζουν ότι ένα σύμβολο διάρκειας $T_{sym}$ καλύπτει πράγματι όλο 
το εύρος ζώνης, αποτελούμενο από $2^{SF}$ chips, σύμφωνα με τη σχέση $(BW \cdot T_{sym} = 2^{SF})$ 
που προέκυψε παραπάνω. Ένα βασικό πλεονέκτημα της διαμόρφωσης CSS είναι το \textit{κέρδος λόγω 
διασποράς} (spreading gain), το οποίο βελτιώνει την ευαισθησία και αξιοπιστία της ζεύξης. 
Ουσιαστικά, ο λόγος του ρυθμού chips προς τον ρυθμό συμβόλων ισούται με $R_c / R_{sym} = 2^{SF}$, 
δηλαδή ισούται με τον αριθμό των chips ανά σύμβολο. Αυτός ο λόγος αντιπροσωπεύει τον παράγοντα 
με τον οποίο ενισχύεται το σήμα έναντι του θορύβου μέσω της διασποράς. Σε λογαριθμική κλίμακα, 
το κέρδος διασποράς μπορεί να προσεγγιστεί από τη σχέση: 
\begin{equation}
G_{sp} \approx 10 \log_{10}(2^{SF})  (dB),
\end{equation} που εκφράζεται σε decibel. Για παράδειγμα, αύξηση του $SF$ κατά 1 (διπλασιασμός 
των chips ανά σύμβολο) προσθέτει περίπου $10\log_{10}2 \approx 3$~dB επιπλέον επεξεργαστικού 
κέρδους. Ένα σύστημα με $SF$ = 12 παρουσιάζει θεωρητικά κέρδος διασποράς $10\log_{10}(2^{12}) 
\approx 36$~dB συγκριτικά με ένα σύστημα χωρίς διασπορά. Στην πράξη, αυτό σημαίνει ότι ο δέκτης 
\en{LoRa} μπορεί να ανιχνεύσει και να αποκωδικοποιήσει σήματα ακόμα και όταν η ισχύς τους 
βρίσκεται αρκετά κάτω από το επίπεδο του θερμικού θορύβου (αρνητικό SNR). Φυσικά, το αυξημένο 
αυτό κέρδος συνοδεύεται από το αντίτιμο της μειωμένης ταχύτητας μετάδοσης, όπως φάνηκε παραπάνω, 
αποτελώντας τη γνωστή ανταλλαγή ευαισθησίας-ρυθμού (sensitivity vs data rate trade-off) στη 
διαμόρφωση LoRa.





% -------------------------------------
% Ενότητα 2.4: Το Πρωτόκολλο LoRaWAN
% -------------------------------------


\section{Το Πρωτόκολλο \en{LoRaWAN}}

Το \en{LoRaWAN (Long Range Wide Area Network)}, αποτελεί ένα πρωτόκολλο επικοινωνίας επιπέδου 
\en{MAC (Media Access Control)}, σχεδιασμένο για να επεκτείνει τη φυσική διαστρωμάτωση της 
τεχνολογίας \en{LoRa}, επιτρέποντας την αξιόπιστη και ασφαλή μετάδοση δεδομένων σε δίκτυα 
ευρείας περιοχής με χαμηλή κατανάλωση ενέργειας \cite{semtech_lora_lorawan}. \\


\begin{Illustration}[!ht] 
  \centering
	\includegraphics[width=0.8\textwidth]{figures/LoRa-LoRaWAN_layers.png} 
  \caption{Τεχνολογική στοίβα των \en{LoRa} και \en{LoRaWAN}.}
  \label{figure2.9}
  \cite{semtech_lora_lorawan} 
\end{Illustration} 


Το πρωτόκολλο αναπτύχθηκε και διαχειρίζεται από τον οργανισμό \en{LoRa Alliance}, ο οποίος 
προωθεί τη συμβατότητα και την τυποποίηση μεταξύ κατασκευαστών. Το \en{LoRaWAN} καθορίζει 
τη δικτυακή αρχιτεκτονική, τα επίπεδα ασφαλείας, και τους τρόπους πρόσβασης στο δίκτυο. Η 
αρχιτεκτονική του \en{LoRaWAN} βασίζεται σε μια τοπολογία τύπου αστέρα-από-αστέρες 
\en{(star-of-stars)}, η οποία περιλαμβάνει τα εξής βασικά στοιχεία \cite{ttn_lorawan}:

\begin{itemize}
  \item \textbf{Τερματικές Συσκευές (\en{End Devices})}: Αισθητήρες ή ενεργοποιητές που 
  συλλέγουν δεδομένα και τα αποστέλλουν μέσω του πρωτοκόλλου \en{LoRa}.
  \item \textbf{Πύλες (\en{Gateways})}: Δέχονται τα ασύρματα σήματα από τις τερματικές 
  συσκευές και τα προωθούν στον Διακομιστή Δικτύου \en{(Network Server)} μέσω ενσύρματων 
  ή ασύρματων συνδέσεων, όπως \en{Ethernet}, \en{Wi-Fi} ή κινητά δίκτυα.
  \item \textbf{Διακομιστής Δικτύου (\en{Network Server})}: Διαχειρίζεται τη ροή των δεδομένων, 
  εξαλείφει τα διπλότυπα μηνύματα, εφαρμόζει πολιτικές ασφαλείας και προωθεί τα δεδομένα στους 
  Διακομιστές Εφαρμογών \en{(Application Servers)}.
  \item \textbf{Διακομιστής Εφαρμογών (\en{Application Server})}: Επεξεργάζεται τα δεδομένα 
  σύμφωνα με τις ανάγκες της εκάστοτε εφαρμογής.
\end{itemize}

\begin{Illustration}[!ht] 
  \centering
	\includegraphics[width=1\textwidth]{figures/LoRaWAN_architecture.png} 
  \caption{Τυπική αρχιτεκτονική \en{LoRaWAN} δικτύου.}
  \label{figure2.10}
  \cite{ttn_lorawan}
\end{Illustration} 

\begin{Illustration}[!ht] 
  \centering
	\includegraphics[width=1\textwidth]{figures/LoRaWAN_Network_Server_architecture.png} 
  \caption{Αρχιτεκτονική \en{LoRaWAN Network Server}.}
  \label{figure2.11}
  \cite{tti_homepage}
\end{Illustration} 







\section{Ηλεκτρικοί Υποσταθμοί και Ανάγκες Εποπτείας}
Οι ηλεκτρικοί υποσταθμοί αποτελούν κρίσιμα σημεία του ηλεκτρικού συστήματος μεταφοράς και διανομής ενέργειας. Ο ρόλος τους είναι η μετατροπή της τάσης από υψηλά επίπεδα μεταφοράς σε χαμηλότερα επίπεδα που είναι κατάλληλα για διανομή και τελική κατανάλωση. Οι υποσταθμοί μπορούν να είναι είτε πρωτεύοντες (μεταφοράς), είτε δευτερεύοντες (διανομής).

Η εποπτεία και διαχείριση των υποσταθμών περιλαμβάνει:
\begin{itemize}
  \item παρακολούθηση ηλεκτρικών παραμέτρων όπως ρεύμα, τάση, ισχύς και συχνότητα ανά φάση,
  \item ανίχνευση βλαβών ή ανομαλιών (π.χ. υπερφόρτιση, βυθίσεις τάσης),
  \item έλεγχο λειτουργικών μονάδων όπως διακόπτες ισχύος και προστατευτικά ρελέ,
  \item λήψη αποφάσεων σε πραγματικό χρόνο για την εξασφάλιση της αδιάλειπτης παροχής και της ασφάλειας του εξοπλισμού.
\end{itemize}

Παραδοσιακά, τέτοια εποπτεία γινόταν με ενσύρματες ή \en{SCADA} λύσεις υψηλού κόστους. Η ενσωμάτωση τεχνολογιών όπως το \en{LoRaWAN} επιτρέπει τη δημιουργία αποκεντρωμένων, χαμηλού κόστους και επεκτάσιμων λύσεων, κατάλληλων ακόμη και για μικρούς ή απομακρυσμένους υποσταθμούς.

\section{Συμπεράσματα}
Οι τεχνολογίες \en{LPWAN} και ιδιαίτερα το \en{LoRaWAN} παρέχουν μια αποτελεσματική λύση για τηλεμετρικές εφαρμογές σε περιβάλλοντα όπου απαιτείται χαμηλή κατανάλωση ισχύος και μεγάλη απόσταση μετάδοσης. Το θεωρητικό αυτό υπόβαθρο θεμελιώνει την επιλογή του \en{LoRaWAN} ως βασική τεχνολογία επικοινωνίας στο σύστημα παρακολούθησης και ελέγχου υποσταθμού που αναπτύχθηκε στο πλαίσιο της παρούσας διπλωματικής εργασίας.
