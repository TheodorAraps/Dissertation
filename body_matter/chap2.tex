% ========================================================
% Κεφάλαιο 2: Τεχνολογίες LPWAN και το Πρωτόκολλο LoRaWAN
% ========================================================


\chapter{Τεχνολογίες \en{LPWAN} και το Πρωτόκολλο \en{LoRaWAN}}
\label{chap:lpwan}


% -------------------------------
% Ενότητα 2.1: Εισαγωγή στα LPWAN
% -------------------------------


\section{Εισαγωγή στα \en{LPWAN}}

Η διαρκώς αυξανόμενη ανάγκη για απομακρυσμένη και ταυτόχρονα αποδοτική, ως προς την ενέργεια, επικοινωνία 
μεταξύ έξυπνων συσκευών και αισθητήρων έχει οδηγήσει στην εμφάνιση και εξέλιξη μιας νέας γενιάς 
ασύρματων τεχνολογιών, γνωστών ως \en{Low Power Wide Area Networks (LPWAN)}. Οι 
τεχνολογίες \en{LPWAN} επιτρέπουν την αποστολή μικρών σε ποσότητα δεδομένων σε μεγάλες
 αποστάσεις με εξαιρετικά χαμηλή κατανάλωση ενέργειας, καθιστώντας τις ιδανικές για 
 εφαρμογές \en{Internet of Things (IoT)}, όπου η διάρκεια ζωής της μπαταρίας και η 
 αξιοπιστία είναι κρίσιμοι παράγοντες.

Σε αντίθεση με τις τεχνολογίες \en{Wi-Fi} ή \en{Bluetooth}, οι οποίες είναι σχεδιασμένες 
για υψηλούς ρυθμούς μετάδοσης δεδομένων σε μικρές αποστάσεις, τα \en{LPWAN} είναι 
προσανατολισμένα στην υποστήριξη ενός μεγάλου αριθμού συσκευών, με δυνατότητα μετάδοσης 
δεδομένων σε αποστάσεις που υπερβαίνουν τα 10 χιλιόμετρα, σε ανοικτό πεδίο, και σε 
συχνότητες που βρίσκονται στο μη αδειοδοτημένο φάσμα (\en{unlicensed spectrum}). Το 
σημαντικότερο, μάλιστα, όφελος έναντι άλλων τεχνολογιών μετάδωσης πληροφορίας μεγάλου έυρους
(όπως το φάσμα κινητής τηλεφωνίας \en{3G, 4G ή 5G}), είναι η ελάχιστη ενέργεια που απαιτείται 
για την τροφοδοσία των αντίστοιχων συσκευών \cite{ICTexpress}.

Οι πιο διαδεδομένες τεχνολογίες \en{LPWAN} είναι οι εξής:

\begin{itemize}
    \item \textbf{\en{NB-IoT (Narrowband Internet of Things)}} – αποτελεί τεχνολογία ασύρματης 
    επικοινωνίας και χαμηλής ισχύος, βασισμένη στο \en{LTE (Long-Term Evolution)}, η οποία λειτουργεί 
    στο αδειοδοτημένο φάσμα και προσφέρει αξιόπιστη κάλυψη εντός κτιρίων (\en{deep indoor penetration}). 
    Αναπτύχθηκε από το \en{3rd Generation Partnership Project (3GPP)} και υποστηρίζεται από το 
    πρότυπο \en{3GPP Release 13.} Έχει σχεδιαστεί για εφαρμογές με ανάγκες μαζικής συνδεσιμότητας 
    και μικρού όγκου δεδομένων, όπως μετρητές νερού ή αερίου. Η χαμηλή κατανάλωση ενέργειας που 
    απαιτέιται για την λειτουργία των συσκευών έχει ως αποτέλεσμα η διάρκειά λειτουργίας τους να 
    φτάνει έως και 10 χρόνια, με την χρήση μίας μόνο μπαταρίας. \cite{telit2019nbiot}
    
    \item \textbf{\en{LTE-M (LTE Cat-M1)}} – επίσης βασίζεται στο \en{LTE} και προσφέρει 
    υψηλότερους ρυθμούς μετάδοσης δεδομένων από το \en{NB-IoT} (έως και 1 \en{Mbps}), 
    διατηρώντας ωστόσο εξίσου χαμηλή κατανάλωση \cite{zipit2023ltem}. Είναι κατάλληλο για φορητές εφαρμογές 
    που απαιτούν αμφίδρομη επικοινωνία σε πραγματικό χρόνο, όπως η παρακολούθηση οχημάτων, 
    οι φορητές ιατρικές συσκευές και οι φορητοί αισθητήρες. Ένα από τα κύρια πλεονεκτήματά 
    του είναι η υποστήριξη κινητικότητας, επιτρέποντας την απρόσκοπτη μετάβαση μεταξύ κυψελών, 
    καθώς και η δυνατότητα φωνητικής επικοινωνίας μέσω \en{VoLTE}. \cite{hosangadi2019system}
    
    \item \textbf{\en{LoRa} και \en{LoRaWAN}} – πρόκειται για το πιο διαδεδομένο πρωτόκολλο 
    σε μη-αδειοδοτη-μένο φάσμα (π.χ. 868~\en{MHz} στην Ευρώπη), με κύρια πλεονεκτήματα την 
    ευκολία υλοποίησης, τη μεγάλη αυτονομία (έως και 10 έτη), τη χαμηλή κατανάλωση ισχύος 
    και την υψηλή ευελιξία ανάπτυξης μέσω ιδιωτικών ή δημόσιων δικτύων. Η τεχνολογία \en{LoRa} 
    αναπτύχθηκε αρχικά από τη γαλλική \en{Cycleo} και κατοχυρώθηκε από τη \en{Semtech}, 
    ενώ το \en{LoRaWAN} αναπτύσσεται και προτυποποιείται από τη \en{LoRa Alliance}. \cite{semtech_lora_lorawan}
\end{itemize}

Τα δίκτυα \en{LPWAN} ενσωματώνονται όλο και περισσότερο σε κρίσιμες υποδομές, όπως είναι τα συστήματα 
παρακολούθησης ενέργειας, γεωργίας ακριβείας, έξυπνων πόλεων και βιομηχανικής αυτοματοποίησης, 
προσφέροντας λύσεις υψηλής κάλυψης, ανθεκτικότητας και χαμηλού κόστους εγκατάστασης.


% -------------------------------
% Ενότητα 2.2: Σύγκριση Τεχνολογιών LPWAN
% -------------------------------


\section{Σύγκριση Τεχνολογιών \en{LPWAN}}
Παρακάτω γίνεται η σύγκριση των διαφόρων τεχνολογιών \en{LPWAN}:

\begin{Illustration}[!ht] \centering
	\includegraphics[width=0.95\textwidth]{figures/IoT_techs.png} 
    \caption{Σύγκριση τεχνολογιών ασύρματης επικοινωνίας (\en{LPWAN}) ως προς τον ρυθμό μετάδοσης, 
    την κατανάλωση ενέργειας, την εμβέλεια και το κόστος.}
    \label{figure2.1}
    \cite{saft2023iot}
\end{Illustration} 

Οι τεχνολογίες \en{LPWAN} αποτελούν βασικό πυλώνα για την υλοποίηση ενεργειακά αποδοτικών και 
μεγάλης εμβέλειας εφαρμογών \en{IoT}, με διαφορετικές προσεγγίσεις ως προς το φάσμα λειτουργίας, 
την κατανάλωση ισχύος, την κινητικότητα και τη δυνατότητα υποστήριξης ποικίλων τύπων δεδομένων. Ακολούθως 
παρουσιάζονται τα προαναφερθέντα χαρακτηριστικά για τις τρεις πιο διαδεδομένες τεχνολογίες \en{LPWAN:}

\begin{table}[H]
\centering
\renewcommand{\arraystretch}{1.5}
\begin{tabular}{|p{4cm}|p{3.4cm}|p{3.4cm}|p{3.4cm}|}
\hline
\textbf{\textgreek{Παράμετρος}} & \textbf{\en{NB-IoT}} & \textbf{\en{LTE-M}} & \textbf{\en{LoRa}} \\
\hline
\textbf{\textgreek{Τυποποίηση}} & \en{3GPP} & \en{3GPP} & \en{LoRa Alliance} \\
\hline
\textbf{\textgreek{Διαμόρφωση}} & \en{QPSK, 16QAM} & \en{QPSK, 16QAM} & \en{CSS} (\en{Chirp Spread Spectrum}) \\
\hline
\textbf{\textgreek{Φάσμα Συχνοτήτων}} & \en{Licensed} \en{3GPP} (180 \en{kHz}) & \en{Licensed} \en{3GPP} (1.4 \en{MHz}) & \en{Unlicensed ISM} (\en{EU 868 MHz}) \\
\hline
\textbf{\textgreek{Κάλυψη (\en{Link Budget})}} & 151 \en{dB} & 146 \en{dB} & 154 \en{dB} \\
\hline
\textbf{\textgreek{Μέγιστο Φορτίο}} & 1600 \en{bytes} & 1000 \en{bytes} & 242 \en{bytes} \\
\hline
\textbf{\textgreek{Διάρκεια Ζωής Μπαταρίας}} & έως 10 έτη & έως 2 έτη & έως 10 έτη \\
\hline
\textbf{\textgreek{Ταχύτητα Μετάδοσης}} & 200 \en{kbps} & 1 \en{Mbps} & 50 \en{kbps} \\
\hline
\textbf{\textgreek{Αμφίδρομη Επικοινωνία}} & Ναι & Ναι & Ναι \\
\hline
\textbf{\textgreek{Ασφάλεια}} & \en{3GPP (128-256 bit)} & \en{3GPP (128-256 bit)} & \en{AES (128 bit)} \\
\hline
\textbf{\textgreek{Κινητικότητα}} & \en{<100 km/h} & \en{<300 km/h} & Ναι \\
\hline
\textbf{\en{QoS}} & Ναι & Ναι & Όχι \\
\hline
\end{tabular}
\caption{\textgreek{Συγκριτικός πίνακας τεχνολογιών} \en{NB-IoT}, \en{LTE-M} \textgreek{και} \en{LoRa}}
\label{tab:lpwan-comparison}
\cite{ICTexpress}, \cite{adelantado2017understanding}, \cite{zipit2023ltem}, \cite{semtech_lora_lorawan}
\end{table}

Η ανάλυση των επιμέρους χαρακτηριστικών των τριών τεχνολογιών δείχνει πως κάθε μία εξυπηρετεί 
διαφορετικές ανάγκες, ανάλογα με το σενάριο χρήσης και τις απαιτήσεις της εκάστοτε εφαρμογής.

Ξεκινώντας από την κινητικότητα, το \en{LTE-M} υπερέχει με διαφορά, καθώς υποστηρίζει μετακινήσεις 
με ταχύτητες έως και 300 \en{km/h} και δυνατότητα \en{handover} μεταξύ κυψελών, κάτι που καθιστά 
εφικτή την αξιόπιστη σύνδεση σε περιπτώσεις όπως είναι η παρακολούθηση οχημάτων ή \en{drones} εν κινήσει. 
Από την άλλη μεριά, το \en{NB-IoT} παρέχει περιορισμένη κινητικότητα και είναι περισσότερο κατάλληλο 
για στατικές συσκευές, ενώ το \en{LoRa} μπορεί να χρησιμοποιηθεί για κινητές εφαρμογές μόνο 
αν βρίσκεται εντός εμβέλειας ενός διαθέσιμου \en{gateway}, γεγονός που περιορίζει τη χρήση 
του σε δυναμικά περιβάλλοντα.

Στο πεδίο της μετάδοσης δεδομένων, το \en{LTE-M} προσφέρει τους υψηλότερους ρυθμούς 
(1 \en{Mbps}), καθώς και υποστήριξη φωνητικής επικοινωνίας μέσω \en{VoLTE}, 
χαρακτηριστικά που απουσιάζουν από τις άλλες δύο τεχνολογίες. Αντίθετα, το \en{LoRa} 
περιορίζεται σε πολύ χαμηλούς ρυθμούς (50 \en{kbps}) και είναι σχεδιασμένο 
κυρίως για απλές, σποραδικές μεταδόσεις.

Όσον αφορά το φάσμα λειτουργίας, τόσο το \en{NB-IoT} όσο και το \en{LTE-M} αξιοποιούν το
αδειοδοτημένο φάσμα, γεγονός που προσφέρει πιο σταθερή σύνδεση, μικρότερο λανθάνοντα χρόνο 
και καλύτερη ποιότητα υπηρεσίας (\en{QoS}). Αυτά τα χαρακτηριστικά είναι κρίσιμα για εφαρμογές 
όπως \en{POS terminals}, όπου απαιτείται γρήγορη και αξιόπιστη μετάδοση συναλλαγών. Από την άλλη, 
το \en{LoRa} λειτουργεί σε μη αδειοδοτημένο φάσμα, που αν και μειώνει το κόστος, υπόκειται σε 
περιορισμούς όπως το \en{duty cycle} και το \en{fair access policy}, μειώνοντας, έτσι, την αξιοπιστία 
σε περιβάλλοντα όπου υπάρχει υψηλή κίνηση δεδομένων.

Σε όρους ενεργειακής απόδοσης, το \en{LoRa} και το \en{NB-IoT} είναι εμφανώς πιο αποτελεσματικά, 
υποστηρίζοντας διάρκεια μπαταρίας έως και 10 έτη. Το \en{LTE-M}, λόγω της μεγαλύτερης κατανάλωσης 
ισχύος, τείνει να έχει μικρότερη διάρκεια ζωής, συνήθως μεταξύ 1-2 ετών, κάτι που πρέπει να 
ληφθεί υπόψη σε εφαρμογές όπου η συντήρηση των κόμβων δεν είναι εύκολη.

Ως προς την εμπορική απήχηση, οι τεχνολογίες του \en{3GPP} (\en{NB-IoT} και \en{LTE-M}) 
προωθούνται κυρίως μέσω παρόχων κινητής τηλεφωνίας και ενσωματώνονται σε λύσεις ευρείας 
κλίμακας από τη βιομηχανία \cite{gsma2022mobileiot}. Αντίθετα, το \en{LoRaWAN}, μέσω της \en{LoRa Alliance}, διατίθεται 
ευρύτερα για αποκεντρωμένες και ιδιωτικές αναπτύξεις, γεγονός που το έχει καταστήσει ιδιαίτερα 
δημοφιλές σε αγροτικές εφαρμογές, αισθητήρες έξυπνων κτιρίων και περιβαλλοντική παρακολούθηση \cite{loraalliance2023report}.

Συνοψίζοντας, δεν υπάρχει μία «καλύτερη» τεχνολογία για κάθε περίπτωση. Η επιλογή εξαρτάται 
από το εκάστοτε έργο και τους στόχους του: αν προέχει η κινητικότητα και η χαμηλή καθυστέρηση, 
το \en{LTE-M} είναι πιο κατάλληλο· αν ζητούμενο είναι η μεγάλη διάρκεια ζωής και το χαμηλό κόστος, 
το \en{LoRa} αποτελεί ιδανική επιλογή· ενώ το \en{NB-IoT} είναι ενδιάμεση λύση για στατικές 
εφαρμογές με αξιόπιστο σήμα και μεγάλη πυκνότητα κόμβων. Η τελική απόφαση λαμβάνει υπόψη 
τεχνικούς περιορισμούς, απαιτήσεις απόδοσης και το οικονομικό κόστος υλοποίησης.


% -------------------------------
% Ενότητα 2.3: Τεχνολογία LoRa
% -------------------------------


\section{Τεχνολογία \en{LoRa}}


%%%%   Υποενότητα 2.3.1: Γενική Επισκόπηση της Τεχνολογίας LoRa   %%%%


\subsection{Γενική Επισκόπηση της Τεχνολογίας \en{LoRa}}

Ξεκινώντας με μία μικρή ιστορική αναδρομή, η τεχνολογία \en{LoRa (Long Range)} αναπτύχθηκε αρχικά το 2009 από 
δύο φίλους, τους \en{Nicolas Sornin} και \en{Olivier Seller}, όπου στην συνέχεια συμμετείχε στην ομάδα 
και ένας τρίτος συνεργάτης, ο \en{François Sforza} και όλοι μαζί δημιούργησαν τη γαλλική εταιρεία \en{Cycleo} 
το 2010. Δύο χρόνια μετά (2012), η \en{Cycleo} εξαγοράστηκε από την αμερικανική εταιρεία \en{Semtech} 
\cite{semtech2020lora}. Η τεχνολογία αυτή λειτουργεί αποκλειστικά στο φυσικό επίπεδο 
(\en{Physical Layer, PHY}) του μοντέλου αναφοράς \en{OSI (Open Systems Interconnection model)},
και βασίζεται στη διαμόρφωση εξάπλωσης φάσματος τύπου \en{Chirp Spread Spectrum (CSS)}, 
που επιτρέπει την αξιόπιστη και χαμηλής κατανάλωσης μετάδοση δεδομένων σε μεγάλες αποστάσεις. 
Χρησιμοποιεί το ελεύθερο φάσμα ραδιοσυχνοτήτων \en{ISM (Industrial, Scientific and Medical)}, 
με κύρια μπάντα συχνοτήτων στην Ευρώπη τα \en{868 MHz} \cite{semtech_lora_lorawan}. 

Η τεχνολογία \en{LoRa} παρέχει σημαντική αντοχή σε παρεμβολές, μιας και χρησιμοποιεί προσαρμοστικό ρυθμό 
μετάδοσης \en{(Adaptive Data Rate - ADR)}, ενώ παράλληλα παρουσιάζει και υψηλή ευαισθησία δεκτών, γεγονότα που 
επιτρέπουν την επικοινωνία ακόμα και σε συνθήκες με μεγάλο θόρυβο από το περιβάλλον. Το εύρος ζώνης που 
χρησιμοποιείται (\en{Bandwidth, BW}) είναι συνήθως \en{125 kHz}, \en{250 kHz} ή \en{500 kHz}, ανάλογα με 
τις ανάγκες της εφαρμογής. Παράλληλα, η διαμόρφωση χρησιμοποιεί διαφορετικούς παράγοντες εξάπλωσης 
(\en{Spreading Factors, SF}) από 7 έως 12, που επηρεάζουν τον ρυθμό μετάδοσης δεδομένων και την εμβέλεια 
του σήματος \cite{ttn_lorawan}.


\begin{Illustration}[!ht] 
  \centering
	\includegraphics[width=0.7\textwidth]{figures/OSI_LoRa.png} 
  \caption{Μοντέλο \en{OSI} σε αντιστοίχιση με τα \en{LoRa} και \en{LoRaWAN} επίπεδα.}
  \label{figure2.2}
  \cite{semtech_lora_lorawan}
\end{Illustration}
























% -------------------------------------
% Ενότητα 2.4: Το Πρωτόκολλο LoRaWAN
% -------------------------------------


\section{Το Πρωτόκολλο \en{LoRaWAN}}

Το \en{LoRaWAN (Long Range Wide Area Network)}, αποτελεί ένα πρωτόκολλο επικοινωνίας επιπέδου 
\en{MAC (Media Access Control)}, σχεδιασμένο για να επεκτείνει τη φυσική διαστρωμάτωση της 
τεχνολογίας \en{LoRa}, επιτρέποντας την αξιόπιστη και ασφαλή μετάδοση δεδομένων σε δίκτυα 
ευρείας περιοχής με χαμηλή κατανάλωση ενέργειας \cite{semtech_lora_lorawan}. \\


\begin{Illustration}[!ht] 
  \centering
	\includegraphics[width=0.8\textwidth]{figures/LoRa-LoRaWAN_layers.png} 
  \caption{Τεχνολογική στοίβα των \en{LoRa} και \en{LoRaWAN}.}
  \label{figure2.3}
  \cite{semtech_lora_lorawan} 
\end{Illustration} 


Το πρωτόκολλο αναπτύχθηκε και διαχειρίζεται από τον οργανισμό \en{LoRa Alliance}, ο οποίος 
προωθεί τη συμβατότητα και την τυποποίηση μεταξύ κατασκευαστών. Το \en{LoRaWAN} καθορίζει 
τη δικτυακή αρχιτεκτονική, τα επίπεδα ασφαλείας, και τους τρόπους πρόσβασης στο δίκτυο. Η 
αρχιτεκτονική του \en{LoRaWAN} βασίζεται σε μια τοπολογία τύπου αστέρα-από-αστέρες 
\en{(star-of-stars)}, η οποία περιλαμβάνει τα εξής βασικά στοιχεία \cite{ttn_lorawan}:

\begin{itemize}
  \item \textbf{Τερματικές Συσκευές (\en{End Devices})}: Αισθητήρες ή ενεργοποιητές που 
  συλλέγουν δεδομένα και τα αποστέλλουν μέσω του πρωτοκόλλου \en{LoRa}.
  \item \textbf{Πύλες (\en{Gateways})}: Δέχονται τα ασύρματα σήματα από τις τερματικές 
  συσκευές και τα προωθούν στον Διακομιστή Δικτύου \en{(Network Server)} μέσω ενσύρματων 
  ή ασύρματων συνδέσεων, όπως \en{Ethernet}, \en{Wi-Fi} ή κινητά δίκτυα.
  \item \textbf{Διακομιστής Δικτύου (\en{Network Server})}: Διαχειρίζεται τη ροή των δεδομένων, 
  εξαλείφει τα διπλότυπα μηνύματα, εφαρμόζει πολιτικές ασφαλείας και προωθεί τα δεδομένα στους 
  Διακομιστές Εφαρμογών \en{(Application Servers)}.
  \item \textbf{Διακομιστής Εφαρμογών (\en{Application Server})}: Επεξεργάζεται τα δεδομένα 
  σύμφωνα με τις ανάγκες της εκάστοτε εφαρμογής.
\end{itemize}

\begin{Illustration}[!ht] 
  \centering
	\includegraphics[width=1\textwidth]{figures/LoRaWAN_architecture.png} 
  \caption{Τυπική αρχιτεκτονική \en{LoRaWAN} δικτύου.}
  \label{figure2.4}
  \cite{ttn_lorawan}
\end{Illustration} 

\begin{Illustration}[!ht] 
  \centering
	\includegraphics[width=1\textwidth]{figures/LoRaWAN_Network_Server_architecture.png} 
  \caption{Αρχιτεκτονική \en{LoRaWAN Network Server}.}
  \label{figure2.5}
  \cite{tti_homepage}
\end{Illustration} 







\section{Ηλεκτρικοί Υποσταθμοί και Ανάγκες Εποπτείας}
Οι ηλεκτρικοί υποσταθμοί αποτελούν κρίσιμα σημεία του ηλεκτρικού συστήματος μεταφοράς και διανομής ενέργειας. Ο ρόλος τους είναι η μετατροπή της τάσης από υψηλά επίπεδα μεταφοράς σε χαμηλότερα επίπεδα που είναι κατάλληλα για διανομή και τελική κατανάλωση. Οι υποσταθμοί μπορούν να είναι είτε πρωτεύοντες (μεταφοράς), είτε δευτερεύοντες (διανομής).

Η εποπτεία και διαχείριση των υποσταθμών περιλαμβάνει:
\begin{itemize}
  \item παρακολούθηση ηλεκτρικών παραμέτρων όπως ρεύμα, τάση, ισχύς και συχνότητα ανά φάση,
  \item ανίχνευση βλαβών ή ανομαλιών (π.χ. υπερφόρτιση, βυθίσεις τάσης),
  \item έλεγχο λειτουργικών μονάδων όπως διακόπτες ισχύος και προστατευτικά ρελέ,
  \item λήψη αποφάσεων σε πραγματικό χρόνο για την εξασφάλιση της αδιάλειπτης παροχής και της ασφάλειας του εξοπλισμού.
\end{itemize}

Παραδοσιακά, τέτοια εποπτεία γινόταν με ενσύρματες ή \en{SCADA} λύσεις υψηλού κόστους. Η ενσωμάτωση τεχνολογιών όπως το \en{LoRaWAN} επιτρέπει τη δημιουργία αποκεντρωμένων, χαμηλού κόστους και επεκτάσιμων λύσεων, κατάλληλων ακόμη και για μικρούς ή απομακρυσμένους υποσταθμούς.

\section{Συμπεράσματα}
Οι τεχνολογίες \en{LPWAN} και ιδιαίτερα το \en{LoRaWAN} παρέχουν μια αποτελεσματική λύση για τηλεμετρικές εφαρμογές σε περιβάλλοντα όπου απαιτείται χαμηλή κατανάλωση ισχύος και μεγάλη απόσταση μετάδοσης. Το θεωρητικό αυτό υπόβαθρο θεμελιώνει την επιλογή του \en{LoRaWAN} ως βασική τεχνολογία επικοινωνίας στο σύστημα παρακολούθησης και ελέγχου υποσταθμού που αναπτύχθηκε στο πλαίσιο της παρούσας διπλωματικής εργασίας.
