% ========================================================
% Κεφάλαιο 2: Τεχνολογίες LPWAN και το Πρωτόκολλο LoRaWAN
% ========================================================


\chapter{Τεχνολογίες \en{LPWAN} και το Πρωτόκολλο \en{LoRaWAN}}
\label{chap:lpwan}


% -------------------------------
% Ενότητα 2.1: Εισαγωγή στα LPWAN
% -------------------------------


\section{Εισαγωγή στα \en{LPWAN}}

Η διαρκώς αυξανόμενη ανάγκη για απομακρυσμένη και ταυτόχρονα αποδοτική, ως προς την ενέργεια, επικοινωνία 
μεταξύ έξυπνων συσκευών και αισθητήρων έχει οδηγήσει στην εμφάνιση και εξέλιξη μιας νέας γενιάς 
ασύρματων τεχνολογιών, γνωστών ως \en{Low Power Wide Area Networks (LPWAN)}. Οι 
τεχνολογίες \en{LPWAN} επιτρέπουν την αποστολή μικρών σε ποσότητα δεδομένων σε μεγάλες
 αποστάσεις με εξαιρετικά χαμηλή κατανάλωση ενέργειας, καθιστώντας τις ιδανικές για 
 εφαρμογές \en{Internet of Things (IoT)}, όπου η διάρκεια ζωής της μπαταρίας και η 
 αξιοπιστία είναι κρίσιμοι παράγοντες.

Σε αντίθεση με τις τεχνολογίες \en{Wi-Fi} ή \en{Bluetooth}, οι οποίες είναι σχεδιασμένες 
για υψηλούς ρυθμούς μετάδοσης δεδομένων σε μικρές αποστάσεις, τα \en{LPWAN} είναι 
προσανατολισμένα στην υποστήριξη ενός μεγάλου αριθμού συσκευών, με δυνατότητα μετάδοσης 
δεδομένων σε αποστάσεις που υπερβαίνουν τα 10 χιλιόμετρα, σε ανοικτό πεδίο, και σε 
συχνότητες που βρίσκονται στο μη αδειοδοτημένο φάσμα (\en{unlicensed spectrum}). Το 
σημαντικότερο, μάλιστα, όφελος έναντι άλλων τεχνολογιών μετάδωσης πληροφορίας μεγάλου έυρους
(όπως το φάσμα κινητής τηλεφωνίας \en{3G, 4G ή 5G}), είναι η ελάχιστη ενέργεια που απαιτείται 
για την τροφοδοσία των αντίστοιχων συσκευών \cite{ICTexpress}.

Οι πιο διαδεδομένες τεχνολογίες \en{LPWAN} είναι οι εξής:

\begin{itemize}
    \item \textbf{\en{NB-IoT (Narrowband Internet of Things)}} – αποτελεί τεχνολογία ασύρματης 
    επικοινωνίας και χαμηλής ισχύος, βασισμένη στο \en{LTE (Long-Term Evolution)}, η οποία λειτουργεί 
    στο αδειοδοτημένο φάσμα και προσφέρει αξιόπιστη κάλυψη εντός κτιρίων (\en{deep indoor penetration}). 
    Αναπτύχθηκε από το \en{3rd Generation Partnership Project (3GPP)} και υποστηρίζεται από το 
    πρότυπο \en{3GPP Release 13.} Έχει σχεδιαστεί για εφαρμογές με ανάγκες μαζικής συνδεσιμότητας 
    και μικρού όγκου δεδομένων, όπως μετρητές νερού ή αερίου. Η χαμηλή κατανάλωση ενέργειας που 
    απαιτέιται για την λειτουργία των συσκευών έχει ως αποτέλεσμα η διάρκειά λειτουργίας τους να 
    φτάνει έως και 10 χρόνια, με την χρήση μίας μόνο μπαταρίας. \cite{telit2019nbiot}
    
    \item \textbf{\en{LTE-M (LTE Cat-M1)}} – επίσης βασίζεται στο \en{LTE} και προσφέρει 
    υψηλότερους ρυθμούς μετάδοσης δεδομένων από το \en{NB-IoT} (έως και 1 \en{Mbps}), 
    διατηρώντας ωστόσο εξίσου χαμηλή κατανάλωση \cite{zipit2023ltem}. Είναι κατάλληλο για φορητές εφαρμογές 
    που απαιτούν αμφίδρομη επικοινωνία σε πραγματικό χρόνο, όπως η παρακολούθηση οχημάτων, 
    οι φορητές ιατρικές συσκευές και οι φορητοί αισθητήρες. Ένα από τα κύρια πλεονεκτήματά 
    του είναι η υποστήριξη κινητικότητας, επιτρέποντας την απρόσκοπτη μετάβαση μεταξύ κυψελών, 
    καθώς και η δυνατότητα φωνητικής επικοινωνίας μέσω \en{VoLTE}. \cite{hosangadi2019system}
    
    \item \textbf{\en{LoRa} και \en{LoRaWAN}} – πρόκειται για το πιο διαδεδομένο πρωτόκολλο 
    σε μη-αδειοδοτη-μένο φάσμα (π.χ. 868~\en{MHz} στην Ευρώπη), με κύρια πλεονεκτήματα την 
    ευκολία υλοποίησης, τη μεγάλη αυτονομία (έως και 10 έτη), τη χαμηλή κατανάλωση ισχύος 
    και την υψηλή ευελιξία ανάπτυξης μέσω ιδιωτικών ή δημόσιων δικτύων. Η τεχνολογία \en{LoRa} 
    αναπτύχθηκε αρχικά από τη γαλλική \en{Cycleo} και κατοχυρώθηκε από τη \en{Semtech}, 
    ενώ το \en{LoRaWAN} αναπτύσσεται και προτυποποιείται από τη \en{LoRa Alliance}. \cite{semtech_lora_lorawan}
\end{itemize}

Τα δίκτυα \en{LPWAN} ενσωματώνονται όλο και περισσότερο σε κρίσιμες υποδομές, όπως είναι τα συστήματα 
παρακολούθησης ενέργειας, γεωργίας ακριβείας, έξυπνων πόλεων και βιομηχανικής αυτοματοποίησης, 
προσφέροντας λύσεις υψηλής κάλυψης, ανθεκτικότητας και χαμηλού κόστους εγκατάστασης.


% -------------------------------
% Ενότητα 2.2: Σύγκριση Τεχνολογιών LPWAN
% -------------------------------


\section{Σύγκριση Τεχνολογιών \en{LPWAN}}
Παρακάτω γίνεται η σύγκριση των διαφόρων τεχνολογιών \en{LPWAN}:

\begin{Illustration}[!ht] \centering
	\includegraphics[width=0.95\textwidth]{figures/IoT_techs.png} 
    \caption{Σύγκριση τεχνολογιών ασύρματης επικοινωνίας (\en{LPWAN}) ως προς τον ρυθμό μετάδοσης, 
    την κατανάλωση ενέργειας, την εμβέλεια και το κόστος.}
    \label{figure2.1}
    \cite{saft2023iot}
\end{Illustration} 

Οι τεχνολογίες \en{LPWAN} αποτελούν βασικό πυλώνα για την υλοποίηση ενεργειακά αποδοτικών και 
μεγάλης εμβέλειας εφαρμογών \en{IoT}, με διαφορετικές προσεγγίσεις ως προς το φάσμα λειτουργίας, 
την κατανάλωση ισχύος, την κινητικότητα και τη δυνατότητα υποστήριξης ποικίλων τύπων δεδομένων. Ακολούθως 
παρουσιάζονται τα προαναφερθέντα χαρακτηριστικά για τις τρεις πιο διαδεδομένες τεχνολογίες \en{LPWAN:}

\begin{table}[H]
\centering
\renewcommand{\arraystretch}{1.5}
\begin{tabular}{|p{4cm}|p{3.4cm}|p{3.4cm}|p{3.4cm}|}
\hline
\textbf{\textgreek{Παράμετρος}} & \textbf{\en{NB-IoT}} & \textbf{\en{LTE-M}} & \textbf{\en{LoRa}} \\
\hline
\textbf{\textgreek{Τυποποίηση}} & \en{3GPP} & \en{3GPP} & \en{LoRa Alliance} \\
\hline
\textbf{\textgreek{Διαμόρφωση}} & \en{QPSK, 16QAM} & \en{QPSK, 16QAM} & \en{CSS} (\en{Chirp Spread Spectrum}) \\
\hline
\textbf{\textgreek{Φάσμα Συχνοτήτων}} & \en{Licensed} \en{3GPP} (180 \en{kHz}) & \en{Licensed} \en{3GPP} (1.4 \en{MHz}) & \en{Unlicensed ISM} (\en{EU 868 MHz}) \\
\hline
\textbf{\textgreek{Κάλυψη (\en{Link Budget})}} & 151 \en{dB} & 146 \en{dB} & 154 \en{dB} \\
\hline
\textbf{\textgreek{Μέγιστο Φορτίο}} & 1600 \en{bytes} & 1000 \en{bytes} & 242 \en{bytes} \\
\hline
\textbf{\textgreek{Διάρκεια Ζωής Μπαταρίας}} & έως 10 έτη & έως 2 έτη & έως 10 έτη \\
\hline
\textbf{\textgreek{Ταχύτητα Μετάδοσης}} & 200 \en{kbps} & 1 \en{Mbps} & 50 \en{kbps} \\
\hline
\textbf{\textgreek{Αμφίδρομη Επικοινωνία}} & Ναι & Ναι & Ναι \\
\hline
\textbf{\textgreek{Ασφάλεια}} & \en{3GPP (128-256 bit)} & \en{3GPP (128-256 bit)} & \en{AES (128 bit)} \\
\hline
\textbf{\textgreek{Κινητικότητα}} & \en{<100 km/h} & \en{<300 km/h} & Ναι \\
\hline
\textbf{\en{QoS}} & Ναι & Ναι & Όχι \\
\hline
\end{tabular}
\caption{\textgreek{Συγκριτικός πίνακας τεχνολογιών} \en{NB-IoT}, \en{LTE-M} \textgreek{και} \en{LoRa}}
\label{tab:lpwan-comparison}
\cite{ICTexpress}, \cite{adelantado2017understanding}, \cite{zipit2023ltem}, \cite{semtech_lora_lorawan}
\end{table}

Η ανάλυση των επιμέρους χαρακτηριστικών των τριών τεχνολογιών δείχνει πως κάθε μία εξυπηρετεί 
διαφορετικές ανάγκες, ανάλογα με το σενάριο χρήσης και τις απαιτήσεις της εκάστοτε εφαρμογής.

Ξεκινώντας από την κινητικότητα, το \en{LTE-M} υπερέχει με διαφορά, καθώς υποστηρίζει μετακινήσεις 
με ταχύτητες έως και 300 \en{km/h} και δυνατότητα \en{handover} μεταξύ κυψελών, κάτι που καθιστά 
εφικτή την αξιόπιστη σύνδεση σε περιπτώσεις όπως είναι η παρακολούθηση οχημάτων ή \en{drones} εν κινήσει. 
Από την άλλη μεριά, το \en{NB-IoT} παρέχει περιορισμένη κινητικότητα και είναι περισσότερο κατάλληλο 
για στατικές συσκευές, ενώ το \en{LoRa} μπορεί να χρησιμοποιηθεί για κινητές εφαρμογές μόνο 
αν βρίσκεται εντός εμβέλειας ενός διαθέσιμου \en{gateway}, γεγονός που περιορίζει τη χρήση 
του σε δυναμικά περιβάλλοντα.

Στο πεδίο της μετάδοσης δεδομένων, το \en{LTE-M} προσφέρει τους υψηλότερους ρυθμούς 
(1 \en{Mbps}), καθώς και υποστήριξη φωνητικής επικοινωνίας μέσω \en{VoLTE}, 
χαρακτηριστικά που απουσιάζουν από τις άλλες δύο τεχνολογίες. Αντίθετα, το \en{LoRa} 
περιορίζεται σε πολύ χαμηλούς ρυθμούς (50 \en{kbps}) και είναι σχεδιασμένο 
κυρίως για απλές, σποραδικές μεταδόσεις.

Όσον αφορά το φάσμα λειτουργίας, τόσο το \en{NB-IoT} όσο και το \en{LTE-M} αξιοποιούν το
αδειοδοτημένο φάσμα, γεγονός που προσφέρει πιο σταθερή σύνδεση, μικρότερο λανθάνοντα χρόνο 
και καλύτερη ποιότητα υπηρεσίας (\en{QoS}). Αυτά τα χαρακτηριστικά είναι κρίσιμα για εφαρμογές 
όπως \en{POS terminals}, όπου απαιτείται γρήγορη και αξιόπιστη μετάδοση συναλλαγών. Από την άλλη, 
το \en{LoRa} λειτουργεί σε μη αδειοδοτημένο φάσμα, που αν και μειώνει το κόστος, υπόκειται σε 
περιορισμούς όπως το \en{duty cycle} και το \en{fair access policy}, μειώνοντας, έτσι, την αξιοπιστία 
σε περιβάλλοντα όπου υπάρχει υψηλή κίνηση δεδομένων.

Σε όρους ενεργειακής απόδοσης, το \en{LoRa} και το \en{NB-IoT} είναι εμφανώς πιο αποτελεσματικά, 
υποστηρίζοντας διάρκεια μπαταρίας έως και 10 έτη. Το \en{LTE-M}, λόγω της μεγαλύτερης κατανάλωσης 
ισχύος, τείνει να έχει μικρότερη διάρκεια ζωής, συνήθως μεταξύ 1-2 ετών, κάτι που πρέπει να 
ληφθεί υπόψη σε εφαρμογές όπου η συντήρηση των κόμβων δεν είναι εύκολη.

Ως προς την εμπορική απήχηση, οι τεχνολογίες του \en{3GPP} (\en{NB-IoT} και \en{LTE-M}) 
προωθούνται κυρίως μέσω παρόχων κινητής τηλεφωνίας και ενσωματώνονται σε λύσεις ευρείας 
κλίμακας από τη βιομηχανία \cite{gsma2022mobileiot}. Αντίθετα, το \en{LoRaWAN}, μέσω της \en{LoRa Alliance}, διατίθεται 
ευρύτερα για αποκεντρωμένες και ιδιωτικές αναπτύξεις, γεγονός που το έχει καταστήσει ιδιαίτερα 
δημοφιλές σε αγροτικές εφαρμογές, αισθητήρες έξυπνων κτιρίων και περιβαλλοντική παρακολούθηση \cite{loraalliance2023report}.

Συνοψίζοντας, δεν υπάρχει μία «καλύτερη» τεχνολογία για κάθε περίπτωση. Η επιλογή εξαρτάται 
από το εκάστοτε έργο και τους στόχους του: αν προέχει η κινητικότητα και η χαμηλή καθυστέρηση, 
το \en{LTE-M} είναι πιο κατάλληλο· αν ζητούμενο είναι η μεγάλη διάρκεια ζωής και το χαμηλό κόστος, 
το \en{LoRa} αποτελεί ιδανική επιλογή· ενώ το \en{NB-IoT} είναι ενδιάμεση λύση για στατικές 
εφαρμογές με αξιόπιστο σήμα και μεγάλη πυκνότητα κόμβων. Η τελική απόφαση λαμβάνει υπόψη 
τεχνικούς περιορισμούς, απαιτήσεις απόδοσης και το οικονομικό κόστος υλοποίησης.


% -------------------------------
% Ενότητα 2.3: Τεχνολογία LoRa
% -------------------------------

\vspace{3em}
\section{Τεχνολογία \en{LoRa}}


%%%%   Υποενότητα 2.3.1: Γενική Επισκόπηση της Τεχνολογίας \en{LoRa}   %%%%


\subsection{Γενική Επισκόπηση της Τεχνολογίας \en{LoRa}}

Ξεκινώντας με μία μικρή ιστορική αναδρομή, η τεχνολογία \en{LoRa (Long Range)} αναπτύχθηκε αρχικά το 2009 από 
δύο φίλους, τους \en{Nicolas Sornin} και \en{Olivier Seller}, όπου στην συνέχεια συμμετείχε στην ομάδα 
και ένας τρίτος συνεργάτης, ο \en{François Sforza} και όλοι μαζί δημιούργησαν τη γαλλική εταιρεία \en{Cycleo} 
το 2010. Δύο χρόνια μετά (2012), η \en{Cycleo} εξαγοράστηκε από την αμερικανική εταιρεία \en{Semtech} 
\cite{semtech2020lora}. Η τεχνολογία αυτή λειτουργεί αποκλειστικά στο φυσικό επίπεδο 
(\en{Physical Layer, PHY}) του μοντέλου αναφοράς \en{OSI (Open Systems Interconnection model)},
και βασίζεται στη διαμόρφωση εξάπλωσης φάσματος τύπου \en{Chirp Spread Spectrum (CSS)}, 
που επιτρέπει την αξιόπιστη και χαμηλής κατανάλωσης μετάδοση δεδομένων σε μεγάλες αποστάσεις. 
Χρησιμοποιεί το ελεύθερο φάσμα ραδιοσυχνοτήτων \en{ISM (Industrial, Scientific and Medical)}, 
με κύρια μπάντα συχνοτήτων στην Ευρώπη τα \en{868 MHz} \cite{semtech_lora_lorawan}. 

Η τεχνολογία \en{LoRa} παρέχει σημαντική αντοχή σε παρεμβολές, μιας και χρησιμοποιεί προσαρμοστικό ρυθμό 
μετάδοσης \en{(Adaptive Data Rate - ADR)}, ενώ παράλληλα παρουσιάζει και υψηλή ευαισθησία δεκτών, γεγονότα που 
επιτρέπουν την επικοινωνία ακόμα και σε συνθήκες με μεγάλο θόρυβο από το περιβάλλον. Το εύρος ζώνης που 
χρησιμοποιείται (\en{Bandwidth, BW}) είναι συνήθως \en{125 kHz}, \en{250 kHz} ή \en{500 kHz}, ανάλογα με 
τις ανάγκες της εφαρμογής. Παράλληλα, η διαμόρφωση χρησιμοποιεί διαφορετικούς παράγοντες εξάπλωσης 
(\en{Spreading Factors, SF}) από 7 έως 12, που επηρεάζουν τον ρυθμό μετάδοσης δεδομένων και την εμβέλεια 
του σήματος \cite{ttn_lorawan}.


\begin{Illustration}[!ht] 
  \centering
	\includegraphics[width=0.7\textwidth]{figures/OSI_LoRa.png} 
  \caption{Μοντέλο \en{OSI} σε αντιστοίχιση με τα \en{LoRa} και \en{LoRaWAN} επίπεδα.}
  \label{figure2.2}
  \cite{semtech_lora_lorawan}
\end{Illustration}


%%%%   Υποενότητα 2.3.2: Ραδιοφωνική Διάδοση   %%%%


% \subsection{Ραδιοφωνική Διάδοση}

% Η τεχνολογία \en{LoRa} εκμεταλλεύεται το μη αδειοδοτημένο φάσμα ραδιοσυχνοτήτων \en{ISM} (συνήθως 
% τα 868 \en{MHz} στην Ευρώπη και 915 \en{MHz} στις ΗΠΑ), προσφέροντας πλεονεκτήματα στην εμβέλεια και την αξιοπιστία 
% του σήματος λόγω της χαμηλής συχνότητας μετάδοσης. Γενικά, όσο χαμηλότερη είναι η συχνότητα, τόσο μικρότερη 
% είναι η απόσβεση και επομένως τόσο μεγαλύτερη είναι η δυνατότητα διείσδυσης σε εμπόδια, όπως κτίρια ή βλάστηση \cite{SemtechModulationBasics}.

% Η εξασθένηση του σήματος στον ελεύθερο χώρο μπορεί να μοντελοποιηθεί με την εξίσωση διάδοσης \en{Friis}:
% \begin{equation}
% P_r = P_t G_t G_r \left(\frac{\lambda}{4\pi d}\right)^2
% \end{equation}
% όπου $P_r$ είναι η ισχύς του λαμβανόμενου σήματος, $P_t$ η ισχύς εκπομπής, $G_t, G_r$ τα κέρδη των κεραιών 
% εκπομπού και δέκτη αντίστοιχα, $d$ η απόσταση μεταξύ πομπού και δέκτη, και $\lambda$ το μήκος κύματος. Σε 
% μορφή δεκαδικών λογαρίθμων (\en{dB}), η απώλεια διάδοσης (\en{Path Loss, PL}) δίνεται ως:
% \begin{equation}
% PL(dB) = 20\log_{10}(d) + 20\log_{10}(f) + 32.45
% \end{equation}
% όπου $f$ η συχνότητα σε \en{MHz} και $d$ η απόσταση σε χιλιόμετρα \cite{SemtechModulationBasics}. Από την παραπάνω εξίσωση είναι 
% εμφανές ότι μεγαλύτερες αποστάσεις και υψηλότερες συχνότητες επιφέρουν σημαντική αύξηση απωλειών. 

% Στο πραγματικό περιβάλλον, πρόσθετα φαινόμενα όπως αποσβέσεις λόγω τοίχων, διάθλαση και σκέδαση επηρεάζουν τη στάθμη του σήματος. 
% Παρά ταύτα, τα σήματα \en{LoRa} παρουσιάζουν αξιοσημείωτη ανθεκτικότητα στη διάλειψη πολλαπλών διαδρομών \en{(multipath fading)}. 
% Αυτό οφείλεται τόσο στο μεγαλύτερο μήκος κύματος (π.χ. περίπου 34 \en{cm} στα 868 \en{MHz}) που μπορεί να διαπεράσει εμπόδια, όσο και στην ίδια 
% τη διαμόρφωση ευρέως φάσματος που «εξομαλύνει» τις διακυμάνσεις πολλαπλών διαδρομών. Ως αποτέλεσμα, ένα δίκτυο \en{LoRa} μπορεί 
% να καλύψει μεγάλες αποστάσεις συγκριτικά με άλλες τεχνολογίες παρόμοιας ισχύος. Για παράδειγμα, υπό συνθήκες οπτικής επαφής 
% έχουν αναφερθεί αποστάσεις άνω των 10–15 \en{km} για συνδέσμους \en{LoRaWAN}, ενώ σε ακραίο πείραμα κατορθώθηκε επικοινωνία σε εύρος 
% ~832 \en{km} με ισχύ μόλις 25 \en{mW} σε υψηλό υψόμετρο, ενώ πρόσφατα καταρίφθηκε αυτό το ρεκόρ απόστασης από νέο πείραμα, όπου επιτεύχθηκε 
% επικοινωνία σε απόσταση 1336 \en{km} \cite{TTNRecord}. Φυσικά, σε πυκνό αστικό περιβάλλον η εμβέλεια μειώνεται (π.χ. τυπικά 2–5 \en{km}), 
% όμως ακόμη και τότε το \en{LoRa} παρουσιάζει αξιόπιστη κάλυψη χάρη στο υψηλό ισοζύγιο ζεύξης του. 

% \subsubsection{Ισοζύγιο Ζεύξης}

% Το ισοζύγιο ζεύξης (\en{Link Budget}) είναι μία θεμελιώδης παράμετρος αξιολόγησης της απόδοσης ενός ασύρματου τηλεπικοινωνιακού 
% συστήματος. Εκφράζει ουσιαστικά τη διαφορά, σε \en{decibel (dB)}, ανάμεσα στην ισχύ του σήματος που εκπέμπεται από τον πομπό και την 
% ελάχιστη ισχύ που απαιτείται για να φτάσει στον δέκτη, ώστε να πραγματοποιηθεί αξιόπιστη επικοινωνία. Με απλά λόγια, είναι το 
% άθροισμα όλων των κερδών (π.χ. ισχύς εκπομπής, κέρδος κεραίας) αφαιρουμένου του συνόλου των απωλειών (π.χ. απόσβεση διαδρομής, 
% ατμοσφαιρικές απώλειες, κλπ.) στην πορεία του σήματος από τον πομπό στον δέκτη. 
% \begin{equation}
% Link\ Budget\ (dB) = P_{TX}(dBm) + G_{TX}(dBi) + G_{RX}(dBi) - Sensitivity_{RX}(dBm) - Losses_{misc}(dB)
% \end{equation}
% όπου $P_{TX}$ είναι η ισχύς εκπομπής, $G_{TX}, G_{RX}$ είναι τα κέρδη κεραίας, $Sensitivity_{RX}$ είναι η ευαισθησία του δέκτη, 
% $Losses_{misc}$ είναι οι λοιπές απώλειες.

% Το ισοζύγιο ζεύξης ενός συστήματος \en{LoRa} είναι ιδιαίτερα υψηλό (τυπικά μπορεί να φτάσει έως και 154 \en{dB}, ανάλογα με τις παραμέτρους της διαμόρφωσης) επιτρέποντας την ανίχνευση 
% σημάτων πολύ χαμηλής στάθμης στο δέκτη. Αυτό προκύπτει από τη συνδυαστική συμβολή: (α) της χρήσης χαμηλού ρυθμού μετάδοσης (διάχυση σήματος 
% σε μεγάλο χρόνο), που παρέχει κέρδος επεξεργασίας έναντι του θορύβου, και (β) της ευαισθησίας των δεκτών \en{LoRa}, η οποία 
% είναι από τις μεγαλύτερες συγκριτικά με άλλες τεχνολογίες \en{LPWAN}. Για δεδομένη ισχύ εκπομπής και ρυθμό, το \en{LoRa} 
% επιτυγχάνει θεωρητικά έως και 4 φορές μεγαλύτερη εμβέλεια σε σχέση με τη παραδοσιακή διαμόρφωση \en{FSK (Frequency Shift Keying)} 
% αντίστοιχης κατηγορίας. 

% \begin{Illustration}[!ht] 
%   \centering
% 	\includegraphics[width=1\textwidth]{figures/LoRa_vs_FSK.png} 
%   \caption{Σύγκριση ευαισθησίας \en{LoRa} και \en{FSK}.}
%   \label{figure2.3}
%   \cite{SemtechModulationBasics} 
% \end{Illustration}

% Χαρακτηριστικά, το \en{LoRa} μπορεί να διατηρήσει αξιόπισμη επικοινωνία ακόμη και με λόγο σήματος προς θόρυβο \en{(SNR)} αρνητικό, 
% δηλαδή με το σήμα θαμμένο κάτω από το φάσμα του θορύβου. Αυτό το πλεονέκτημα οφείλεται στη διαμόρφωση ευρέως φάσματος και στη 
% μεγάλη διάρκεια συμβόλου που επιτρέπει στο δέκτη να συνδυάσει ενέργεια σήματος σε χρόνο και συχνότητα. Συνεπώς, στο επίπεδο 
% διάδοσης, το \en{LoRa} αξιοποιεί τις φυσικές ιδιότητες των συχνοτήτων \en{sub-GHz} και την ανθεκτική διαμόρφωσή του για να μεγιστοποιήσει 
% την ακτίνα κάλυψης και την αξιοπιστία επικοινωνίας ακόμη και σε αντίξοες συνθήκες μετάδοσης \cite{SemtechModulationBasics}. \pagebreak

% \subsubsection{Ευαισθησία δεκτών, αντοχή σε παρεμβολές}

% Η υψηλή ευαισθησία του δέκτη είναι θεμελιώδες χαρακτηριστικό των συστημάτων \en{LoRa}. Η
% θεωρητική ευαισθησία ορίζεται από το θερμικό όριο θορύβου και το εύρος ζώνης λήψης. Για έναν
% δέκτη με συντελεστή θορύβου $NF$ (σε \en{dB}) και απαιτούμενο \en{SNR} κατώφλι $SNR_{\min}$ για τη
% διαμόρφωση, η ελάχιστη ισχύς σήματος που μπορεί να ανιχνευθεί δίνεται προσεγγιστικά (σε \en{dBm})
% από: 
% \begin{equation}
% S_{\text{\en{min}}} \approx -174 + 10 \log_{10}(BW) + NF + SNR_{\text{\en{min}}},
% \end{equation}
% όπου $-174$ \en{dBm/Hz} είναι η φασματική πυκνότητα ισχύος του θερμικού θορύβου σε θερμοκρασία
% δωματίου. Για τους τυπικούς δέκτες \en{LoRa} (π.χ. \en{chip SX1276}) το $NF$ είναι περίπου 6 \en{dB}
% \cite{SemtechModulationBasics}. 

% Λαμβάνοντας $BW=125$ \en{kHz} και $NF=6$ \en{dB}, η ευαισθησία εξαρτάται
% κυρίως από το απαιτούμενο $SNR_{\min}$ του εκάστοτε \en{Spreading Factor}. Πειραματικά και από
% προσομοιώσεις έχει βρεθεί ότι το \en{SF7} απαιτεί περίπου $SNR_{\min}\approx -7.5$ \en{dB} για αξιόπιστη
% ανίχνευση, ενώ το \en{SF12} απαιτεί $SNR_{\min}\approx -20$ dB \cite{SemtechModulationBasics}. Υπό
% αυτές τις συνθήκες, προκύπτει ευαισθησία δέκτη περίπου $S_{\min}\approx -125$ \en{dBm} για \en{SF7} και
% μέχρι $S_{\min}\approx -137$ \en{dBm} για \en{SF12} στο κανάλι (125 kHz). Αυτές οι τιμές είναι εξαιρετικά
% χαμηλές, εξηγώντας την δυνατότητα του \en{LoRa} να λαμβάνει σήματα σε αποστάσεις πολλών
% χιλιομέτρων. Για σύγκριση, ένα τυπικό δίκτυο \en{Wi-Fi} απαιτεί σήματα ισχύος άνω των -90 \en{dBm}, ενώ το
% \en{LoRa} μπορεί να λειτουργήσει με σήματα χιλιάδες φορές πιο ασθενή.

% Η αντοχή του \en{LoRa} σε παρεμβολές είναι επίσης αξιοσημείωτη. Ως διαμόρφωση ευρέος φάσματος,
% παρουσιάζει εγγενή ανοχή σε στενής ζώνης παρεμβολές: ένα παρεμβάλλον σήμα περιορισμένου
% εύρους (π.χ. θόρυβος σε μια στενή συχνότητα) θα επηρεάσει μόνο ένα μικρό μέρος του \en{chirp},
% επιτρέποντας στο υπόλοιπο φάσμα του συμβόλου να συνεισφέρει στην ορθή αποκωδικοποίηση.
% Επιπλέον, λόγω του μεγάλου $T_{\text{sym}}$, το \en{LoRa} δίνει δυνατότητα χρονικής ολοκλήρωσης
% της ισχύος σήματος, γεγονός που βελτιώνει τον αντι-παρεμβολικό λόγο. Μελέτες έχουν δείξει ότι το
% \en{LoRa} διατηρεί αξιόπιστη επικοινωνία ακόμα και υπό έντονο περιβάλλον θορύβου και πολλαπλών
% διαδρομών, όπου άλλες τεχνολογίες αποτυγχάνουν \cite{Staniec2018}. Οι δοκιμές του \en{Kamil Staniec} κ.ά.
% (2018) έδειξαν ότι σε παρουσία ισχυρών παρεμβολών, το \en{LoRa} εμφανίζει μεν αύξηση του ρυθμού
% σφαλμάτων, αλλά παραμένει λειτουργικό σε \en{SNR} όπου ένα σύστημα χωρίς διασπορά θα είχε
% καταρρεύσει πλήρως \cite{Staniec2018}. Αυτό αποδίδεται τόσο στο κέρδος επεξεργασίας που
% προσφέρει η διαμόρφωση \en{CSS}, όσο και στον διορθωτικό κώδικα που διορθώνει μεμονωμένα
% σφάλματα bits λόγω παρεμβολών. 

% Ωστόσο, αξίζει να σημειωθεί ότι η παρεμβολή μεταξύ συσκευών
% \en{LoRa} του ίδιου δικτύου είναι δυνατή σε σενάρια υψηλής πυκνότητας: αν πολλοί κόμβοι
% μεταδίδουν ταυτόχρονα στον ίδιο δίαυλο και με ίδιο \en{SF}, οι συγκρούσεις πλαισίων θα οδηγήσουν σε
% απώλειες πακέτων (λόγω του \en{ALOHA-based} πρωτοκόλλου πρόσβασης του \en{LoRaWAN}). Η
% χρήση διαφορετικών καναλιών συχνοτήτων και η ορθογωνιότητα των \en{SF} μετριάζουν αυτό το
% πρόβλημα, όπως αναφέρθηκε, αλλά δεν το εξαλείφουν πλήρως. Συνεπώς, η ανθεκτικότητα του
% \en{LoRa} σε παρεμβολές είναι υψηλή συγκριτικά με στενόζωνες διαμορφώσεις, όμως η πραγματική
% απόδοση εξαρτάται και από τον προσεκτικό σχεδιασμό του δικτύου (κατανομή καναλιών, \en{SF}, χρόνου
% εκπομπής ανά κόμβο κ.λπ.)

% \subsubsection{Ζώνη \en{Fresnel}}

% Η ζώνη \en{Fresnel} αποτελεί μια ελλειψοειδή περιοχή γύρω από την ευθεία γραμμή μεταξύ πομπού και δέκτη, όπου η παρουσία 
% εμποδίων μπορεί να προκαλέσει διάθλαση και συμβολή των ραδιοκυμάτων, επηρεάζοντας αρνητικά την ποιότητα της επικοινωνίας. 
% Η ακτίνα της πρώτης ζώνης \en{Fresnel} ($F_1$) υπολογίζεται από τη σχέση:
% \begin{equation}
% F_1 = \sqrt{\frac{\lambda d_1 d_2}{d_1 + d_2}}
% \end{equation}
% όπου $d_1, d_2$ είναι οι αποστάσεις από το σημείο παρεμβολής στον πομπό και τον δέκτη αντίστοιχα, και 
% $\lambda$ το μήκος κύματος. Για αξιόπιστη επικοινωνία, θεωρείται ιδανικό να είναι τουλάχιστον το 60\% 
% της πρώτης ζώνης \en{Fresnel} ελεύθερο από εμπόδια \cite{goldsmith2005wireless}.

% Για παράδειγμα, στα 868 \en{MHz} ($\lambda \thickapprox 0.345 m$), σε μια απόσταση 5 \en{km} με εμπόδιο 
% στη μέση της απόστασης (2.5 \en{km} από κάθε πλευρά), η πρώτη ζώνη \en{Fresnel} έχει ακτίνα περίπου:

% \begin{equation}F_1 = \sqrt{\frac{0.345 \times 2500 \times 2500}{5000}} \approx 20.8\,m\end{equation}

% Η ύπαρξη εμποδίων εντός της ζώνης αυτής μειώνει δραματικά την ισχύ του λαμβανόμενου σήματος λόγω καταστροφικής 
% συμβολής μεταξύ απευθείας και ανακλώμενων κυμάτων.



% \subsubsection{Πολλαπλές διαδρομές (\en{Multipath Propagation})}

% Η διάδοση των ραδιοκυμάτων συνήθως δεν γίνεται μέσω μίας μόνο διαδρομής, αλλά μέσω πολλαπλών διαδρομών λόγω 
% ανακλάσεων, διάθλασης και διάχυσης. Αυτό το φαινόμενο οδηγεί σε διάλειψη σήματος (\en{fading}), δηλαδή έντονες 
% διακυμάνσεις στην ένταση του λαμβανόμενου σήματος. Τα σήματα \en{LoRa}, χάρη στην τεχνική διάδοσης φάσματος 
% (\en{Chirp Spread Spectrum}), είναι ιδιαίτερα ανθεκτικά σε αυτό το φαινόμενο, καθώς η ενέργεια του σήματος 
% είναι απλωμένη σε μεγάλη περιοχή του φάσματος και χρόνου, επιτρέποντας στον δέκτη να συνδυάζει αποτελεσματικά 
% την ενέργεια των διαφορετικών διαδρομών και να ανακτά τα δεδομένα \cite{Staniec2018}.

% \subsubsection{Φαινόμενο Doppler (\en{Doppler Effect})}

% Το φαινόμενο Doppler αναφέρεται στη μεταβολή της συχνότητας ενός σήματος όταν υπάρχει σχετική 
% κίνηση μεταξύ πομπού και δέκτη. Η μετατόπιση Doppler ($\Delta f$) για ένα σήμα που μεταδίδεται από 
% κινούμενο πομπό προς σταθερό δέκτη δίνεται από:

% \begin{equation}
% \Delta f = f_c \frac{v}{c}
% \end{equation}

% όπου $f_c$ η φέρουσα συχνότητα, $v$ η σχετική ταχύτητα μεταξύ πομπού και δέκτη, και $c$ η ταχύτητα του 
% φωτός ($3 \times 10^8 \, m/s$).

% Η διαμόρφωση \en{LoRa} παρουσιάζει ιδιαίτερη αντοχή στο φαινόμενο Doppler λόγω της ευρέως 
% φάσματος φύσης των σημάτων chirp. Το εύρος ζώνης που χρησιμοποιεί το \en{LoRa} (τυπικά 125~kHz ή 
% περισσότερο) και η χρήση chirp που καλύπτουν όλο το εύρος συχνοτήτων παρέχουν ανθεκτικότητα στις 
% μικρές μετατοπίσεις συχνότητας. Για παράδειγμα, με $f_c = 868~MHz$ και ταχύτητα $v = 100~km/h \approx 27.8~m/s$:

% \[
% \Delta f = 868 \times 10^6 \cdot \frac{27.8}{3 \times 10^8} \approx 80~Hz
% \]

% Αυτή η μετατόπιση είναι αμελητέα σε σχέση με το εύρος ζώνης των 125~kHz, καθιστώντας το \en{LoRa} 
% ιδανικό ακόμη και σε εφαρμογές κινητών συσκευών (π.χ. αισθητήρες οχημάτων) 
% \cite{SemtechModulationBasics}.

% \subsubsection{Ατμοσφαιρικά και περιβαλλοντικά φαινόμενα}

% Τέλος, στην διάδοση των ραδιοκυμάτων σημαντική είναι και η επίδραση περιβαλλοντικών παραγόντων, όπως οι 
% μεταβολές στην ατμοσφαιρική υγρασία, βροχή, ή και θερμοκρασιακές διακυμάνσεις, που μπορεί να οδηγήσουν σε 
% πρόσθετες απώλειες (όπως το φαινόμενο της ατμοσφαιρικής απορρόφησης). Ωστόσο, σε συχνότητες sub-GHz (868 MHz), 
% αυτές οι επιδράσεις είναι σχετικά μικρές και δεν επηρεάζουν σημαντικά την αξιοπιστία του σήματος.

% Συνοψίζοντας, η επιλογή των συχνοτήτων του \en{LoRa}, η τεχνική διαμόρφωσης και η υψηλή ευαισθησία των δεκτών 
% καθιστούν την τεχνολογία αυτή ιδιαίτερα ανθεκτική σε προβλήματα που προκύπτουν κατά τη διάδοση των ραδιοκυμάτων, 
% καθιστώντας την ιδανική για εφαρμογές μεγάλης εμβέλειας και χαμηλής ισχύος.













% \subsection{Ραδιοφωνική Διάδοση}

% Η τεχνολογία \en{LoRa} εκμεταλλεύεται το μη αδειοδοτημένο φάσμα ραδιοσυχνοτήτων \en{ISM} 
% (συνήθως τα 868 \en{MHz} στην Ευρώπη και 915 \en{MHz} στις ΗΠΑ), προσφέροντας πλεονεκτήματα 
% στην εμβέλεια και την αξιοπιστία του σήματος λόγω της χαμηλής συχνότητας μετάδοσης. Γενικά, 
% όσο χαμηλότερη είναι η συχνότητα, τόσο μικρότερη είναι η απόσβεση και επομένως τόσο μεγαλύτερη 
% είναι η δυνατότητα διείσδυσης σε εμπόδια, όπως κτίρια ή βλάστηση \cite{SemtechModulationBasics}.

% Η εξασθένηση του σήματος στον ελεύθερο χώρο μοντελοποιείται με την εξίσωση διάδοσης \en{Friis}:
% \begin{equation}
% P_r = P_t G_t G_r \left(\frac{\lambda}{4\pi d}\right)^2
% \end{equation}
% όπου $P_r$ είναι η ισχύς λήψης, $P_t$ η ισχύς εκπομπής, $G_t, G_r$ τα κέρδη των κεραιών πομπού 
% και δέκτη, $d$ η απόσταση μεταξύ πομπού και δέκτη και $\lambda$ το μήκος κύματος. Σε λογαριθμική 
% μορφή (\en{dB}), η απώλεια διάδοσης (\en{Path Loss, PL}) δίνεται από:
% \begin{equation}
% PL(dB) = 20\log_{10}(d) + 20\log_{10}(f) + 32.45
% \end{equation}
% όπου $f$ η συχνότητα σε \en{MHz} και $d$ η απόσταση σε \en{km} \cite{SemtechModulationBasics}. 
% Είναι προφανές ότι μεγαλύτερες αποστάσεις και υψηλότερες συχνότητες επιφέρουν μεγαλύτερες απώλειες.

% \subsubsection{Ισοζύγιο Ζεύξης (\en{Link Budget})}

% Το ισοζύγιο ζεύξης είναι θεμελιώδης παράμετρος αξιολόγησης της απόδοσης ασύρματων συστημάτων. 
% Εκφράζει τη διαφορά σε \en{dB} μεταξύ της ισχύος εκπομπής και της ελάχιστης απαιτούμενης ισχύος 
% για αξιόπιστη επικοινωνία:
% \begin{equation}
% \text{Link Budget (dB)} = P_{TX}(\text{dBm}) + G_{TX}(\text{dBi}) + G_{RX}(\text{dBi}) - Sensitivity_{RX}(\text{dBm}) - Losses_{misc}(\text{dB})
% \end{equation}

% όπου:
% \begin{itemize}
% \item $P_{TX}$ η ισχύς εκπομπής,
% \item $G_{TX}, G_{RX}$ τα κέρδη των κεραιών,
% \item $Sensitivity_{RX}$ η ευαισθησία του δέκτη,
% \item $Losses_{misc}$ οι λοιπές απώλειες (καλωδίων, ατμοσφαιρικές).
% \end{itemize}

% Η υψηλή τιμή του (π.χ. 154 \en{dB} για \en{LoRa}) επιτρέπει επικοινωνία σε μεγάλες αποστάσεις, 
% υπερτερώντας έναντι της κλασικής διαμόρφωσης \en{FSK}. Το \en{LoRa} διατηρεί αξιοπιστία ακόμα 
% και σε αρνητικό λόγο σήματος προς θόρυβο (\en{SNR}), λόγω της διάχυσης φάσματος.

% \paragraph{Ευαισθησία Δέκτη (\en{Receiver Sensitivity})}

% Η ελάχιστη ανιχνεύσιμη ισχύς σήματος δίνεται από:
% \begin{equation}
% S_{\text{min}} \approx -174 + 10 \log_{10}(BW) + NF + SNR_{\text{min}}
% \end{equation}
% όπου $-174$ \en{dBm/Hz} είναι το θερμικό όριο, $BW$ το εύρος ζώνης και $NF$ ο συντελεστής 
% θορύβου (~6 \en{dB} για LoRa). Π.χ. για \en{BW}=125 \en{kHz}, ευαισθησία -125 \en{dBm} 
% (\en{SF7}) έως -137 \en{dBm} (\en{SF12}) \cite{SemtechModulationBasics}.

% \subsubsection{Ζώνη \en{Fresnel}}

% Η ζώνη \en{Fresnel} είναι η περιοχή γύρω από την οπτική γραμμή μεταξύ πομπού-δέκτη όπου η 
% ύπαρξη εμποδίων μειώνει την ισχύ λόγω συμβολής. Η ακτίνα της πρώτης ζώνης ($F_1$) είναι:
% \begin{equation}
% F_1 = \sqrt{\frac{\lambda d_1 d_2}{d_1 + d_2}}
% \end{equation}
% όπου $d_1, d_2$ οι αποστάσεις πομπού-εμποδίου-δέκτη. Για $\lambda=0.345 m$ (868 \en{MHz}) 
% και απόσταση 2.5 \en{km} εκατέρωθεν, $F_1 \approx 20.8 m$. Είναι επιθυμητό το 60\% της 
% ζώνης να είναι ελεύθερο \cite{goldsmith2005wireless}.

% \subsubsection{Πολλαπλές Διαδρομές (\en{Multipath Propagation})}

% Το \en{LoRa} είναι ανθεκτικό στο φαινόμενο πολλαπλών διαδρομών (\en{fading}) χάρη στο 
% μεγάλο μήκος κύματος (~34 \en{cm} στα 868 \en{MHz}) και τη διάδοση φάσματος \en{CSS}, 
% επιτρέποντας στον δέκτη συνδυασμό ενέργειας από διαφορετικές διαδρομές \cite{Staniec2018}.

% \subsubsection{Φαινόμενο Doppler (\en{Doppler Effect})}

% Η μετατόπιση Doppler ($\Delta f$) δίνεται από:
% \begin{equation}
% \Delta f = f_c \frac{v}{c}
% \end{equation}
% με $f_c$ συχνότητα, $v$ ταχύτητα και $c$ ταχύτητα φωτός ($3 \times 10^8 \text{m/s}$). 
% Για \en{LoRa} στα 868 \en{MHz} και $v=100 km/h$, η μετατόπιση είναι ~80 \en{Hz}, αμελητέα 
% συγκριτικά με το εύρος ζώνης (125 \en{kHz}).

% \subsubsection{Ατμοσφαιρικά Φαινόμενα}

% Σε υπο-GHz, φαινόμενα όπως υγρασία, βροχή και θερμοκρασιακές μεταβολές έχουν μικρές 
% επιδράσεις, ενισχύοντας την αξιοπιστία του \en{LoRa}.

% Συνοπτικά, το \en{LoRa} αξιοποιεί συχνότητες sub-GHz, διάδοση φάσματος και υψηλή 
% ευαισθησία για μεγάλη εμβέλεια και αξιόπιστη επικοινωνία ακόμη και σε αντίξοες συνθήκες 
% μετάδοσης.


\subsection{Ραδιοφωνική Διάδοση σε αστικό περιβάλλον}

Στο αστικό περιβάλλον, η αξιοπιστία μιας ασύρματης ζεύξης \en{LoRa} επηρεάζεται καθοριστικά από
φαινόμενα ραδιοδιάδοσης όπως η απώλεια διαδρομής \en{(path loss)}, η ζώνη \en{Fresnel}, η
πολυδιαδρομική διάδοση \en{(multipath propagation)} και το φαινόμενο \en{Doppler}. Παρακάτω
παρουσιάζονται αυτά τα φαινόμενα με έμφαση στη φυσική τους ερμηνεία και τη σημασία τους για το
LoRa, καθώς και ένα αριθμητικό παράδειγμα υπολογισμού του ισοζυγίου ζεύξης \en{(link budget)} σε
αστικές συνθήκες.

\subsubsection{Απώλεια Διαδρομής \en{(Path Loss)}}

Η απώλεια διαδρομής εκφράζει την εξασθένηση της ισχύος του σήματος καθώς αυτό διαδίδεται μέσω
του χώρου. Σε ελεύθερο χώρο (χωρίς εμπόδια), η απώλεια διαδρομής αυξάνεται με την απόσταση και
την συχνότητα σύμφωνα με το θεμελιώδες μοντέλο \en{Friis}. Η εξίσωση της ελεύθερης διαδρομής σε
μορφή λογαριθμικής \en{(dB)} δίνεται από: 
$$L_{FS}(dB) = 32.45 + 20\log_{10}(d_{km}) + 20\log_{10}(f_{MHz})$$ 
όπου $d_{km}$ η απόσταση σε χιλιόμετρα και $f_{MHz}$ η συχνότητα σε \en{MHz}. Για
παράδειγμα, σε συχνότητα $868~MHz$ και απόσταση $2~km$ (σενάριο ζεύξης \en{LoRa} στην
Ευρώπη), η απώλεια ελεύθερου χώρου είναι: 
$$L_{FS} = 32.45 + 20\log_{10}(2) + 20\log_{10}(868) \approx 97.4~dB.$$ 
Αυτό σημαίνει ότι το σήμα εξασθενεί κατά περίπου 97 \en{dB} λόγω διάδοσης σε ελεύθερο χώρο. Σε
πραγματικές αστικές συνθήκες όμως, η απώλεια διαδρομής είναι σημαντικά μεγαλύτερη από την
ιδανική περίπτωση ελεύθερου χώρου. Κτίρια, τοίχοι, και γενικά τα εμπόδια σκιάζουν και διαθλούν το
σήμα, με αποτέλεσμα πρόσθετες απώλειες \en{(shadowing, diffraction losses)} \cite{SemtechModulationBasics}. 

Εμπειρικά μοντέλα διάδοσης για πόλεις (όπως το μοντέλο \en{Okumura-Hata}) τυπικά προβλέπουν εκθέτη 
απωλειών μεγαλύτερο από 2 (συχνά 2.7–4 ανάλογα με την πυκνότητα των κτιρίων), γεγονός που συνεπάγεται δεκάδες 
\en{dB} επιπλέον εξασθένησης συγκριτικά με το ελεύθερο πεδίο. Για παράδειγμα, ένα μοντέλο \en{Hata} σε πυκνή αστική
περιοχή μπορεί να δώσει απώλεια διαδρομής περίπου 130–140 \en{dB} στα 2 \en{km}, τιμή αρκετά υψηλότερη από τα
περίπου 97 \en{dB} του ελεύθερου χώρου. Επομένως, το σφάλμα ζεύξης \en{(link margin)} σε αστικά περιβάλλοντα
μειώνεται δραστικά αν δεν υπάρχει καθαρή οπτική επαφή. Είναι ζωτικής σημασίας να λαμβάνεται
υπόψη αυτή η πρόσθετη εξασθένιση κατά τον σχεδιασμό δικτύων \en{LoRa}, ώστε η διαθέσιμη στάθμη
σήματος να παραμένει πάνω από την ευαισθησία του δέκτη για αξιόπιστη επικοινωνία.

\subsubsection{Ζώνη \en{Fresnel} και Διάθλαση}

Η ζώνη \en{Fresnel} περιγράφει μια ελλειψοειδή περιοχή γύρω από την ευθεία της οπτικής επαφής μεταξύ
πομπού-δέκτη, μέσα στην οποία η διάδοση συμβάλλει εποικοδομητικά στην λήψη. Για την πρώτη
ζώνη \en{Fresnel} ($n=1$), η ακτίνα $F_1$ στο ενδιάμεσο της διαδρομής (εκεί όπου η ζώνη είναι μεγαλύτερη)
δίνεται από: 

$$F_1 = \sqrt{\frac{\lambda\, d_1\, d_2}{d_1 + d_2}},$$ 
όπου $\lambda$ το μήκος κύματος του σήματος, και $d_1$, $d_2$ οι αποστάσεις του σημείου από τον
πομπό και το δέκτη αντίστοιχα. Για τη συχνότητα $868~MHz$ ($\lambda\approx0.345~m$)
και συνολική απόσταση $2~\text{km}$, η ακτίνα της 1ης ζώνης \en{Fresnel} στο μέσο της διαδρομής (δηλ.
$d_1=d_2=1~km$) είναι: 
$$F_1 \approx \sqrt{\frac{0.345 \times 1000 \times 1000}{2000}} \approx 13~m.$$ 
Η φυσική σημασία αυτής της ζώνης είναι ότι τουλάχιστον το 60\% της πρώτης ζώνης \en{Fresnel} πρέπει
να είναι ελεύθερο από εμπόδια για να μην προκληθεί σημαντική πρόσθετη εξασθένηση \cite{IJASRE2018}. Αν
αντικείμενα (π.χ. κτίρια) παρεμβάλλονται και εισχωρούν βαθιά στη ζώνη \en{Fresnel}, το σήμα θα υποστεί
διάθλαση \en{(diffraction)} γύρω από τα εμπόδια, επιφέροντας μεγάλες απώλειες πέραν της ελεύθερης
διάδοσης. Σε αστικό περιβάλλον, συνήθως η ζεύξη δεν έχει καθαρή οπτική επαφή – η πρώτη ζώνη
\en{Fresnel} συχνά τέμνεται από κτίρια, δέντρα ή άλλες δομές. Αυτό οδηγεί σε μη γραμμική οπτική ζεύξη
\en{(NLoS)}, όπου η λήψη βασίζεται σε διερχόμενα και περιθλασμένα κύματα. Το αποτέλεσμα είναι
το σήμα να μειωθεί σημαντικά. Για τον σχεδιασμό δικτύου \en{LoRa} στην πόλη, συστήνεται η ανύψωση των
κεραιών (π.χ. εγκατάσταση \en{gateway} σε ψηλά κτίρια) ώστε να μεγιστοποιείται η εκκαθάριση της ζώνης
\en{Fresnel} και να ελαχιστοποιούνται οι απώλειες διάθλασης.

\begin{Illustration}[!ht] 
  \centering
	\includegraphics[width=1\textwidth]{figures/Fresnel_zone.png} 
  \caption{Ζώνη \en{Fresnel}.}
  \label{figure2.3}
  \cite{loradocs} 
\end{Illustration}

\subsubsection{Πολυδιαδρομική Διάδοση \en{(Multipath Propagation)}}

Λόγω των ανακλάσεων σε επιφάνειες όπως κτίρια, το έδαφος και άλλα εμπόδια, ένα ασύρματο σήμα
μπορεί να φτάσει στον δέκτη μέσω πολλαπλών διαδρομών. Αυτή η πολυδιαδρομική διάδοση
προκαλεί διαλείψεις \en{(fading)}, καθώς τα σήματα από διαφορετικές διαδρομές μπορεί να φτάσουν με
διαφορετική καθυστέρηση και φάση. Εάν οι φάσεις τους είναι αντίθετες, ενδέχεται να επέλθει
καταστροφική συμβολή, μειώνοντας σημαντικά την ισχύ του λαμβανόμενου σήματος \en{(deep fade)}. Σε
ένα δυναμικό περιβάλλον, ακόμη και μικρές μετακινήσεις ή αλλαγές μπορούν να μεταβάλουν το
μοτίβο συμβολής, προκαλώντας ταχεία διακύμανση του σήματος \en{(selective fading)}. 

Για τα δίκτυα \en{LoRa}, η πολυδιαδρομή αποτελεί κρίσιμο φαινόμενο σε αστικές περιοχές, ωστόσο η ίδια
η διαμόρφωση \en{LoRa} παρουσιάζει αξιοσημείωτη αντοχή σε \en{multipath} εξασθένιση. Η διαμόρφωση
\en{Chirp Spread Spectrum (CSS)} που χρησιμοποιεί το \en{LoRa} εκπέμπει το σύμβολο ως ένα ευρύ φάσμα
συχνοτήτων \en{(chirp)}, γεγονός που το καθιστά λιγότερο επιρρεπές σε συχνόλεκτες διαλείψεις: το φάσμα
του σήματος είναι σχετικά ευρύ και λειτουργεί αποτελεσματικά σαν ένα είδος διασποράς στο πεδίο
του χρόνου και της συχνότητας. Σύμφωνα με τεκμηρίωση της \en{Semtech}, το φαρδύ \en{chirp}
προσδίδει στο \en{LoRa} «ανοσία στην πολυδιαδρομή και στο \en{fading}, καθιστώντας το ιδανικό για
αστικά και προαστιακά περιβάλλοντα όπου αυτά τα φαινόμενα κυριαρχούν» \cite{SemtechModulationBasics}. Πειραματικές
μελέτες επιβεβαιώνουν αυτήν την ανθεκτικότητα: ο \en{Staniec} και ο \en{Kowal} (2018) ανέφεραν ότι το \en{LoRa}
παρουσιάζει αξιοσημείωτη ανοχή σε έντονες συνθήκες \en{multipath} και παρεμβολών, ιδίως στα
χαμηλότερα \en{bit-rate} (μεγαλύτερους \en{spreading factors}). Συγκεκριμένα, οι μετρήσεις τους σε
θάλαμο πολλαπλών ανακλάσεων έδειξαν πως για ένα εύρος ρυθμίσεων \en{LoRa} υφίστανται περιοχές
ευαισθησίας: μια «λευκή» περιοχή όπου το σήμα είναι πρακτικά ανεπηρέαστο από \en{multipath}, μια
«ανοιχτή γκρίζα» όπου το σύστημα εξακολουθεί να αντέχει στο \en{multipath} αλλά αρχίζει να επηρεάζεται
από ισχυρές παρεμβολές, και μια «σκούρα γκρίζα» περιοχή όπου υπό ακραίες συνθήκες το \en{LoRa}
γίνεται ευάλωτο και στα δύο φαινόμενα \cite{staniec2018}. 

Παρότι το \en{LoRa} είναι εγγενώς ανθεκτικό, η πολυδιαδρομική διάδοση σε αστικά περιβάλλοντα μπορεί
ακόμη να δημιουργήσει προκλήσεις. Αν οι χρονικές καθυστερήσεις ορισμένων διαδρομών
πλησιάσουν τη διάρκεια συμβόλου \en{LoRa}, μπορεί να προκληθεί παρεμβολή συμβόλων (\en{inter-symbol
interference}). Ωστόσο, δεδομένου ότι οι ρυθμοί μετάδοσης \en{LoRa} είναι χαμηλοί (μεγάλες διάρκειες
συμβόλων ειδικά σε υψηλό \en{SF}), οι περισσότερες ανακλώμενες συνιστώσες καταφθάνουν εντός του
παραθύρου ενός συμβόλου και συγχωνεύονται χωρίς να καταστρέφουν την πληροφορία. Έτσι, στην
πράξη το \en{LoRa} σπανίως υφίσταται ολική απώλεια λόγω \en{multipath}, σε αντίθεση με τεχνολογίες
υψηλότερου ρυθμού όπου το \en{multipath} οδηγεί σε έντονο επιλεκτικό \en{fading}. Παρ’ όλα αυτά, για μέγιστη
αξιοπιστία σε πόλεις, συνιστάται η επιλογή παραμέτρων που μεγιστοποιούν το \en{link margin} (π.χ.
υψηλός \en{SF} που προσφέρει μεγαλύτερη ευαισθησία δέκτη) ώστε ακόμη και τυχόν βαθιές διαλείψεις να
μην ρίχνουν το σήμα κάτω από το όριο λήψης.

\subsubsection{Φαινόμενο \en{Doppler}}

Το φαινόμενο \en{Doppler} είναι η μεταβολή της συχνότητας ενός κύματος λόγω σχετικής κίνησης πομπού
ή δέκτη. Στα ασύρματα δίκτυα, εάν ένας κόμβος \en{LoRa} κινείται (π.χ. αισθητήρας σε όχημα) ή αν το
περιβάλλον μεταβάλει αποτελεσματικά τη συχνότητα (π.χ. ανακλώμενη διάδοση από κινούμενα
αντικείμενα), η φέρουσα συχνότητα του σήματος όπως την αντιλαμβάνεται ο δέκτης μετατοπίζεται. Η
μετατόπιση \en{Doppler} $\Delta f$ προσεγγίζεται από τη σχέση: 
$$\Delta f \approx \frac{v}{c} f_c,$$ 
όπου $v$ η σχετική ταχύτητα πομπού-δέκτη, $c$ η ταχύτητα του φωτός και $f_c$ η φερουσα
συχνότητα. Για παράδειγμα, στα $868~MHz$, αν ένας αισθητήρας κινείται με $v=100~km/
h$ ( $\thickapprox 27.8 m/s$), τότε η παρατηρούμενη συχνότητα μετατοπίζεται κατά: 
$$\Delta f \approx \frac{27.8}{3\times10^8} \times 868\times10^6 \approx 80~Hz.$$ 
Μια μετατόπιση περίπου 80 $Hz$ είναι αμελητέα συγκριτικά με το εύρος ζώνης του \en{LoRa} (τυπικά 125 $kHz$), και
συνεπώς δεν υποβαθμίζει την αξιοπιστία του σήματος. Γενικά, το \en{LoRa} είναι εξαιρετικά ανεκτικό στο
\en{Doppler} για τις συνήθεις ταχύτητες που συναντώνται σε αστικά σενάρια (π.χ. οχήματα). Η ίδια η
διαμόρφωση με \en{chirp} καθιστά το σήμα ανθεκτικό σε μικρές συχνιακές αποκλίσεις: μια μόνιμη
μετατόπιση συχνότητας λόγω \en{Doppler} απλώς μεταφράζεται σε μια μικρή χρονική ολίσθηση του
τοπικού χρονοδιαγράμματος αποδιαμόρφωσης, κάτι που ο δέκτης \en{LoRa} μπορεί να αντιμετωπίσει
χωρίς σημαντικές απώλειες απόδοσης. Στην πράξη, αυτό σημαίνει ότι δεν απαιτούνται
κρύσταλλοι υψηλής ακρίβειας για τους ταλαντωτές και ότι το \en{LoRa} λειτουργεί αξιόπιστα ακόμη και
σε κινητικές εφαρμογές, όπως σε αισθητήρες πίεσης ελαστικών, συστήματα διοδίων ή συσκευές σε
μέσα μεταφοράς \cite{SemtechModulationBasics}. 

Φυσικά, σε ακραίες περιπτώσεις πολύ υψηλών ταχυτήτων (πέρα από τα αστικά δεδομένα, π.χ. σε
δορυφορικές ζεύξεις \en{LoRa}) το φαινόμενο \en{Doppler} μπορεί να γίνει υπολογίσιμο, απαιτώντας τεχνικές
διόρθωσης \en{(frequency offset compensation)}. Όμως για επίγειες αστικές επικοινωνίες \en{LoRa}, ακόμη και
ταχύτητες της τάξης των 100–200 $km/h$ μπορούν να εξυπηρετηθούν χωρίς αξιόλογη υποβάθμιση της
ευαισθησίας \cite{SemtechModulationBasics}. Συνοψίζοντας, το φαινόμενο \en{Doppler} δεν αποτελεί κυρίαρχο περιορισμό στην
αξιοπιστία ενός στατικού δικτύου \en{LoRa} ή με αργά κινούμενους κόμβους στην πόλη.

\subsubsection{Παράδειγμα Υπολογισμού Απώλειας Διαδρομής και \en{Link Budget}}

Στην ενότητα αυτή παρουσιάζεται ένα αριθμητικό παράδειγμα που συνδυάζει τον υπολογισμό της
απώλειας διαδρομής και του ισοζυγίου ζεύξης (link budget) για μια χαρακτηριστική σύνδεση LoRa
σε αστικό περιβάλλον. Ας θεωρήσουμε ένα σενάριο όπου: 
Συχνότητα λειτουργίας: $f = 868~\text{MHz}$ (ζώνη EU863-870). 
Απόσταση πομπού-δέκτη: $d = 2~\text{km}$ (αστικό περιβάλλον, πιθανώς χωρίς καθαρή
οπτική επαφή). 
Ισχύς εκπομπής πομπού: $P_{\text{TX}} = 14~\text{dBm}$ (τυπικό μέγιστο \en{LoRa} επιτρεπόμενο
στην ΕΕ). 
Κέρδος κεραίας πομπού/δέκτη: $G_{\text{TX}} = 0~\text{dBi}$ (μικρή μονόπολη στον
αισθητήρα), $G_{\text{RX}} = 2~\text{dBi}$ (κεραία gateway). 
Απώλειες καλωδίων/συνδέσεων: $L_{\text{misc}} = 2~\text{dB}$ (π.χ. απώλεια ομοαξονικού
στον σταθμό βάσης). 
Ευαισθησία δέκτη: $S_{\min} \approx -137~\text{dBm}$ (υψηλή ευαισθησία, π.χ. \en{LoRa} δέκτης
σε SF=12, BW=125 kHz ). 
1. Υπολογισμός απώλειας διαδρομής: Χρησιμοποιούμε πρώτα την εξίσωση ελεύθερου χώρου. Για
$d=2$ km, $f=868$ MHz, όπως υπολογίστηκε προηγουμένως, $L_{\text{FS}} \approx 97.4$ dB. Σε αστικό
περιβάλλον χωρίς οπτική επαφή, θα προσθέσουμε μια επιπλέον απώλεια λόγω σκίασης/διάθλασης. Ας
υποθέσουμε μια συντηρητική επιπλέον εξασθένηση $L_{\text{urban}} = 20~\text{dB}$ (λόγω κτιρίων
που μερικώς φράσσουν τη ζώνη \en{Fresnel} και προκαλούν διάθλαση). Έτσι, η συνολική εκτιμώμενη
απώλεια διαδρομής γίνεται: 
$$L_{\text{path}} \;=\; L_{\text{FS}} + L_{\text{urban}} \;\approx\; 97.4 + 20 \;=\; 117.4~\text{dB}.$$
2. Λήψη και Ισοζύγιο Ζεύξης: Το ισοζύγιο ζεύξης λαμβάνει υπόψη όλα τα κέρδη και τις απώλειες από
τον πομπό έως τον δέκτη. Η ισχύς που φτάνει στον δέκτη ($P_{\text{RX}}$ σε dBm) δίνεται από: 
$$P_{\text{RX}} = P_{\text{TX}} + G_{\text{TX}} + G_{\text{RX}} - L_{\text{path}} - L_{\text{misc}}.$$ 
Αντικαθιστώντας τις αριθμητικές τιμές: 
$$P_{\text{RX}} = 14~\text{dBm} + 0~\text{dB} + 2~\text{dB} - 117.4~\text{dB} - 2~\text{dB}.$$
$$P_{\text{RX}} \approx -103.4~\text{dBm}.$$
Η λαμβανόμενη ισχύς εκτιμάται περίπου $-103.4$ dBm. 
3. Σύγκριση με ευαισθησία δέκτη: Δεδομένου ότι ο δέκτης \en{LoRa} (με SF=12) μπορεί να ανιχνεύσει
σήματα έως και $S_{\min}\approx -137$ dBm, το συγκεκριμένο σενάριο παρουσιάζει ένα περιθώριο
ζεύξης γύρω στα $33.6$ dB (διαφορά μεταξύ $-103.4$ και $-137$ dBm). Αυτό το περιθώριο είναι πολύ
άνετο – υπερκαλύπτει τυχόν επιπλέον απώλειες λόγω πιο έντονου fading ή παρεμβολών,
εξασφαλίζοντας αξιόπιστη επικοινωνία. Σημειώνεται ότι το υπολογισθέν $L_{\text{path}}$ περιείχε
ήδη ένα αποθεματικό 20 dB για αστικό περιβάλλον· στην πράξη, αν η ζεύξη είχε οπτική επαφή (LoS) το
περιθώριο θα ήταν ακόμη μεγαλύτερο. 
4. Επίδραση παραμέτρων LoRa: Αξίζει να τονιστεί ότι η ευαισθησία $-137$ dBm αντιστοιχεί σε
διαμόρφωση \en{LoRa} με τον πιο παρατεταμένο χρόνο συμβόλου (SF12, BW 125   kHz) . Εάν
χρησιμοποιούνταν μια ταχύτερη διαμόρφωση (π.χ. SF7 με ευαισθησία περί τα $-123$   dBm), το
περιθώριο ζεύξης θα μειωνόταν (~20 dB λιγότερο ευαίσθητος δέκτης) αλλά πιθανώς να παρέμενε
επαρκές για 2 km. Στην περίπτωσή μας, με SF12, το πολύ μεγάλο link margin των 33+ dB υποδηλώνει
ότι η ζεύξη θα εξακολουθούσε να λειτουργεί ακόμα κι αν η απώλεια διαδρομής ήταν σημαντικά
μεγαλύτερη (π.χ. μέχρι ~150 dB συνολικά). Πράγματι, τα σύγχρονα \en{LoRa} transceivers υποστηρίζουν
μέγιστο link budget της τάξης των 155–170 dB , ικανό να καλύψει αποστάσεις πολλών δεκάδων
χιλιομέτρων σε ιδανικές συνθήκες. Στο δικό μας σενάριο, μια απώλεια ~117 dB είναι αρκετά χαμηλή
συγκριτικά με το διαθέσιμο link budget (~154–168 dB ανάλογα τον δέκτη ), εξηγώντας γιατί οι
ζεύξεις \en{LoRa} μπορούν να επιτύχουν αξιόπιστη επικοινωνία ακόμα και σε αστικό περιβάλλον με
διάφορες προκλήσεις διάδοσης.
Συμπέρασμα του παραδείγματος: Με τις παραπάνω παραμέτρους, η ζεύξη \en{LoRa} στα 2 km όχι μόνο
"κλείνει" (δηλαδή το σήμα υπερβαίνει το κατώφλι ευαισθησίας του δέκτη), αλλά διαθέτει και
σημαντικό περιθώριο αξιοπιστίας. Αυτό το περιθώριο μπορεί να απορροφήσει επιπλέον απώλειες από
φαινόμενα όπως εξασθένιση multipath, μελλοντική υποβάθμιση σήματος λόγω παρεμβολών, ή μείωση
ισχύος μπαταρίας του πομπού. Δείχνει επίσης τη σπουδαιότητα του link budget: συνδυάζοντας
υψηλή ευαισθησία δέκτη, επαρκή ισχύ εκπομπής και κεραίες με μικρά έστω κέρδη, το \en{LoRa} πετυχαίνει
μεγάλες εμβέλειες. Σε ακραίες αστικές συνθήκες (π.χ. πολύ πυκνό αστικό τοπίο, εσωτερικό κτιρίων), το
περιθώριο αυτό θα μειωθεί, ωστόσο η εγγενής ανθεκτικότητα του πρωτοκόλλου (λόγω του χαμηλού
ρυθμού μετάδοσης και του spread spectrum) επιτρέπει στο \en{LoRa} να διατηρεί επικοινωνία ακόμη και
εκεί όπου άλλα συστήματα υψηλότερης συχνότητας ή ταχύτητας αποτυγχάνουν. 

% -------------------------------------
% Ενότητα 2.4: Το Πρωτόκολλο LoRaWAN
% -------------------------------------


\section{Το Πρωτόκολλο \en{LoRaWAN}}

Το \en{LoRaWAN (Long Range Wide Area Network)}, αποτελεί ένα πρωτόκολλο επικοινωνίας επιπέδου 
\en{MAC (Media Access Control)}, σχεδιασμένο για να επεκτείνει τη φυσική διαστρωμάτωση της 
τεχνολογίας \en{LoRa}, επιτρέποντας την αξιόπιστη και ασφαλή μετάδοση δεδομένων σε δίκτυα 
ευρείας περιοχής με χαμηλή κατανάλωση ενέργειας \cite{semtech_lora_lorawan}. \\


\begin{Illustration}[!ht] 
  \centering
	\includegraphics[width=0.8\textwidth]{figures/LoRa-LoRaWAN_layers.png} 
  \caption{Τεχνολογική στοίβα των \en{LoRa} και \en{LoRaWAN}.}
  \label{figure2.90}
  \cite{semtech_lora_lorawan} 
\end{Illustration} 


Το πρωτόκολλο αναπτύχθηκε και διαχειρίζεται από τον οργανισμό \en{LoRa Alliance}, ο οποίος 
προωθεί τη συμβατότητα και την τυποποίηση μεταξύ κατασκευαστών. Το \en{LoRaWAN} καθορίζει 
τη δικτυακή αρχιτεκτονική, τα επίπεδα ασφαλείας, και τους τρόπους πρόσβασης στο δίκτυο. Η 
αρχιτεκτονική του \en{LoRaWAN} βασίζεται σε μια τοπολογία τύπου αστέρα-από-αστέρες 
\en{(star-of-stars)}, η οποία περιλαμβάνει τα εξής βασικά στοιχεία \cite{ttn_lorawan}:

\begin{itemize}
  \item \textbf{Τερματικές Συσκευές (\en{End Devices})}: Αισθητήρες ή ενεργοποιητές που 
  συλλέγουν δεδομένα και τα αποστέλλουν μέσω του πρωτοκόλλου \en{LoRa}.
  \item \textbf{Πύλες (\en{Gateways})}: Δέχονται τα ασύρματα σήματα από τις τερματικές 
  συσκευές και τα προωθούν στον Διακομιστή Δικτύου \en{(Network Server)} μέσω ενσύρματων 
  ή ασύρματων συνδέσεων, όπως \en{Ethernet}, \en{Wi-Fi} ή κινητά δίκτυα.
  \item \textbf{Διακομιστής Δικτύου (\en{Network Server})}: Διαχειρίζεται τη ροή των δεδομένων, 
  εξαλείφει τα διπλότυπα μηνύματα, εφαρμόζει πολιτικές ασφαλείας και προωθεί τα δεδομένα στους 
  Διακομιστές Εφαρμογών \en{(Application Servers)}.
  \item \textbf{Διακομιστής Εφαρμογών (\en{Application Server})}: Επεξεργάζεται τα δεδομένα 
  σύμφωνα με τις ανάγκες της εκάστοτε εφαρμογής.
\end{itemize}

\begin{Illustration}[!ht] 
  \centering
	\includegraphics[width=1\textwidth]{figures/LoRaWAN_architecture.png} 
  \caption{Τυπική αρχιτεκτονική \en{LoRaWAN} δικτύου.}
  \label{figure2.10}
  \cite{ttn_lorawan}
\end{Illustration} 

\begin{Illustration}[!ht] 
  \centering
	\includegraphics[width=1\textwidth]{figures/LoRaWAN_Network_Server_architecture.png} 
  \caption{Αρχιτεκτονική \en{LoRaWAN Network Server}.}
  \label{figure2.11}
  \cite{tti_homepage}
\end{Illustration} 







\section{Ηλεκτρικοί Υποσταθμοί και Ανάγκες Εποπτείας}
Οι ηλεκτρικοί υποσταθμοί αποτελούν κρίσιμα σημεία του ηλεκτρικού συστήματος μεταφοράς και διανομής ενέργειας. Ο ρόλος τους είναι η μετατροπή της τάσης από υψηλά επίπεδα μεταφοράς σε χαμηλότερα επίπεδα που είναι κατάλληλα για διανομή και τελική κατανάλωση. Οι υποσταθμοί μπορούν να είναι είτε πρωτεύοντες (μεταφοράς), είτε δευτερεύοντες (διανομής).

Η εποπτεία και διαχείριση των υποσταθμών περιλαμβάνει:
\begin{itemize}
  \item παρακολούθηση ηλεκτρικών παραμέτρων όπως ρεύμα, τάση, ισχύς και συχνότητα ανά φάση,
  \item ανίχνευση βλαβών ή ανομαλιών (π.χ. υπερφόρτιση, βυθίσεις τάσης),
  \item έλεγχο λειτουργικών μονάδων όπως διακόπτες ισχύος και προστατευτικά ρελέ,
  \item λήψη αποφάσεων σε πραγματικό χρόνο για την εξασφάλιση της αδιάλειπτης παροχής και της ασφάλειας του εξοπλισμού.
\end{itemize}

Παραδοσιακά, τέτοια εποπτεία γινόταν με ενσύρματες ή \en{SCADA} λύσεις υψηλού κόστους. Η ενσωμάτωση τεχνολογιών όπως το \en{LoRaWAN} επιτρέπει τη δημιουργία αποκεντρωμένων, χαμηλού κόστους και επεκτάσιμων λύσεων, κατάλληλων ακόμη και για μικρούς ή απομακρυσμένους υποσταθμούς.

\section{Συμπεράσματα}
Οι τεχνολογίες \en{LPWAN} και ιδιαίτερα το \en{LoRaWAN} παρέχουν μια αποτελεσματική λύση για τηλεμετρικές εφαρμογές σε περιβάλλοντα όπου απαιτείται χαμηλή κατανάλωση ισχύος και μεγάλη απόσταση μετάδοσης. Το θεωρητικό αυτό υπόβαθρο θεμελιώνει την επιλογή του \en{LoRaWAN} ως βασική τεχνολογία επικοινωνίας στο σύστημα παρακολούθησης και ελέγχου υποσταθμού που αναπτύχθηκε στο πλαίσιο της παρούσας διπλωματικής εργασίας.
