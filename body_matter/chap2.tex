% ========================================================
% Κεφάλαιο 2: Τεχνολογίες LPWAN και το Πρωτόκολλο LoRaWAN
% ========================================================

\chapter{Τεχνολογίες \en{LPWAN} και το Πρωτόκολλο \en{LoRaWAN}}
\label{chap:lpwan}


% -------------------------------
% Ενότητα 2.1: Εισαγωγή στα LPWAN
% -------------------------------


\section{Εισαγωγή στα \en{LPWAN}}

Η διαρκώς αυξανόμενη ανάγκη για απομακρυσμένη και ταυτόχρονα αποδοτική, ως προς την ενέργεια, επικοινωνία 
μεταξύ έξυπνων συσκευών και αισθητήρων έχει οδηγήσει στην εμφάνιση και εξέλιξη μιας νέας γενιάς 
ασύρματων τεχνολογιών, γνωστών ως \en{Low Power Wide Area Networks (LPWAN)}. Οι 
τεχνολογίες \en{LPWAN} επιτρέπουν την αποστολή μικρών σε ποσότητα δεδομένων σε μεγάλες
 αποστάσεις με εξαιρετικά χαμηλή κατανάλωση ενέργειας, καθιστώντας τις ιδανικές για 
 εφαρμογές \en{Internet of Things (IoT)}, όπου η διάρκεια ζωής της μπαταρίας και η 
 αξιοπιστία είναι κρίσιμοι παράγοντες.

Σε αντίθεση με τις τεχνολογίες \en{Wi-Fi} ή \en{Bluetooth}, οι οποίες είναι σχεδιασμένες 
για υψηλούς ρυθμούς μετάδοσης δεδομένων σε μικρές αποστάσεις, τα \en{LPWAN} είναι 
προσανατολισμένα στην υποστήριξη ενός μεγάλου αριθμού συσκευών, με δυνατότητα μετάδοσης 
δεδομένων σε αποστάσεις που υπερβαίνουν τα 10 χιλιόμετρα, σε ανοικτό πεδίο, και σε 
συχνότητες που βρίσκονται στο μη αδειοδοτημένο φάσμα (\en{unlicensed spectrum}). Το 
σημαντικότερο, μάλιστα, όφελος έναντι άλλων τεχνολογιών μετάδωσης πληροφορίας μεγάλου έυρους
(όπως το φάσμα κινητής τηλεφωνίας \en{3G, 4G ή 5G}), είναι η ελάχιστη ενέργεια που απαιτείται 
για την τροφοδοσία των αντίστοιχων συσκευών \cite{ICTexpress}.

Οι πιο διαδεδομένες τεχνολογίες \en{LPWAN} είναι οι εξής:

\begin{itemize}
    \item \textbf{\en{NB-IoT (Narrowband Internet of Things)}} - αποτελεί τεχνολογία ασύρματης 
    επικοινωνίας και χαμηλής ισχύος, βασισμένη στο \en{LTE (Long-Term Evolution)}, η οποία λειτουργεί 
    στο αδειοδοτημένο φάσμα και προσφέρει αξιόπιστη κάλυψη εντός κτιρίων (\en{deep indoor penetration}). 
    Αναπτύχθηκε από το \en{3rd Generation Partnership Project (3GPP)} και υποστηρίζεται από το 
    πρότυπο \en{3GPP Release 13.} Έχει σχεδιαστεί για εφαρμογές με ανάγκες μαζικής συνδεσιμότητας 
    και μικρού όγκου δεδομένων, όπως μετρητές νερού ή αερίου. Η χαμηλή κατανάλωση ενέργειας που 
    απαιτείται για την λειτουργία των συσκευών έχει ως αποτέλεσμα η διάρκειά λειτουργίας τους να 
    φτάνει έως και 10 χρόνια, με την χρήση μίας μόνο μπαταρίας. \cite{telit2019nbiot}
    
    \item \textbf{\en{LTE-M (LTE Cat-M1)}} - επίσης βασίζεται στο \en{LTE} και προσφέρει 
    υψηλότερους ρυθμούς μετάδοσης δεδομένων από το \en{NB-IoT} (έως και $1 Mbps$), 
    διατηρώντας ωστόσο εξίσου χαμηλή κατανάλωση \cite{zipit2023ltem}. Είναι κατάλληλο για φορητές εφαρμογές 
    που απαιτούν αμφίδρομη επικοινωνία σε πραγματικό χρόνο, όπως η παρακολούθηση οχημάτων, 
    οι φορητές ιατρικές συσκευές και οι φορητοί αισθητήρες. Ένα από τα κύρια πλεονεκτήματά 
    του είναι η υποστήριξη κινητικότητας, επιτρέποντας την απρόσκοπτη μετάβαση μεταξύ κυψελών, 
    καθώς και η δυνατότητα φωνητικής επικοινωνίας μέσω \en{VoLTE}. \cite{hosangadi2019system}
    
    \item \textbf{\en{LoRa} και \en{LoRaWAN}} - πρόκειται για το πιο διαδεδομένο πρωτόκολλο 
    σε μη-αδειοδοτη-μένο φάσμα (π.χ. $868 MHz$ στην Ευρώπη), με κύρια πλεονεκτήματα την 
    ευκολία υλοποίησης, τη μεγάλη αυτονομία (έως και 10 έτη), τη χαμηλή κατανάλωση ισχύος 
    και την υψηλή ευελιξία ανάπτυξης μέσω ιδιωτικών ή δημόσιων δικτύων. Η τεχνολογία \en{LoRa} 
    αναπτύχθηκε αρχικά από τη γαλλική \en{Cycleo} και κατοχυρώθηκε από τη \en{Semtech}, 
    ενώ το \en{LoRaWAN} αναπτύσσεται και προτυποποιείται από τη \en{LoRa Alliance}. \cite{semtech_lora_lorawan}
\end{itemize}

Τα δίκτυα \en{LPWAN} ενσωματώνονται όλο και περισσότερο σε κρίσιμες υποδομές, όπως είναι τα συστήματα 
παρακολούθησης ενέργειας, γεωργίας ακριβείας, έξυπνων πόλεων και βιομηχανικής αυτοματοποίησης, 
προσφέροντας λύσεις υψηλής κάλυψης, ανθεκτικότητας και χαμηλού κόστους εγκατάστασης.


% -------------------------------
% Ενότητα 2.2: Σύγκριση Τεχνολογιών LPWAN
% -------------------------------


\section{Σύγκριση Τεχνολογιών \en{LPWAN}}
Παρακάτω γίνεται η σύγκριση των διαφόρων τεχνολογιών \en{LPWAN}:

\begin{Illustration}[!ht] \centering
	\includegraphics[width=0.95\textwidth]{figures/IoT_techs.png} 
    \caption{Σύγκριση τεχνολογιών ασύρματης επικοινωνίας (\en{LPWAN}) ως προς τον ρυθμό μετάδοσης, 
    την κατανάλωση ενέργειας, την εμβέλεια και το κόστος.}
    \label{figure2.1}
    \cite{saft2023iot}
\end{Illustration} 

Οι τεχνολογίες \en{LPWAN} αποτελούν βασικό πυλώνα για την υλοποίηση ενεργειακά αποδοτικών και 
μεγάλης εμβέλειας εφαρμογών \en{IoT}, με διαφορετικές προσεγγίσεις ως προς το φάσμα λειτουργίας, 
την κατανάλωση ισχύος, την κινητικότητα και τη δυνατότητα υποστήριξης ποικίλων τύπων δεδομένων. Ακολούθως 
παρουσιάζονται τα προαναφερθέντα χαρακτηριστικά για τις τρεις πιο διαδεδομένες τεχνολογίες \en{LPWAN:}

\begin{table}[H]
\centering
\renewcommand{\arraystretch}{1.5}
\begin{tabular}{|p{4cm}|p{3.4cm}|p{3.4cm}|p{3.4cm}|}
\hline
\textbf{\textgreek{Παράμετρος}} & \textbf{\en{NB-IoT}} & \textbf{\en{LTE-M}} & \textbf{\en{LoRa}} \\
\hline
\textbf{\textgreek{Τυποποίηση}} & \en{3GPP} & \en{3GPP} & \en{LoRa Alliance} \\
\hline
\textbf{\textgreek{Διαμόρφωση}} & \en{QPSK, 16QAM} & \en{QPSK, 16QAM} & \en{CSS} (\en{Chirp Spread Spectrum}) \\
\hline
\textbf{\textgreek{Φάσμα Συχνοτήτων}} & \en{Licensed} \en{3GPP} (180 \en{kHz}) & \en{Licensed} \en{3GPP} (1.4 \en{MHz}) & \en{Unlicensed ISM} (\en{EU 868 MHz}) \\
\hline
\textbf{\textgreek{Κάλυψη (\en{Link Budget})}} & 151 \en{dB} & 146 \en{dB} & 154 \en{dB} \\
\hline
\textbf{\textgreek{Μέγιστο Φορτίο}} & 1600 \en{bytes} & 1000 \en{bytes} & 242 \en{bytes} \\
\hline
\textbf{\textgreek{Διάρκεια Ζωής Μπαταρίας}} & έως 10 έτη & έως 2 έτη & έως 10 έτη \\
\hline
\textbf{\textgreek{Ταχύτητα Μετάδοσης}} & 200 \en{kbps} & 1 \en{Mbps} & 50 \en{kbps} \\
\hline
\textbf{\textgreek{Αμφίδρομη Επικοινωνία}} & Ναι & Ναι & Ναι \\
\hline
\textbf{\textgreek{Ασφάλεια}} & \en{3GPP (128-256 bit)} & \en{3GPP (128-256 bit)} & \en{AES (128 bit)} \\
\hline
\textbf{\textgreek{Κινητικότητα}} & \en{<100 km/h} & \en{<300 km/h} & Ναι \\
\hline
\textbf{\en{QoS}} & Ναι & Ναι & Όχι \\
\hline
\end{tabular}
\caption{\textgreek{Συγκριτικός πίνακας τεχνολογιών} \en{NB-IoT}, \en{LTE-M} \textgreek{και} \en{LoRa}}
\label{tab:lpwan-comparison}
\cite{ICTexpress}, \cite{adelantado2017understanding}, \cite{zipit2023ltem}, \cite{semtech_lora_lorawan}
\end{table}

\par\smallskip
\noindent\textbf{Σημείωση για τους αριθμούς.} Οι ρυθμοί και οι επιδόσεις είναι ενδεικτικοί: 
επηρεάζονται από εύρος ζώνης, \en{coding rate}, επαναλήψεις, περιβάλλον καναλιού και ρυθμίσεις συστημάτων. 
Για \en{LoRa} (\en{EU}, $BW=125 kHz$) το φυσικό \en{bitrate} είναι της τάξης $\sim$$0.3-5.5 kbps$ 
(ανά $SF$), ενώ με $BW=500 kHz$ μπορεί να φτάσει $\sim$$22 kbps$ \cite{SemtechModulationBasics,LoRaWANSpec}.
Για \en{NB-IoT}/\en{LTE-M} οι τιμές διαφέρουν ανά \en{release}/φορέα/ρύθμιση και δίνονται συνήθως ως εύρος.

\noindent\textbf{Σημείωση για \en{duty-cycle} (Ευρώπη).} Στο \en{EU863-870} το επιτρεπόμενο \en{duty-cycle} δεν είναι 
παντού «1\%»: εξαρτάται από υπο-ζώνη (π.χ.\ 0.1\%, 1\% ή 10\%). Επομένως, ο επιτρεπτός ρυθμός αποστολών 
συνδέεται άμεσα με τον χρόνο στον αέρα (\en{ToA}) του κάθε πακέτου.


Η ανάλυση των επιμέρους χαρακτηριστικών των τριών τεχνολογιών δείχνει πως κάθε μία εξυπηρετεί 
διαφορετικές ανάγκες, ανάλογα με το σενάριο χρήσης και τις απαιτήσεις της εκάστοτε εφαρμογής.

Ξεκινώντας από την κινητικότητα, το \en{LTE-M} υπερέχει με διαφορά, καθώς υποστηρίζει μετακινήσεις 
με ταχύτητες έως και $300 km/h$ και δυνατότητα \en{handover} μεταξύ κυψελών, κάτι που καθιστά 
εφικτή την αξιόπιστη σύνδεση σε περιπτώσεις όπως είναι η παρακολούθηση οχημάτων ή \en{drones} εν κινήσει. 
Από την άλλη μεριά, το \en{NB-IoT} παρέχει περιορισμένη κινητικότητα και είναι περισσότερο κατάλληλο 
για στατικές συσκευές, ενώ το \en{LoRa} μπορεί να χρησιμοποιηθεί για κινητές εφαρμογές μόνο 
αν βρίσκεται εντός εμβέλειας ενός διαθέσιμου \en{gateway}, γεγονός που περιορίζει τη χρήση 
του σε δυναμικά περιβάλλοντα.

Στο πεδίο της μετάδοσης δεδομένων, το \en{LTE-M} προσφέρει τους υψηλότερους ρυθμούς 
($1 Mbps$), καθώς και υποστήριξη φωνητικής επικοινωνίας μέσω \en{VoLTE}, 
χαρακτηριστικά που απουσιάζουν από τις άλλες δύο τεχνολογίες. Αντίθετα, το \en{LoRa} 
περιορίζεται σε πολύ χαμηλούς ρυθμούς (δεν υπερβαίνουν τα $50 kbps$) και είναι σχεδιασμένο 
κυρίως για απλές, σποραδικές μεταδόσεις.

Όσον αφορά το φάσμα λειτουργίας, τόσο το \en{NB-IoT} όσο και το \en{LTE-M} αξιοποιούν το
αδειοδοτημένο φάσμα, γεγονός που προσφέρει πιο σταθερή σύνδεση, μικρότερο λανθάνοντα χρόνο 
και καλύτερη ποιότητα υπηρεσίας (\en{QoS}). Αυτά τα χαρακτηριστικά είναι κρίσιμα για εφαρμογές 
όπως \en{POS terminals}, όπου απαιτείται γρήγορη και αξιόπιστη μετάδοση συναλλαγών. Από την άλλη, 
το \en{LoRa} λειτουργεί σε μη αδειοδοτημένο φάσμα, που αν και μειώνει το κόστος, υπόκειται σε 
περιορισμούς όπως το \en{duty-cycle} και το \en{fair access policy}, μειώνοντας, έτσι, την αξιοπιστία 
σε περιβάλλοντα όπου υπάρχει υψηλή κίνηση δεδομένων.

Σε όρους ενεργειακής απόδοσης, το \en{LoRa} και το \en{NB-IoT} είναι εμφανώς πιο αποτελεσματικά, 
υποστηρίζοντας διάρκεια μπαταρίας έως και 10 έτη. Το \en{LTE-M}, λόγω της μεγαλύτερης κατανάλωσης 
ισχύος, τείνει να έχει μικρότερη διάρκεια ζωής, συνήθως μεταξύ 1-2 ετών, κάτι που πρέπει να 
ληφθεί υπόψη σε εφαρμογές όπου η συντήρηση των κόμβων δεν είναι εύκολη.

Ως προς την εμπορική απήχηση, οι τεχνολογίες του \en{3GPP} (\en{NB-IoT} και \en{LTE-M}) 
προωθούνται κυρίως μέσω παρόχων κινητής τηλεφωνίας και ενσωματώνονται σε λύσεις ευρείας 
κλίμακας από τη βιομηχανία \cite{gsma2022mobileiot}. Αντίθετα, το \en{LoRaWAN}, μέσω της \en{LoRa Alliance}, διατίθεται 
ευρύτερα για αποκεντρωμένες και ιδιωτικές αναπτύξεις, γεγονός που το έχει καταστήσει ιδιαίτερα 
δημοφιλές σε αγροτικές εφαρμογές, αισθητήρες έξυπνων κτιρίων και περιβαλλοντική παρακολούθηση \cite{loraalliance2023report}.

Συνοψίζοντας, δεν υπάρχει μία «καλύτερη» τεχνολογία για κάθε περίπτωση. Η επιλογή εξαρτάται 
από το εκάστοτε έργο και τους στόχους του: αν προέχει η κινητικότητα και η χαμηλή καθυστέρηση, 
το \en{LTE-M} είναι πιο κατάλληλο· αν ζητούμενο είναι η μεγάλη διάρκεια ζωής και το χαμηλό κόστος, 
το \en{LoRa} αποτελεί ιδανική επιλογή· ενώ το \en{NB-IoT} είναι ενδιάμεση λύση για στατικές 
εφαρμογές με αξιόπιστο σήμα και μεγάλη πυκνότητα κόμβων. Η τελική απόφαση λαμβάνει υπόψη 
τεχνικούς περιορισμούς, απαιτήσεις απόδοσης και το οικονομικό κόστος υλοποίησης.


% -------------------------------
% Ενότητα 2.3: Τεχνολογία LoRa
% -------------------------------

\vspace{3em}
\section{Τεχνολογία \en{LoRa}}


%%%%   Υποενότητα 2.3.1: Γενική Επισκόπηση της Τεχνολογίας \en{LoRa}   %%%%


\subsection{Γενική Επισκόπηση της Τεχνολογίας \en{LoRa}}

Ξεκινώντας με μία μικρή ιστορική αναδρομή, η τεχνολογία \en{LoRa (Long Range)} αναπτύχθηκε αρχικά το 2009 από 
δύο φίλους, τους \en{Nicolas Sornin} και \en{Olivier Seller}, όπου στην συνέχεια συμμετείχε στην ομάδα 
και ένας τρίτος συνεργάτης, ο \en{François Sforza} και όλοι μαζί δημιούργησαν τη γαλλική εταιρεία \en{Cycleo} 
το 2010. Δύο χρόνια μετά (2012), η \en{Cycleo} εξαγοράστηκε από την αμερικανική εταιρεία \en{Semtech} 
\cite{semtech2020lora}. Η τεχνολογία αυτή λειτουργεί αποκλειστικά στο φυσικό επίπεδο 
(\en{Physical Layer, PHY}) του μοντέλου αναφοράς \en{OSI (Open Systems Interconnection model)},
και βασίζεται στη διαμόρφωση εξάπλωσης φάσματος τύπου \en{Chirp Spread Spectrum (CSS)}, 
που επιτρέπει την αξιόπιστη και χαμηλής κατανάλωσης μετάδοση δεδομένων σε μεγάλες αποστάσεις. 
Χρησιμοποιεί το ελεύθερο φάσμα ραδιοσυχνοτήτων \en{ISM (Industrial, Scientific and Medical)}, 
με κύρια μπάντα συχνοτήτων στην Ευρώπη τα $868 MHz$ \cite{semtech_lora_lorawan}. 

Η τεχνολογία \en{LoRa} παρέχει σημαντική αντοχή σε παρεμβολές, μιας και χρησιμοποιεί προσαρμοστικό ρυθμό 
μετάδοσης \en{(Adaptive Data Rate - ADR)}, ενώ παράλληλα παρουσιάζει και υψηλή ευαισθησία δεκτών, γεγονότα που 
επιτρέπουν την επικοινωνία ακόμα και σε συνθήκες με μεγάλο θόρυβο από το περιβάλλον. Το εύρος ζώνης που 
χρησιμοποιείται (\en{Bandwidth, BW}) είναι συνήθως $125 kHz, 250 kHz ή 500 kHz$, ανάλογα με 
τις ανάγκες της εφαρμογής. Παράλληλα, η διαμόρφωση χρησιμοποιεί διαφορετικούς παράγοντες εξάπλωσης 
(\en{Spreading Factors, SF}) από 7 έως 12, που επηρεάζουν τον ρυθμό μετάδοσης δεδομένων και την εμβέλεια 
του σήματος \cite{ttn_lorawan}.


\begin{Illustration}[!ht] 
  \centering
	\includegraphics[width=0.7\textwidth]{figures/OSI_LoRa.png} 
  \caption{Μοντέλο \en{OSI} σε αντιστοίχιση με τα \en{LoRa} και \en{LoRaWAN} επίπεδα.}
  \label{figure2.2}
  \cite{semtech_lora_lorawan}
\end{Illustration}


%%%%   Υποενότητα 2.3.2: Ραδιοφωνική Διάδοση σε αστικό περιβάλλον   %%%%


\subsection{Ραδιοφωνική Διάδοση σε αστικό περιβάλλον}

Στο αστικό περιβάλλον, η αξιοπιστία μιας ασύρματης ζεύξης \en{LoRa} επηρεάζεται καθοριστικά από
φαινόμενα ραδιοδιάδοσης όπως η απώλεια διαδρομής \en{(path loss)}, η ζώνη \en{Fresnel}, η
πολυδιαδρομική διάδοση \en{(multipath propagation)} και το φαινόμενο \en{Doppler}. Παρακάτω
παρουσιάζονται αυτά τα φαινόμενα με έμφαση στη φυσική τους ερμηνεία και τη σημασία τους για το
\en{LoRa}, καθώς και ένα αριθμητικό παράδειγμα υπολογισμού του ισοζυγίου ζεύξης \en{(link budget)} σε
αστικές συνθήκες.

\subsubsection{Απώλεια Διαδρομής \en{(Path Loss)}}

Η απώλεια διαδρομής εκφράζει την εξασθένηση της ισχύος του σήματος καθώς αυτό διαδίδεται μέσω
του χώρου. Σε ελεύθερο χώρο (χωρίς εμπόδια), η απώλεια διαδρομής αυξάνεται με την απόσταση και
την συχνότητα σύμφωνα με το θεμελιώδες μοντέλο \en{Friis}. Η εξίσωση της ελεύθερης διαδρομής σε
μορφή λογαριθμικής $(dB)$ δίνεται από: 
\begin{equation}
L_{FS}(dB) = 32.45 + 20\log_{10}(d_{km}) + 20\log_{10}(f_{MHz})
\end{equation}
όπου $d_{km}$ η απόσταση σε χιλιόμετρα και $f_{MHz}$ η συχνότητα σε $MHz$. Για
παράδειγμα, σε συχνότητα $868 MHz$ και απόσταση $2 km$ (σενάριο ζεύξης \en{LoRa} στην
Ευρώπη), η απώλεια ελεύθερου χώρου είναι: 
$$L_{FS} = 32.45 + 20\log_{10}(2) + 20\log_{10}(868) \approx 97.24 dB.$$ 
Αυτό σημαίνει ότι το σήμα εξασθενεί κατά περίπου 97 \en{dB} λόγω διάδοσης σε ελεύθερο χώρο. Σε
πραγματικές αστικές συνθήκες όμως, η απώλεια διαδρομής είναι σημαντικά μεγαλύτερη από την
ιδανική περίπτωση ελεύθερου χώρου. Κτίρια, τοίχοι, και γενικά τα εμπόδια σκιάζουν και διαθλούν το
σήμα, με αποτέλεσμα πρόσθετες να υπάρχουν απώλειες \en{(shadowing, diffraction losses)} \cite{SemtechModulationBasics}. 

Εμπειρικά μοντέλα διάδοσης για πόλεις (όπως το μοντέλο \en{Okumura-Hata}) τυπικά προβλέπουν εκθέτη 
απωλειών μεγαλύτερο από 2 (συχνά 2.7-4 ανάλογα με την πυκνότητα των κτιρίων), γεγονός που συνεπάγεται δεκάδες 
$dB$ επιπλέον εξασθένησης συγκριτικά με το ελεύθερο πεδίο. Για παράδειγμα, ένα μοντέλο \en{Hata} σε πυκνή αστική
περιοχή μπορεί να δώσει απώλεια διαδρομής περίπου $130-140 dB$ στα $2 km$, τιμή αρκετά υψηλότερη από τα
περίπου $97 dB$ του ελεύθερου χώρου. Επομένως, το σφάλμα ζεύξης \en{(link margin)} σε αστικά περιβάλλοντα
μειώνεται δραστικά αν δεν υπάρχει καθαρή οπτική επαφή. Είναι ζωτικής σημασίας να λαμβάνεται
υπόψη αυτή η πρόσθετη εξασθένιση κατά τον σχεδιασμό δικτύων \en{LoRa}, ώστε η διαθέσιμη στάθμη
σήματος να παραμένει πάνω από την ευαισθησία του δέκτη για αξιόπιστη επικοινωνία.

\subsubsection{Ζώνη \en{Fresnel} και Διάθλαση}

Η ζώνη \en{Fresnel} περιγράφει μια ελλειψοειδή περιοχή γύρω από την ευθεία της οπτικής επαφής μεταξύ
πομπού-δέκτη, μέσα στην οποία η διάδοση συμβάλλει εποικοδομητικά στην λήψη. Για την πρώτη
ζώνη \en{Fresnel} ($n=1$), η ακτίνα $F_1$ στο ενδιάμεσο της διαδρομής (εκεί όπου η ζώνη είναι μεγαλύτερη)
δίνεται από: 
\begin{equation}
F_1 = \sqrt{\frac{\lambda\, d_1\, d_2}{d_1 + d_2}},
\end{equation}
όπου $\lambda$ το μήκος κύματος του σήματος, και $d_1$, $d_2$ οι αποστάσεις του σημείου από τον
πομπό και το δέκτη αντίστοιχα. Για τη συχνότητα $868 MHz$ ($\lambda\approx0.345 m$)
και συνολική απόσταση $2 km$, η ακτίνα της 1ης ζώνης \en{Fresnel} στο μέσο της διαδρομής (δηλ.
$d_1=d_2=1 km$) είναι: 
$$F_1 \approx \sqrt{\frac{0.345 \times 1000 \times 1000}{2000}} \approx 13 m.$$ 
Η φυσική σημασία αυτής της ζώνης είναι ότι τουλάχιστον το 60\% της πρώτης ζώνης \en{Fresnel} πρέπει
να είναι ελεύθερο από εμπόδια για να μην προκληθεί σημαντική πρόσθετη εξασθένηση \cite{IJASRE2018}. Αν
αντικείμενα (π.χ. κτίρια) παρεμβάλλονται και εισχωρούν βαθιά στη ζώνη \en{Fresnel}, το σήμα θα υποστεί
διάθλαση \en{(diffraction)} γύρω από τα εμπόδια, επιφέροντας μεγάλες απώλειες πέραν της ελεύθερης
διάδοσης. Σε αστικό περιβάλλον, συνήθως η ζεύξη δεν έχει καθαρή οπτική επαφή - η πρώτη ζώνη
\en{Fresnel} συχνά τέμνεται από κτίρια, δέντρα ή άλλες δομές. Αυτό οδηγεί σε μη γραμμική οπτική ζεύξη
\en{(NLoS)}, όπου η λήψη βασίζεται σε διερχόμενα και περιθλασμένα κύματα. Το αποτέλεσμα είναι
το σήμα να μειωθεί σημαντικά. Για τον σχεδιασμό δικτύου \en{LoRa} στην πόλη, συστήνεται η ανύψωση των
κεραιών (π.χ. εγκατάσταση \en{gateway} σε ψηλά κτίρια) ώστε να μεγιστοποιείται η εκκαθάριση της ζώνης
\en{Fresnel} και να ελαχιστοποιούνται οι απώλειες διάθλασης.

\begin{Illustration}[!ht] 
  \centering
	\includegraphics[width=1\textwidth]{figures/Fresnel_zone.png} 
  \caption{Ζώνη \en{Fresnel} με 40\% της κάλυψη από εμπόδια.}
  \label{figure2.3}
  \cite{loradocs} 
\end{Illustration}

\subsubsection{Πολυδιαδρομική Διάδοση \en{(Multipath Propagation)}}

Λόγω των ανακλάσεων σε επιφάνειες όπως κτίρια, το έδαφος και άλλα εμπόδια, ένα ασύρματο σήμα
μπορεί να φτάσει στον δέκτη μέσω πολλαπλών διαδρομών. Αυτή η πολυδιαδρομική διάδοση
προκαλεί διαλείψεις \en{(fading)}, μιας και τα σήματα από διαφορετικές διαδρομές μπορεί να φτάσουν με
διαφορετική καθυστέρηση και φάση. Συνεπώς, εάν οι φάσεις τους είναι αντίθετες, ενδέχεται να επέλθει
καταστροφική συμβολή, μειώνοντας σημαντικά την ισχύ του λαμβανόμενου σήματος \en{(deep fade)}. Σε
ένα δυναμικό περιβάλλον, ακόμη και μικρές μετακινήσεις ή αλλαγές μπορούν να μεταβάλουν το
μοτίβο συμβολής, προκαλώντας ταχεία διακύμανση του σήματος \en{(selective fading)}. 

Για τα δίκτυα \en{LoRa}, η πολυδιαδρομή αποτελεί κρίσιμο φαινόμενο σε αστικές περιοχές, ωστόσο η ίδια
η διαμόρφωση \en{LoRa} παρουσιάζει αξιοσημείωτη αντοχή σε \en{multipath} εξασθένιση. Η διαμόρφωση
\en{Chirp Spread Spectrum (CSS)} που χρησιμοποιεί το \en{LoRa} εκπέμπει το σύμβολο ως ένα ευρύ φάσμα
συχνοτήτων \en{(chirp)}, γεγονός που το καθιστά λιγότερο επιρρεπές σε συχνόλεκτες διαλείψεις: το φάσμα
του σήματος είναι σχετικά ευρύ και λειτουργεί αποτελεσματικά σαν ένα είδος διασποράς στο πεδίο
του χρόνου και της συχνότητας \cite{staniec2018}. Σύμφωνα με τεκμηρίωση της \en{Semtech}, το φαρδύ \en{chirp}
προσδίδει στο \en{LoRa} «ανοσία στην πολυδιαδρομή και στο \en{fading}, καθιστώντας το ιδανικό για
αστικά και προαστιακά περιβάλλοντα όπου αυτά τα φαινόμενα κυριαρχούν» \cite{SemtechModulationBasics}. Πειραματικές
μελέτες επιβεβαιώνουν αυτήν την ανθεκτικότητα: ο \en{Staniec} και ο \en{Kowal} (2018) ανέφεραν ότι το \en{LoRa}
παρουσιάζει αξιοσημείωτη ανοχή σε έντονες συνθήκες \en{multipath} και παρεμβολών, ιδίως στα
χαμηλότερα \en{bit-rate} (μεγαλύτερους \en{spreading factors}). Συγκεκριμένα, οι μετρήσεις τους σε
θάλαμο πολλαπλών ανακλάσεων έδειξαν πως για ένα εύρος ρυθμίσεων \en{LoRa} υφίστανται περιοχές
ευαισθησίας: μια «λευκή» περιοχή όπου το σήμα είναι πρακτικά ανεπηρέαστο από \en{multipath}, μια
«ανοιχτή γκρίζα» όπου το σύστημα εξακολουθεί να παρουσιάζει ανοχή στο \en{multipath}, αλλά αρχίζει να επηρεάζεται
από ισχυρές παρεμβολές, και μια «σκούρα γκρίζα» περιοχή όπου υπό ακραίες συνθήκες το \en{LoRa}
γίνεται ευάλωτο και στα δύο φαινόμενα \cite{staniec2018}. 

Παρότι το \en{LoRa} είναι εγγενώς ανθεκτικό, η πολυδιαδρομική διάδοση σε αστικά περιβάλλοντα μπορεί
ακόμη να δημιουργήσει προκλήσεις. Αν οι χρονικές καθυστερήσεις ορισμένων διαδρομών
πλησιάσουν τη διάρκεια συμβόλου \en{LoRa}, μπορεί να προκληθεί παρεμβολή συμβόλων (\en{inter-symbol
interference}). Ωστόσο, δεδομένου ότι οι ρυθμοί μετάδοσης \en{LoRa} είναι χαμηλοί (μεγάλες διάρκειες
συμβόλων ειδικά σε υψηλό $SF$), οι περισσότερες ανακλώμενες συνιστώσες καταφθάνουν εντός του
παραθύρου ενός συμβόλου και συγχωνεύονται χωρίς να καταστρέφουν την πληροφορία. Έτσι, στην
πράξη το \en{LoRa} σπανίως υφίσταται ολική απώλεια λόγω \en{multipath}, σε αντίθεση με τεχνολογίες
υψηλότερου ρυθμού όπου το \en{multipath} οδηγεί σε έντονο επιλεκτικό \en{fading}. Παρ’ όλα αυτά, για μέγιστη
αξιοπιστία σε πόλεις, συνιστάται η επιλογή παραμέτρων που μεγιστοποιούν το \en{link margin} (π.χ.
υψηλός $SF$ που προσφέρει μεγαλύτερη ευαισθησία δέκτη) ώστε ακόμη και τυχόν βαθιές διαλείψεις να
μην ρίχνουν το σήμα κάτω από το όριο λήψης.

\subsubsection{Ισοζύγιο Ζεύξης \en{(Link Budget)}}

Το ισοζύγιο ζεύξης (\en{link budget}) αποτελεί έναν από τους πιο κρίσιμους παράγοντες στον 
σχεδιασμό συστημάτων ασύρματης επικοινωνίας, καθώς εκφράζει το συνολικό εύρος απωλειών που 
μπορεί να αντέξει το σήμα κατά τη μετάδοση, χωρίς να χάσει τη δυνατότητα αξιόπιστης λήψης. 
Πρακτικά, το ισοζύγιο ζεύξης υπολογίζεται ως η διαφορά μεταξύ της ισχύος εκπομπής και της 
ελάχιστης ισχύος που απαιτεί ο δέκτης για να λειτουργήσει αποτελεσματικά:
\begin{equation}
Link\ Budget\ (dB) = P_{TX}(dBm) + G_{TX}(dBi) + G_{RX}(dBi) - Sensitivity_{RX}(dBm) - Losses_{misc}(dB)
\end{equation}

όπου:
\begin{itemize}
  \item $P_{TX}$ είναι η ισχύς εξόδου του πομπού (σε $dBm$),
  \item $G_{TX}, G_{RX}$ είναι τα κέρδη των κεραιών εκπομπού και δέκτη αντίστοιχα (σε $dBi$),
  \item $Sensitivity_{RX}$ είναι η ευαισθησία του δέκτη (σε $dBm$),
  \item $Losses_{misc}$ είναι διάφορες επιπλέον απώλειες (π.χ. λόγω καλωδίων, συνδέσεων ή περιβαλλοντικών συνθηκών).
\end{itemize}

Η υψηλή τιμή του ισοζυγίου ζεύξης υποδηλώνει ότι το σύστημα μπορεί να λειτουργήσει αξιόπιστα 
ακόμα και με πολύ ασθενή σήματα ή σε δύσκολες συνθήκες μετάδοσης (π.χ. απομακρυσμένες συσκευές, 
αστικά περιβάλλοντα με πολλά εμπόδια). Η τεχνολογία \en{LoRa} είναι γνωστή για το ιδιαίτερα υψηλό 
ισοζύγιο ζεύξης της, που τυπικά μπορεί να φτάσει έως και $154 dB$, ανάλογα με τις παραμέτρους της 
διαμόρφωσης (κυρίως τον παράγοντα εξάπλωσης $SF$ και το εύρος ζώνης \en{BW}).

Η επίτευξη υψηλού ισοζυγίου ζεύξης στο \en{LoRa} προέρχεται από δύο βασικά χαρακτηριστικά:
\begin{itemize}
\item \textbf{Υψηλή Ευαισθησία Δεκτών}: Οι δέκτες \en{LoRa} είναι σχεδιασμένοι ώστε να μπορούν 
να αποκωδικοποιούν σήματα πολύ χαμηλής έντασης, ακόμη και κάτω από το επίπεδο του θορύβου. 
Χαρακτηριστικά, η ευαισθησία των δεκτών \en{LoRa} κυμαίνεται από $-125 dBm$ (για $SF7, BW=125 kHz$) 
έως $-137 dBm$ (για $SF12, BW=125 kHz$), κάτι που επιτρέπει την λήψη εξαιρετικά ασθενών σημάτων 
από πολύ μεγάλες αποστάσεις \cite{SemtechModulationBasics}.
\item \textbf{Χαμηλός Ρυθμός Μετάδοσης και Διαμόρφωση Ευρέως Φάσματος \en{(Spread Spectrum)}}: 
Η διαμόρφωση \en{LoRa} (\en{Chirp Spread Spectrum, CSS}) διαχέει το σήμα σε μεγάλο χρονικό 
διάστημα και εύρος ζώνης, αυξάνοντας την πιθανότητα αξιόπιστης λήψης του σήματος. Με τον 
τρόπο αυτό επιτυγχάνεται ένα κέρδος επεξεργασίας (\en{processing gain}) που δίνεται περίπου 
από τη σχέση:
\begin{equation}
G_{processing} = 10 \cdot \log_{10}\left(\frac{BW}{R_{data}}\right)
\end{equation}

όπου \(BW\) είναι το εύρος ζώνης μετάδοσης και \(R_{data}\) ο ρυθμός δεδομένων. Αυτό το κέρδος 
επεξεργασίας ενισχύει το σήμα σε σχέση με τον θόρυβο, επιτρέποντας την αποκωδικοποίηση ακόμη 
και υπό πολύ χαμηλές τιμές λόγου σήματος προς θόρυβο (\en{SNR}) \cite{SemtechModulationBasics}.
\end{itemize}

Η τεχνολογία \en{LoRa} συγκρινόμενη με την παραδοσιακή διαμόρφωση συχνότητας (\en{Frequency 
Shift Keying - FSK}), η οποία χρησιμοποιείται ευρέως σε άλλες εφαρμογές ασύρματης επικοινωνίας, 
παρουσιάζει σημαντικά υψηλότερη ευαισθησία. Αυτό οφείλεται στο ότι η \en{FSK} απαιτεί ένα 
ελάχιστο θετικό \en{SNR} για αξιόπιστη αποκωδικοποίηση, ενώ η \en{LoRa} μπορεί να λειτουργήσει 
ακόμα και με αρνητικό \en{SNR}, με το σήμα κυριολεκτικά «θαμμένο» μέσα στο θόρυβο, όπως φαίνεται 
και στο Σχήμα \ref{figure2.4}.

\begin{Illustration}[!ht]
\centering
\includegraphics[width=1\textwidth]{figures/LoRa_vs_FSK.png}
\caption{Σύγκριση ευαισθησίας \en{LoRa} και \en{FSK}, υπογραμμίζοντας την υπεροχή της τεχνολογίας 
\en{LoRa} σε συνθήκες χαμηλού λόγου σήματος προς θόρυβο (\en{SNR}).}
\label{figure2.4}
\cite{SemtechModulationBasics}
\end{Illustration}

Συνοψίζοντας, το υψηλό ισοζύγιο ζεύξης είναι ο κύριος λόγος που η τεχνολογία \en{LoRa} μπορεί 
να επιτύχει επικοινωνία σε πολύ μεγάλες αποστάσεις (έως και δεκάδες χιλιόμετρα σε ανοιχτό πεδίο), 
με πολύ χαμηλή ισχύ εκπομπής, καθιστώντας την ιδανική για εφαρμογές χαμηλής κατανάλωσης σε 
περιβάλλοντα όπου η συντήρηση και η αντικατάσταση μπαταριών είναι δύσκολη ή και αδύνατη.

\subsubsection{Φαινόμενο \en{Doppler}}

Το φαινόμενο \en{Doppler} είναι η μεταβολή της συχνότητας ενός κύματος λόγω σχετικής κίνησης πομπού
ή δέκτη. Στα ασύρματα δίκτυα, εάν ένας κόμβος \en{LoRa} κινείται (π.χ. αισθητήρας σε όχημα) ή αν το
περιβάλλον μεταβάλει αποτελεσματικά τη συχνότητα (π.χ. ανακλώμενη διάδοση από κινούμενα
αντικείμενα), η φέρουσα συχνότητα του σήματος όπως την αντιλαμβάνεται ο δέκτης μετατοπίζεται. Η
μετατόπιση \en{Doppler} $\Delta f$ προσεγγίζεται από τη σχέση: 
\begin{equation}
\Delta f \approx \frac{v}{c} f_c,
\end{equation}
όπου $v$ η σχετική ταχύτητα πομπού-δέκτη, $c$ η ταχύτητα του φωτός και $f_c$ η φερουσα
συχνότητα. Για παράδειγμα, στα $868 MHz$, αν ένας αισθητήρας κινείται με $v=100 km/
h$ ( $\thickapprox 27.8 m/s$), τότε η παρατηρούμενη συχνότητα μετατοπίζεται κατά: 
$$\Delta f \approx \frac{27.8}{3\times10^8} \times 868\times10^6 \approx 80 Hz.$$ 
Μια μετατόπιση περίπου $80 Hz$ είναι αμελητέα συγκριτικά με το εύρος ζώνης του \en{LoRa} (τυπικά $125 kHz$), και
συνεπώς δεν υποβαθμίζει την αξιοπιστία του σήματος. Γενικά, το \en{LoRa} είναι εξαιρετικά ανεκτικό στο φαινόμενο 
\en{Doppler} για τις συνήθεις ταχύτητες που συναντώνται σε αστικά σενάρια (π.χ. οχήματα). Η ίδια η
διαμόρφωση με \en{chirp} καθιστά το σήμα ανθεκτικό σε μικρές συχνές αποκλίσεις: μια μόνιμη
μετατόπιση συχνότητας λόγω \en{Doppler} απλώς μεταφράζεται σε μια μικρή χρονική ολίσθηση του
τοπικού χρονοδιαγράμματος αποδιαμόρφωσης, κάτι που ο δέκτης \en{LoRa} μπορεί να αντιμετωπίσει
χωρίς σημαντικές απώλειες απόδοσης. Στην πράξη, αυτό σημαίνει ότι δεν απαιτούνται
κρύσταλλοι υψηλής ακρίβειας για τους ταλαντωτές και ότι το \en{LoRa} λειτουργεί αξιόπιστα ακόμη και
σε κινητικές εφαρμογές, όπως σε αισθητήρες πίεσης ελαστικών, συστήματα διοδίων ή συσκευές σε
μέσα μεταφοράς \cite{SemtechModulationBasics}. 

Φυσικά, σε ακραίες περιπτώσεις πολύ υψηλών ταχυτήτων (πέρα από τα αστικά δεδομένα, π.χ. σε
δορυφορικές ζεύξεις \en{LoRa}) το φαινόμενο \en{Doppler} μπορεί να γίνει υπολογίσιμο, απαιτώντας τεχνικές
διόρθωσης \en{(frequency offset compensation)}. Όμως για επίγειες αστικές επικοινωνίες \en{LoRa}, ακόμη και
ταχύτητες της τάξης των $100-200 km/h$ μπορούν να εξυπηρετηθούν χωρίς αξιόλογη υποβάθμιση της
ευαισθησίας \cite{SemtechModulationBasics}. Συνοψίζοντας, το φαινόμενο \en{Doppler} δεν αποτελεί κυρίαρχο περιορισμό στην
αξιοπιστία ενός στατικού δικτύου \en{LoRa} ή με αργά κινούμενους κόμβους στην πόλη.

\subsubsection{Παράδειγμα Υπολογισμού Απώλειας Διαδρομής και \en{Link Budget}}

Στην ενότητα αυτή παρουσιάζεται ένα αριθμητικό παράδειγμα που συνδυάζει τον υπολογισμό της
απώλειας διαδρομής και του ισοζυγίου ζεύξης \en{(link budget)} για μια χαρακτηριστική σύνδεση \en{LoRa}
σε αστικό περιβάλλον. Ας θεωρήσουμε ένα σενάριο όπου: 
\begin{itemize}
  \item Συχνότητα λειτουργίας: $f = 868 MHz$ (ζώνη \en{EU863-870}). 
  \item Απόσταση πομπού-δέκτη: $d = 2 km$ (αστικό περιβάλλον, πιθανώς χωρίς καθαρή οπτική επαφή). 
  \item Ισχύς εκπομπής πομπού: $P_{TX} = 14 dBm$ (τυπικό μέγιστο \en{LoRa} επιτρεπόμενο στην ΕΕ). 
  \item Κέρδος κεραίας πομπού/δέκτη: $G_{TX} = 0 dBi$ (μικρή μονοπολική στοναισθητήρα), 
  $G_{RX} = 2 dBi$ (κεραία \en{gateway}). 
  \item Απώλειες καλωδίων/συνδέσεων: $L_{misc} = 2 dB$ (π.χ. απώλεια ομοαξονικού στον σταθμό βάσης). 
  \item Ευαισθησία δέκτη: $S_{\min} \approx -137 dBm$ (υψηλή ευαισθησία, π.χ. \en{LoRa} δέκτης
  σε $SF=12, BW=125 kHz $) \cite{LansitecLoRaRange2024}. 
\end{itemize}

1. \textbf{Υπολογισμός απώλειας διαδρομής}: Χρησιμοποιούμε πρώτα την εξίσωση ελεύθερου χώρου. Για
$d=2 km$, $f=868 MHz$, όπως υπολογίστηκε προηγουμένως, $L_{FS} \approx 97.4 dB$. Σε αστικό
περιβάλλον χωρίς οπτική επαφή, θα προσθέσουμε μια επιπλέον απώλεια λόγω σκίασης/διάθλασης. Ας
υποθέσουμε μια συντηρητική επιπλέον εξασθένηση $L_{urban} = 20~dB$ (λόγω κτιρίων
που μερικώς φράσσουν τη ζώνη \en{Fresnel} και προκαλούν διάθλαση). Έτσι, η συνολική εκτιμώμενη
απώλεια διαδρομής γίνεται: 
$$L_{path} \;=\; L_{FS} + L_{urban} \;\approx\; 97.4 + 20 \;=\; 117.4 dB.$$

2. \textbf{Λήψη και Ισοζύγιο Ζεύξης}: Το ισοζύγιο ζεύξης λαμβάνει υπόψη όλα τα κέρδη και τις απώλειες από
τον πομπό έως τον δέκτη. Η ισχύς που φτάνει στον δέκτη ($P_{RX}$ σε $dBm$) δίνεται από: 
$$P_{RX} = P_{TX} + G_{TX} + G_{RX} - L_{path} - L_{misc}.$$ 
Αντικαθιστώντας τις αριθμητικές τιμές: 
$$P_{RX} = 14 dBm + 0 dB + 2 dB - 117.4~dB - 2 dB.$$
$$P_{RX} \approx -103.4 dBm.$$
Η λαμβανόμενη ισχύς εκτιμάται περίπου $-103.4 dBm$.

3. \textbf{Σύγκριση με ευαισθησία δέκτη}: Δεδομένου ότι ο δέκτης \en{LoRa} (με $SF=12$) μπορεί να ανιχνεύσει
σήματα έως και $S_{\min}\approx -137 dBm$, το συγκεκριμένο σενάριο παρουσιάζει ένα περιθώριο
ζεύξης γύρω στα $33.6 dB$ (διαφορά μεταξύ $-103.4 dBm$ και $-137 dBm$). Αυτό το περιθώριο είναι πολύ
άνετο - υπερκαλύπτει τυχόν επιπλέον απώλειες λόγω πιο έντονου \en{fading} ή παρεμβολών,
εξασφαλίζοντας αξιόπιστη επικοινωνία. Σημειώνεται ότι το υπολογισθέν $L_{path}$ περιείχε
ήδη ένα αποθεματικό $20 dB$ για αστικό περιβάλλον. Στην πράξη, αν η ζεύξη είχε οπτική επαφή \en{(LoS)} το
περιθώριο θα ήταν ακόμη μεγαλύτερο. 

4. \textbf{Επίδραση παραμέτρων \en{LoRa}}: Αξίζει να τονιστεί ότι η ευαισθησία $-137 dBm$ αντιστοιχεί σε
διαμόρφωση \en{LoRa} με τον πιο παρατεταμένο χρόνο συμβόλου ($SF12, BW 125 kHz$) \cite{LansitecLoRaRange2024}. Εάν
χρησιμοποιούνταν μια ταχύτερη διαμόρφωση (π.χ. $SF7$ με ευαισθησία περί τα $-123 dBm$), το
περιθώριο ζεύξης θα μειωνόταν (περίπου $20 dB$ λιγότερο ευαίσθητος δέκτης) αλλά πιθανώς να παρέμενε
επαρκές για $2 km$. Στην περίπτωσή μας, με $SF12$, το πολύ μεγάλο \en{link margin} των $33$+ $dB$ υποδηλώνει
ότι η ζεύξη θα εξακολουθούσε να λειτουργεί ακόμα κι αν η απώλεια διαδρομής ήταν σημαντικά
μεγαλύτερη (π.χ. περίπου μέχρι $150 dB$ συνολικά). Πράγματι, τα σύγχρονα \en{LoRa transceivers} υποστηρίζουν
μέγιστο \en{link budget} της τάξης των $155-170 dB$, ικανό να καλύψει αποστάσεις πολλών δεκάδων
χιλιομέτρων σε ιδανικές συνθήκες. Στο δικό μας σενάριο, μια απώλεια περίπου $117 dB$ είναι αρκετά χαμηλή
συγκριτικά με το διαθέσιμο \en{link budget} (περίπου $154 dB$ με $168 dB$ ανάλογα τον δέκτη), εξηγώντας γιατί οι
ζεύξεις \en{LoRa} μπορούν να επιτύχουν αξιόπιστη επικοινωνία ακόμα και σε αστικό περιβάλλον με
διάφορες προκλήσεις διάδοσης \cite{ttn_lorawan}.

\textbf{Συμπέρασμα του παραδείγματος}: Με τις παραπάνω παραμέτρους, η ζεύξη \en{LoRa} στα $2 km$ όχι μόνο
«κλείνει» (δηλαδή το σήμα υπερβαίνει το κατώφλι ευαισθησίας του δέκτη), αλλά διαθέτει και
σημαντικό περιθώριο αξιοπιστίας. Αυτό το περιθώριο μπορεί να απορροφήσει επιπλέον απώλειες από
φαινόμενα όπως εξασθένιση \en{multipath}, μελλοντική υποβάθμιση σήματος λόγω παρεμβολών, ή μείωση
ισχύος μπαταρίας του πομπού. Δείχνει επίσης τη σπουδαιότητα του \en{link budget}: συνδυάζοντας
υψηλή ευαισθησία δέκτη, επαρκή ισχύ εκπομπής και κεραίες με μικρά έστω κέρδη, το \en{LoRa} πετυχαίνει
μεγάλες εμβέλειες. Σε ακραίες αστικές συνθήκες (π.χ. πολύ πυκνό αστικό τοπίο, εσωτερικό κτιρίων), το
περιθώριο αυτό θα μειωθεί, ωστόσο η εγγενής ανθεκτικότητα του πρωτοκόλλου (λόγω του χαμηλού
ρυθμού μετάδοσης και του \en{spread spectrum}) επιτρέπει στο \en{LoRa} να διατηρεί επικοινωνία ακόμη και
εκεί όπου άλλα συστήματα υψηλότερης συχνότητας ή ταχύτητας αποτυγχάνουν. 


%%%%   Υποενότητα 2.3.3: Chirp Spread Spectrum και η υλοποίησή του στη \en{LoRa}   %%%%


\subsection{\en{Chirp Spread Spectrum} και η υλοποίησή του στη \en{LoRa}}

Η τεχνολογία \en{LoRa} χρησιμοποιεί τη διαμόρφωση φάσματος διασποράς με \en{chirp (Chirp 
Spread Spectrum, CSS)}. Η τεχνική αυτή αναπτύχθηκε αρχικά τη δεκαετία του 1940 για εφαρμογές 
ραντάρ και έκτοτε έχει χρησιμοποιηθεί σε στρατιωτικά και ασφαλή συστήματα επικοινωνιών \cite{RohdeSchwarz2018}. 
Τα τελευταία χρόνια γνωρίζει ευρεία εφαρμογή σε ασύρματες επικοινωνίες δεδομένων, λόγω των 
σχετικά χαμηλών απαιτήσεων ισχύος και της έμφυτης ανθεκτικότητάς της έναντι παραγόντων 
υποβάθμισης του καναλιού, όπως η πολλαπλή διάδοση, το \en{fading}, το φαινόμενο \en{Doppler} κλπ. που 
αναφέρθηκαν προηγουμένως \cite{SemtechModulationBasics}. Μάλιστα, ένα φυσικό στρώμα βασισμένο σε \en{CSS} 
υιοθετήθηκε από το πρότυπο \en{IEEE 802.15.4a} για ασύρματα προσωπικά δίκτυα χαμηλού ρυθμού 
(\en{LR-WPAN}) που απαιτούν μεγαλύτερη εμβέλεια και κινητικότητα, ως εναλλακτική του \en{DSSS} με 
\en{O-QPSK} \cite{RohdeSchwarz2018}.

Σε αυτό το σχήμα διαμόρφωσης, κάθε σύμβολο μεταδίδεται ως ένα ημιτονοειδές σήμα (σήμα \en{chirp}). 
Συνεπώς, αντί για χρήση ψευδοτυχαίων κώδικων διάχυσης φάσματος όπως στο \en{DSSS}, στη \en{LoRa} 
το φάσμα διασπείρεται μέσω σημάτων \en{chirp} - παλμών των οποίων η συχνότητα μεταβάλλεται 
γραμμικά με τον χρόνο. Ένα τέτοιο \en{chirp} σαρώνει ολόκληρο το
διαθέσιμο εύρος ζώνης κατά τη διάρκεια ενός συμβόλου. Ένα σημαντικό πλεονέκτημα αυτής της
μεθόδου είναι ότι τυχόν διαφορές χρονισμού ή/και συχνότητας μεταξύ πομπού και δέκτη
αντισταθμίζονται αυτόματα (καθώς και οι δύο υφίστανται την ίδια μετατόπιση), γεγονός που
απλοποιεί σημαντικά τον σχεδιασμό και τη διαδικασία συγχρονισμού του δέκτη \cite{SemtechModulationBasics}.
Το φάσμα συχνότητας του \en{chirp} καθορίζεται από το εύρος ζώνης $BW$ (\en{Bandwidth}) του σήματος - στην ουσία, το
εύρος ζώνης του \en{chirp} είναι ίσο με το εύρος ζώνης του εκπεμπόμενου σήματος. Η επιθυμητή
πληροφορία (τα δεδομένα) «τεμαχίζεται» σε πολύ υψηλότερο ρυθμό (\en{chips}) και κάθε σύμβολο
μεταδίδεται ως ένα \en{chirp} μέσα στο συγκεκριμένο εύρος ζώνης, ενσωματώνοντας τα δεδομένα στην
αρχική φάση ή συχνότητα του \en{chirp}.

Στην πράξη, ένα \en{up-chirp} (που η συχνότητά του αυξάνεται γραμμικά) χρησιμοποιείται για την εκπομπή 
των συμβόλων, ενώ η \en{LoRa} ορίζει και ειδικά \en{down-chirps} (φθίνουσας συχνότητας) για σήματα 
συγχρονισμού και τερματισμού. Η κυματομορφή ενός ιδανικού \en{up-chirp} μπορεί να περιγραφεί μαθηματικά 
ως σήμα συχνότητας που μεταβάλλεται με σταθερό ρυθμό. Για παράδειγμα, ένα απλοποιημένο μοντέλο \en{up-chirp} 
ξεκινώντας από συχνότητα $f_0$ μπορεί να γραφεί ως:
\begin{equation}
s(t) = \cos\bigl(2\pi f_0 t + \pi \frac{BW}{T_s} t^2\bigr), \quad 0 \le t < T_s,
\end{equation}

όπου $T_s$ η διάρκεια του συμβόλου (\en{chirp}). Στο διάστημα αυτό, η στιγμιαία συχνότητα 
του $s(t)$ αυξάνεται γραμμικά από $f_0$ έως $f_0+BW$. Χάρη στην τεχνική \en{CSS}, οι δέκτες \en{LoRa} μπορούν να 
ανιχνεύσουν σήματα έως και 19.5 \en{dB} κάτω από το επίπεδο θορύβου του καναλιού, αξιοποιώντας διαδικασίες 
συσχέτισης (\en{correlation/demodulation}) που συμπιέζουν το διεσπαρμένο σήμα πίσω στο αρχικό φάσμα του 
\cite{RohdeSchwarz2018}. Συγκριτικά με τα συστήματα \en{DSSS} που χρησιμοποιούν ακολουθίες ψευδοθορύβου 
(όπως π.χ. το 802.11 ή το \en{UMTS}), η \en{LoRa} χρησιμοποιεί \en{chirp pulses} αντί για ψευδοτυχαίους κώδικες για τη 
διασπορά \cite{RohdeSchwarz2018}. Το εκπεμπόμενο σήμα \en{LoRa} έχει συνεχή χαρακτηριστική με σταθερή περιβάλλουσα διαμόρφωση 
(\en{constant envelope FM}), η οποία αυξάνεται ή μειώνεται μονοτονικά εντός του διαθέσιμου φάσματος. Αυτή η 
προσέγγιση επιτυγχάνει την ίδια λειτουργία διάχυσης φάσματος, με χαμηλή πολυπλοκότητα και χωρίς να απαιτείται 
ακριβής γεννήτρια χρονισμού για μακρές ακολουθίες κώδικα, όπως συνέβαινε σε \en{DSSS} εφαρμογές (π.χ. \en{GPS}) 
\cite{SemtechModulationBasics}. Συνολικά, η \en{CSS} διαμόρφωση της \en{LoRa} παρέχει μια ανθεκτική λύση \en{spread spectrum} με 
χαμηλό κόστος και κατανάλωση, κατάλληλη για δίκτυα \en{IoT} μεγάλης εμβέλειας.

\begin{Illustration}[!ht]
\centering
\includegraphics[width=0.7\textwidth]{figures/CSS_up_chrip_down_chirp.png}
\caption{Παράλληλη σύγκριση ενός \en{down-chirp} (αριστερά) και ενός \en{up-chirp} (δεξιά), 
που δείχνει τη γραμμική μείωση/αύξηση της στιγμιαίας συχνότητας (πάνω) συνοδευόμενη 
από το αντίστοιχο σήμα στο πεδίο του χρόνου, όπου η απόσταση των κυμάτων εκτείνεται/συμπιέζεται 
καθ’ όλη τη διάρκεια του συμβόλου.}
\label{figure2.5}
\cite{Baral2020UnderstandingLoRa}
\end{Illustration}

Η μετάδοση ενός πακέτου \en{LoRa} αρχίζει με έναν προκαθορισμένο πρόλογο (\en{preamble}) από
διαδοχικά \en{up-chirps} που επιτρέπουν στον δέκτη να αντιληφθεί την παρουσία σήματος και να
συγχρονίσει τη συχνότητα και το ρολόι του. Τυπικά χρησιμοποιούνται 8 σύμβολα προοιμίου (στην
Ευρώπη), τα οποία ακολουθούνται από 2 ειδικά \en{down-chirps} που σηματοδοτούν το τέλος του
προοιμίου και βοηθούν στον ακριβή συγχρονισμό φάσης του δεκτή \cite{GhoslyaCSS2024}. Μετά το προοίμιο,
έπονται τα \en{chirps} που μεταφέρουν τα ωφέλιμα δεδομένα, καθένα εκ των οποίων έχει
τροποποιηθεί κυκλικά ως προς τη φάση ώστε να αντιστοιχεί σε μια συγκεκριμένη ακολουθία \en{bits}.

\begin{Illustration}[!ht]
\centering
\includegraphics[width=1\textwidth]{figures/LoRa_Symbols_chirps.png}
\caption{Φασματογράφημα σήματος \en{LoRa} που παρουσιάζει 8 αρχικά \en{up-chirps} προοιμίου, 2
\en{down-chirps} συγχρονισμού, και ακολουθία 5 \en{chirp} με κωδικοποιημένα δεδομένα (διαφορετική
κυκλική μετατόπιση σε κάθε σύμβολο). Το σήμα σαρώνει πλήρως ένα εύρος ζώνης 125 $kHz$ με γραμμικά
αυξανόμενη ή μειούμενη συχνότητα σε κάθε \en{chirp}.}
\label{figure2.6}
\cite{GhoslyaCSS2024}
\end{Illustration}


%%%%   Υποενότητα 2.3.4: Παράγοντας Εξάπλωσης και Ευαισθησία Δέκτη   %%%%


\subsection{Παράγοντας Εξάπλωσης και Ευαισθησία Δέκτη}

Όπως έχει ήδη φανεί, βασική παράμετρος στην διαμόρφοωση του σήματος \en{LoRa} είναι ο Παράγοντας Εξάπλωσης (\en{Spreading Factor, SF}).
Ο $SF$ ορίζει τον βαθμό διασποράς του σήματος - ουσιαστικά ισούται με τον αριθμό των \en{bits} που
κωδικοποιούνται σε κάθε σύμβολο. Μια μετάδοση με $SF=n$ σημαίνει ότι κάθε σύμβολο αντιστοιχεί
σε $n$ \en{bits} ωφέλιμης πληροφορίας, τα οποία διασπείρονται σε ένα \en{chirp} διάρκειας $2^n$ \en{chips}.

\begin{Illustration}[!ht]
\centering
\includegraphics[width=0.9\textwidth]{figures/LoRa_waveform_parameters.png}
\caption{Σχηματική απεικόνιση της κυματομορφής \en{Chirp Spread Spectrum} στο \en{LoRa} 
για \en{Spreading Factor} (4 \en{bits}). Το διάγραμμα δείχνει πώς κάθε σύμβολο σαρώνει γραμμικά το 
εύρος ζώνης από $f_{\mathrm{low}}$ σε $f_{\mathrm{high}}$, καθώς και την αντιστοίχιση 
των \en{chips} σε διακριτές τιμές (0-15) μέσα σε ένα σύμβολο.}
\label{figure2.7}
\cite{SiiLoraM2M}
\end{Illustration}

Συνολικά, υπάρχουν διαθέσιμες 6 διακριτές τιμές (τυπικά $SF=7$ έως $SF=12$) για το \en{LoRa
PHY}\footnote{Υπάρχει επίσης $SF=6$ σε ορισμένες υλοποιήσεις, αλλά χρησιμοποιείται μόνο σε ειδικές
περιπτώσεις και με διαφοροποιημένο πρωτόκολλο διαμόρφωσης.}. Μεγαλύτερος $SF$ συνεπάγεται
περισσότερα \en{chips} ανά σύμβολο και άρα μεγαλύτερη διάρκεια συμβόλου, κάτι που μειώνει τον
ρυθμό μετάδοσης δεδομένων αλλά αυξάνει το \en{processing gain} και την ευαισθησία του δέκτη. Αντίθετα,
μικρότερος $SF$ δίνει ταχύτερη μετάδοση (περισσότερα $symbol/s$) αλλά με χαμηλότερη επεξεργαστική
ενίσχυση και συνεπώς μικρότερη ακτίνα αξιόπιστης επικοινωνίας. Στην πράξη, κάθε αύξηση του $SF$
κατά 1 περίπου μονάδα διπλασιάζει τη διάρκεια του συμβόλου (για σταθερό εύρος ζώνης), με αποτέλεσμα
να υποδιπλασιάζεται ο ρυθμός δεδομένων και να απαιτείται υψηλότερο $E_b/N_0$ (\en{$SNR$ per bit}) για σωστή λήψη
(αν δεν αλλάξει η ισχύς) \cite{LoRaWANSpec}. Από την άλλη, ένα μεγαλύτερο $SF$ επιτρέπει στο σήμα να
ταξιδέψει μακρύτερα, αφού μπορεί να ανακτηθεί σωστά ακόμα και με πολύ χαμηλότερο $SNR$ στο
δέκτη λόγω του υψηλού κέρδους διασποράς - π.χ. ένα πακέτο με $SF=12$ μπορεί να φτάσει αποδοτικά
πιο μακριά από ένα με $SF=7$, υπό τις ίδιες λοιπές συνθήκες \cite{LoRaWANSpec}. Η επιλογή του $SF$ σε
δίκτυα \en{LoRaWAN} γίνεται συνήθως δυναμικά, μέσω του μηχανισμού \en{Adaptive Data Rate}, ώστε να
επιτευχθεί ισορροπία μεταξύ ρυθμού μετάδοσης και αξιοπιστίας/απόστασης για κάθε συσκευή.

Ένα ιδιαίτερα σημαντικό χαρακτηριστικό της \en{LoRa} είναι ότι τα σήματα που χρησιμοποιούν
διαφορετικούς $SF$ (σε κοινό κανάλι συχνότητας και εύρους ζώνης) είναι σχεδόν ορθογώνια μεταξύ
τους \cite{ttn_lorawan}. Αυτό σημαίνει ότι μια πύλη \en{(gateway) LoRa} μπορεί να λαμβάνει και να διαχωρίζει
ταυτόχρονα πολλαπλές μεταδόσεις στην ίδια συχνότητα, εφόσον αυτές γίνονται με διαφορετικούς
\en{spreading factors} - χωρίς ουσιαστική αλληλοπαρεμβολή. Η ορθογωνικότητα προκύπτει διότι τα
διασπειρόμενα σήματα με διαφορετικό $SF$ έχουν πολύ χαμηλή συσχέτιση: η εγκάρσια συσχέτιση των
αντίστοιχων σειρών \en{chips} τείνει στο μηδέν, με αποτέλεσμα ένα πακέτο π.χ. $SF=10$ να εμφανίζεται ως
θόρυβος στον δέκτη που «ακούει» $SF=7$ και αντιστρόφως \cite{OlssonFinnsson2017}. Αυτό επιτρέπει την
ταυτόχρονη εξυπηρέτηση πολλών κόμβων στον ίδιο δίαυλο - ουσιαστικά πολλαπλασιάζοντας την
χωρητικότητα του καναλιού, αφού συσκευές με διαφορετικό $SF$ δεν μετρούν ως \en{collision} μεταξύ τους
στο επίπεδο \en{PHY}. Η ιδιότητα αυτή αξιοποιείται από το πρωτόκολλο \en{LoRaWAN} για τον έλεγχο της
συμφόρησης: το δίκτυο μπορεί να αναθέτει υψηλότερους $SF$ σε απομακρυσμένες ή δυσμενείς
συσκευές και χαμηλότερους $SF$ σε συσκευές με καλό σήμα, έτσι ώστε όλες να γίνονται αξιόπιστα
αντιληπτές από τον δέκτη, μοιράζοντας το κανάλι χωρίς ανταγωνισμό \cite{LoRaWANSpec}. 

Σημειώνεται ότι η ορθογωνικότητα μεταξύ διαφορετικών $SF$ ισχύει αυστηρά μόνο όταν
χρησιμοποιείται το ίδιο $BW$. Αν δύο μεταδόσεις χρησιμοποιούν διαφορετικό εύρος ζώνης και
κατάλληλους $SF$ τέτοιους ώστε να έχουν τον ίδιο ρυθμό \en{chips}, τότε οι γραμμικές σαρώσεις συχνότητας
(\en{chirps}) θα έχουν ίδια «κλίση» και τα δύο σήματα δεν θα διαχωρίζονται καλά. Για παράδειγμα, ένας
συνδυασμός $SF=7$ με $BW=125 kHz$ και ένας άλλος με $SF=9$ και $BW=250 kHz$ παράγουν το ίδιο \en{chip
rate} ($R_c$) και ουσιαστικά το ίδιο \en{chirp rate} (ρυθμό μεταβολής συχνότητας), άρα τα σήματα δεν είναι
ορθογώνια μεταξύ τους. Στην πράξη αυτό αποφεύγεται διότι κάθε κανάλι \en{LoRa}
ορίζεται από συγκεκριμένο εύρος ζώνης (π.χ. $125 kHz$) και μόνο ο $SF$ μεταβάλλεται. Όλοι οι κόμβοι σε
ένα συγκεκριμένο κανάλι χρησιμοποιούν το ίδιο $BW$ (συνήθως $125 kHz$ στην Ευρώπη),
διασφαλίζοντας την ορθογωνικότητα μεταξύ των διαθέσιμων $SF7-SF12$. 

\begin{table}[ht]
  \centering
  \begin{tabular}{c c c}
    \toprule
    \en{Spreading Factor} & \en{Chips per symbol} & $SNR_{limit} (dB)$ \\
    \midrule
    7  & 128   & -7.5  \\
    8  & 256   & -10   \\
    9  & 512   & -12.5 \\
    10 & 1024  & -15   \\
    11 & 2048  & -17.5 \\
    12 & 4096  & -20   \\
    \bottomrule
  \end{tabular}
  \caption{Πίνακας παραμέτρων \en{Spreading Factor} και αντίστοιχων ορίων \en{SNR}.}
  \label{tab:sf_snr}
\end{table}

\subsubsection{Ευαισθησία Δέκτη}

Η ευαισθησία  του δέκτη ενός συστήματος \en{LoRa} είναι το ελάχιστο επίπεδο ισχύος στο οποίο 
μπορεί να ανιχνευθεί σωστά το σήμα, δεδομένου ενός συγκεκριμένου $SNR$. Η ευαισθησία 
υπολογίζεται από τη σχέση:
\begin{equation}
S(dBm) = -174 + 10 \log_{10}(BW) + NF + SNR_{limit},
\end{equation}
όπου:
\begin{itemize}
\item $-174 dBm/Hz$: η θερμική πυκνότητα θορύβου στους $25\,^\circ\mathrm{C}$,
\item $BW$: το εύρος ζώνης σε $Hz$,
\item $NF$: ο συντελεστής θορύβου \en{Noise Figure} του δέκτη (συνήθως περίπου $6dB$, προκαθορισμένο αναλόγως με το \en{hardware}),
\item $SNR$: το όριο $SNR$ του συγκεκριμένου $SF$.
\end{itemize}

Ο Πίνακας \ref{tab:sf_sensitivity} παρουσιάζει τα όρια \en{SNR} και την αντίστοιχη ευαισθησία 
του δέκτη για διαφορετικούς παράγοντες εξάπλωσης, με τυπική τιμή $NF=6 dB$ και $BW=125 kHz$.

\begin{table}[ht]
  \centering
  \begin{tabular}{c c c}
    \toprule
    \en{Spreading Factor} & $SNR_{limit} (dB)$ & $S_{(sensitivity)} (dBm)$ \\
    \midrule
    7  & -7.5   & -125 \\
    8  & -10    & -127 \\
    9  & -12.5  & -130 \\
    10 & -15    & -132 \\
    11 & -17.5  & -135 \\
    12 & -20    & -137 \\
    \bottomrule
  \end{tabular}
  \caption{Όρια \en{SNR} και ευαισθησία δέκτη για διαφορετικά \en{Spreading Factors}, με $BW=125kHz$ και $NF=6dB$.}
  \label{tab:sf_sensitivity}
\end{table}

Από τον παραπάνω πίνακα, φαίνεται ότι η αύξηση του $SF$ βελτιώνει σημαντικά την ευαισθησία, 
επιτρέποντας την ανίχνευση σημάτων σε χαμηλότερα επίπεδα ισχύος, κάτι που είναι καθοριστικό 
για τη μεγιστοποίηση της εμβέλειας σε δίκτυα \en{LoRaWAN}.


%%%%   Υποενότητα 2.3.5: Παράγοντας Εξάπλωσης και Ευαισθησία Δέκτη   %%%%


\subsection{Ρυθμός Συμβόλων, \en{Chips, Bitrate} και Εξισώσεις}


\begin{Illustration}[!ht]
\centering
\includegraphics[width=1\textwidth]{figures/LoRa_waveform_values.png}
\caption{Απεικόνιση κυματομορφών \en{Chirp Spread Spectrum} στο \en{LoRa} 
για \en{Spreading Factor} 4, με ένδειξη των διακριτών τιμών (0, 4, 8, 12) 
στην αντίστοιχη χρονική θέση μέσα στο σύμβολο $T_s$. Τα μπλε σημάδια 
δείχνουν το κλάσμα της περιόδου συμβόλου στο οποίο εκπέμπεται κάθε τιμή, 
ενώ η πράσινη σήμανση αντιστοιχεί στον αριθμό του \en{chip}.}
\label{figure2.8}
\cite{SiiLoraM2M}
\end{Illustration}

Το μήκος του συμβόλου στη \en{LoRa} (η διάρκεια ενός \en{chirp}) εξαρτάται άμεσα από τον παράγοντα
εξάπλωσης και το εύρος ζώνης. Συγκεκριμένα, η χρονική διάρκεια $T_s$ ενός συμβόλου ισούται με τον
αριθμό των \en{chips} ανά σύμβολο διά το $BW$. Επειδή κάθε σύμβολο αποτελείται από $2^{SF}$ \en{chips},
προκύπτει:
\begin{equation}
T_{s} = \frac{2^{SF}}{BW},
\end{equation}

όπου $BW$ το χρησιμοποιούμενο εύρος ζώνης (σε $Hz$) \cite{SemtechModulationBasics}. Για παράδειγμα, με $SF=7$
και $BW = 125 kHz$, έχουμε $T_s = 2^7/125000 \approx 1.024 ms$, ενώ με $SF=12$ (ίδιο $BW$) $T_s =
2^{12}/125000 \approx 32.768 ms$ - δηλαδή 32 φορές μεγαλύτερο. Ο ρυθμός συμβόλων $R_s$
(\en{symbols per second}) είναι απλά το αντίστροφο του $T_s$:
\begin{equation}
R_{s} = \frac{1}{T_{s}} = \frac{BW}{2^{SF}},
\end{equation}

Όταν το $SF$ αυξάνεται, ο ρυθμός συμβόλων μειώνεται εκθετικά (κάθε αύξηση $SF$ κατά 1 $\Rightarrow
R_s$ στο μισό). Η έννοια του \en{chip} αντιστοιχεί στο μικρότερο χρονικό βήμα εντός ενός συμβόλου -
πρακτικά είναι το ελάχιστο διάστημα φάσης του \en{chirp} στο οποίο κωδικοποιείται πληροφορία. Με
$2^{SF}$ \en{chips} ανά σύμβολο και $R_s = BW/2^{SF}$, μπορούμε να δούμε ότι ο ρυθμός \en{chips} $R_c$
ισούται με: 
\begin{equation}
R_c = R_s \cdot 2^{SF} = BW,
\end{equation}

Το αποτέλεσμα αυτό σημαίνει ότι ανεξάρτητα από τον $SF$, το \en{modem} παράγει τον ίδιο αριθμό \en{chips} ανά
δευτερόλεπτο, ίσο με το εύρος ζώνης. Για παράδειγμα, σε $BW=125 kHz$, εκπέμπονται $125000 chips/
s$ (δηλ. κάθε \en{chip} έχει διάρκεια $8\;\mu s$), ενώ σε $BW=500 kHz$ εκπέμπονται $500000 chips/s$,
ανεξάρτητα από το $SF$. Αυτή η ιδιότητα συνάδει με τον ορισμό που δίνεται στα φυλλάδια της \en{Semtech}:
«ένα \en{chip} εκπέμπεται ανά $Hz$ ανά δευτερόλεπτο» \cite{SemtechModulationBasics}. 

Εφόσον κάθε σύμβολο αντιστοιχεί σε $SF$ \en{bits} (πριν την προσθήκη κώδικα διόρθωσης), μπορούμε να
εκφράσουμε τον ακαθάριστο ρυθμό \en{bit} της διαμόρφωσης \en{LoRa}. Χωρίς χρήση \en{FEC}, κάθε σύμβολο
φέρει $SF$ \en{bits} πληροφορίας και μεταδίδεται σε χρόνο $T_s$, άρα ο ρυθμός \en{bit} θα ήταν $R_b = SF \cdot
R_s = SF \cdot \frac{BW}{2^{SF}}$. Στην \en{LoRa} όμως εφαρμόζεται επιπρόσθετα ένας κώδικας \en{forward
error correction (FEC)} τύπου \en{Hamming}, που εισάγει πλεονάζοντα \en{bits} για διόρθωση λαθών. Ο βαθμός
κωδικοποίησης εκφράζεται με λόγο $4/(4+CR)$, όπου $CR$ είναι ένας ακέραιος 1-4 (π.χ. $CR=1$
αντιστοιχεί σε \en{code rate} $4/5$, $CR=4$ σε $4/8$ κ.ο.κ.). Ο καθαρός ρυθμός \en{bit} (ονομαστικός, \en{net bit
rate}) λαμβάνοντας υπόψη την πλεονάζουσα κωδικοποίηση, δίνεται από την εξίσωση:
\begin{equation}
R_b = SF \cdot \frac{4}{4+CR} \cdot \frac{BW}{2^{SF}}.
\end{equation}

όπου τα $SF$ και $CR$ ορίζονται όπως παραπάνω \cite{SemtechModulationBasics}. Μπορούμε να ορίσουμε για
ευκολία έναν παράγοντα $RateCode = \frac{4}{4+CR}$ (π.χ. $RateCode = 0.8$ για $CR=1$, ή $0.5$ για
$CR=4$). Τότε η σχέση γίνεται: 
\begin{equation}
R_b = SF \cdot RateCode \cdot \frac{BW}{2^{SF}}.
\end{equation}

Από τις παραπάνω σχέσεις προκύπτει ότι για δεδομένο $BW$ και $CR$, η αύξηση του $SF$ μειώνει τον
$R_b$ εκθετικά. Για παράδειγμα, με $SF=7$, $BW=125 kHz$ και $CR=1$ (κώδικας 4/5), ο καθαρός
ρυθμός είναι:
\begin{equation}
R_b = 7 \cdot \frac{4}{5} \cdot \frac{125000}{128} \approx 5470 bit/s.
\end{equation}

δηλαδή περίπου $5.47 kbps$. Με τις ίδιες παραμέτρους αλλά $SF=12$, ο ρυθμός μειώνεται δραστικά:
\begin{equation}
R_b = 7 \cdot \frac{4}{5} \cdot \frac{125000}{4096} \approx 293 bit/s.
\end{equation}

δηλαδή μόλις $0.293 kbps$. Αν χρησιμοποιηθεί ο μέγιστος πλεονασμός ($CR=4$, δηλ. 4/8), ο ρυθμός
για $SF=12$ πέφτει ακόμα χαμηλότερα, περίπου $183 bit/s$. Αυτός είναι και ο ελάχιστος ρυθμός
δεδομένων σε δίκτυο \en{LoRa} για ένα μόνο κανάλι $125 kHz$. Αντιστρόφως, η χρήση μεγαλύτερου εύρους
ζώνης αυξάνει γραμμικά τον ρυθμό: για $BW=250 kHz$ ο ρυθμός \en{bit} διπλασιάζεται (με σταθερά $SF, CR$),
ενώ για $BW=500 kHz$ τετραπλασιάζεται, κ.ο.κ. \cite{LoRaWANSpec}. 

Σε υψηλές τιμές $SF$, η μεγάλη διάρκεια συμβόλου μπορεί να καταστήσει τις επικοινωνίες πιο ευάλωτες
σε αστάθεια του ταλαντωτή ή σε θόρυβο φάσης. Γι’ αυτό, στα πρότυπα \en{LoRaWAN} συνιστάται η
ενεργοποίηση μιας ρύθμισης \en{Low Data Rate Optimization} (συντομογραφία \en{DE}) για $SF \ge 11$ όταν
$BW=125 kHz$ \cite{LoRaWANSpec}. Αυτή η ρύθμιση πρακτικά μειώνει το \en{effective data rate}
εισάγοντας μικρή καθυστέρηση στη διαμόρφωση (μείωση του ρυθμού \en{symbols/modulation}), ώστε να
βελτιωθεί η αξιοπιστία στον δέκτη. Στις εξισώσεις, η ενεργοποίηση του $DE$ λαμβάνεται υπόψη ως $(SF
- 2)$ στον παρονομαστή ορισμένων όρων (βλ. επόμενη ενότητα), δηλαδή θεωρητικά σαν να μειώνεται
ο διαθέσιμος αριθμός \en{bit} ανά σύμβολο κατά 2 όταν το $DE=1$. 

\subsubsection{Βελτιστοποιήσεις \en{CSS} για \en{LoRa}}

Παρά τους σχετικά χαμηλούς ρυθμούς της καθιερωμένης
διαμόρφωσης \en{LoRa}, πρόσφατες ερευνητικές προσπάθειες στοχεύουν στη βελτίωση της φασματικής
αποδοτικότητάς της. Για παράδειγμα, έχει προταθεί ένα σχήμα διαμόρφωσης \en{Slope-Shift Keying \en{LoRa}}
που προσθέτει, πέρα από το κανονικό \en{up-chirp}, τη χρήση ενός \en{down-chirp} και κυκλικών μετατοπίσεών
του για τη μετάδοση επιπλέον πληροφορίας σε κάθε σύμβολο \cite{Hanif2021Slope}. Με τον τρόπο
αυτό αυξάνεται το αλφάβητο συμβόλων και μπορεί να επιτευχθεί υψηλότερος ρυθμός \en{bit} στην ίδια
ζώνη συχνοτήτων, παραμένοντας συμβατό με τους δέκτες (απαιτεί όμως νέους αλγόριθμους
ανίχνευσης συμβόλων). Μια συναφής προσέγγιση αξιοποιεί τεχνική \en{Index Modulation} στα \en{chirp}
σήματα (ενσωματώνοντας πληροφορία στην επιλογή συγκεκριμένων θέσεων συχνότητας), ώστε να 
βελτιώσει περαιτέρω τον ρυθμό χωρίς επιπλέον ισχύ ή εύρος ζώνης \cite{Hanif2021Index}. Επίσης,
έχει παρουσιαστεί μια επέκταση \en{CSS} όπου χρησιμοποιούνται ορθογώνια \en{chirps} ταυτόχρονα στους
τετραγωνικούς άξονες \en{I} και \en{Q} (δηλ. μετάδοση στο σύμπλοκο επίπεδο με διαμόρφωση \en{IQ})
\cite{Tutorial2022CSS}. Αυτή η μέθοδος, γνωστή ως \en{IQ-CSS}, θεωρητικά διπλασιάζει τον ρυθμό \en{bit} για
το ίδιο $BW$ και $SF$ (καθώς μεταφέρονται διαφορετικά \en{bits} στο \en{I} και στο \en{Q} κανάλι ανά σύμβολο)
\cite{Tutorial2022CSS}. Όλες αυτές οι τεχνικές βρίσκονται υπό μελέτη και υπόσχονται σημαντική
αύξηση της απόδοσης των \en{LoRa} συστημάτων, διατηρώντας παράλληλα τα πλεονεκτήματα μεγάλης
εμβέλειας της \en{CSS} διαμόρφωσης. 


%%%%   Υποενότητα 2.3.6: Υπολογισμός Χρόνου Εκπομπής (Time-on-Air) και Παράγοντες που τον Επηρεάζουν   %%%%


\subsection{Υπολογισμός Χρόνου Εκπομπής \en{(Time-on-Air)} και Παράγοντες που τον Επηρεάζουν}


Ο Χρόνος στον Αέρα ενός πακέτου \en{LoRa (Time-on-Air, ToA)} είναι το χρονικό διάστημα κατά το οποίο το
πακέτο εκπέμπεται και καταλαμβάνει το κανάλι. Περιλαμβάνει τόσο τον χρόνο εκπομπής του
προοιμίου \en{(preamble)} όσο και τον χρόνο εκπομπής του κύριου τμήματος \en{(header + payload)}. Ο \en{ToA}
εξαρτάται από μια σειρά παραμέτρων του φυσικού επιπέδου: τον \en{spreading factor}, το εύρος ζώνης, το
μέγεθος του \en{payload (bytes)}, τον κωδικό διόρθωσης λαθών $CR$, την παρουσία ή όχι ρητής
επικεφαλίδας \en{(explicit header vs implicit)}, την ενεργοποίηση του \en{CRC}, καθώς και τη ρύθμιση \en{low data
rate optimization (DE)} που αναφέρθηκε προηγουμένως \cite{LoRaWANSpec}. Η ακριβής διάρκειά του μπορεί να
υπολογιστεί με βάση τις παραμέτρους αυτές, χρησιμοποιώντας τις σχέσεις που ακολουθούν.

Καταρχάς, υπενθυμίζεται η διάρκεια συμβόλου $T_s = 2^{SF}/BW$. Ένα τυπικό πακέτο \en{LoRaWAN}
χρησιμοποιεί προοίμιο μήκους $n_{preamble}=8$ συμβόλων (που αποτελείται από
$n_{preamble}-1=7$ \en{up-chirps} συν ένα ειδικό τελικό \en{up-chirp} και 2.25 \en{down-chirps} για συγχρονισμό
στο δέκτη). Επιπλέον, το πρωτόκολλο ορίζει ότι ο δέκτης θα αναζητήσει τον προοίμιο με περιθώριο περίπου +4
σύμβολα και μια ίση προκαθορισμένη συσχέτιση $=0.25$ συμβόλου για τον συγχρονισμό
\cite{LoRaWANSpec}. Συνολικά, αυτό προσθέτει $4.25$ σύμβολα επιπλέον του $n_{preamble}$ στον
υπολογισμό του χρόνου προοιμίου. Επομένως, ο χρόνος προοιμίου είναι:
\begin{equation}
T_{preamble} = (n_{preamble} + 4.25) \cdot T_s
\end{equation}

Με $n_{preamble}=8$, έχουμε $T_{preamble} = 12.25 \, T_s$. Για παράδειγμα, αν $T_s =
1.024 ms$ (π.χ. $SF7, 125 kHz$), το προοίμιο διαρκεί περίπου $12.544 ms$.

Στη συνέχεια, πρέπει να υπολογιστεί ο αριθμός συμβόλων του \en{payload} (μαζί με τυχόν \en{header}) που
θα μεταδοθούν, δηλ. πόσα \en{LoRa symbols} απαιτούνται για να χωρέσουν τα δεδομένα και η επικεφαλίδα
με το επιλεγμένο $SF$ και $CR$. Η εξίσωση που δίνει τον αριθμό συμβόλων \en{payload}
($N_{payload}$) είναι αρκετά σύνθετη και περιλαμβάνει έναν τελεστή \en{ceil} (στρογγυλοποίηση
προς τα πάνω), λόγω του ότι αν δεν ταιριάζει ακριβώς ένας ακέραιος αριθμός \en{bits} σε $N$ σύμβολα, θα
απαιτηθεί ένα επιπλέον σύμβολο. Συνοπτικά, η σχέση (όπως προκύπτει από τη \en{Semtech} και τα
πρότυπα \en{LoRaWAN}) είναι η ακόλουθη \cite{SemtechModulationBasics}:
\begin{equation}
N_{payload}
= 8
+ \max\!\Biggl\{\,
   \left\lceil \frac{8\,PL - 4\,SF + 28 + 16\,CRC - 20\,H}{4\,(SF - 2\,DE)} \;\times\;(CR + 4) \right\rceil
   \;,\;
   0
\Biggr\}
\end{equation}

όπου: 
\begin{itemize}
  \item $PL$ είναι το μέγεθος του \en{payload} σε \en{bytes} (ωφέλιμα δεδομένα), 
  \item $CRC=1$ αν συμπεριλαμβάνεται \en{2-byte} $CRC$ στο πακέτο (συνήθως ενεργοποιημένο, αλλιώς $0$),
  \item $H$ είναι \en{bit} ένδειξης \en{header}: $H=0$ για \en{explicit header} (παρουσία τυπικής κεφαλίδας
  \en{LoRaWAN}, μήκους 20 \en{bits}) ή $H=1$ για \en{implicit mode} (χωρίς καθόλου header - ο δέκτης
  γνωρίζει εκ των προτέρων το μέγεθος και $CR$),
  \item $DE=1$ αν έχει ενεργοποιηθεί \en{Low Data Rate Optimization} (σε $SF=11,12, 125 kHz$, αλλιώς $DE=0$)
  \item $CR$ είναι ο ακέραιος (1..4) δείκτης του \en{rate} (όπου ο ρυθμός διόρθωσης είναι $4/(4+CR)$). 
\end{itemize}

Η παραπάνω σχέση μπορεί να αναλυθεί ως εξής: ο αριθμός των \en{bits} του αδιαμόρφωτου \en{payload} είναι
$8\,PL$. Σε αυτό προστίθενται 28 \en{bits} επικεφαλίδων πρωτοκόλλου (υπογραμμίζεται ότι όλα τα πακέτα
\en{LoRaWAN} έχουν μια σταθερή προσθήκη 13 \en{bytes} = 26 \en{bits} υψιπέδων + 2 \en{bits} διαχωρισμού = 28 \en{bits}) και
δυνητικά 16 \en{bits CRC}, ενώ αφαιρούνται 20 \en{bits} αν δεν υπάρχει header ($H=1$ δηλ. \en{implicit mode}). Στον
παρονομαστή, $4(SF - 2DE)$ προκύπτει από το ότι ο τρόπος διαμόρφωσης μοιράζει τα \en{bits} του κώδικα
σε ομάδες των 4 συμβόλων (\en{interleaving} σε τετράδες) και ότι αν το $DE=1$ μειώνει το αποτελεσματικό
$SF$ κατά 2 όπως προαναφέρθηκε. Τέλος, πολλαπλασιάζουμε με $(CR+4)$ διότι για κάθε ομάδα 4
payload \en{bits} προστίθενται $CR$ \en{bits FEC} (σύμφωνα με τον λόγο $4/(4+CR)$). Ολόκληρο το κλάσμα μέσα
στη $\lceil \cdot \rceil$ μας δίνει τον αριθμό τετράδων συμβόλων που απαιτούνται για να χωρέσουν
όλα τα \en{bits}, και η $\lceil \rceil$ στρογγυλοποιεί προς τα πάνω στον επόμενο ακέραιο αν υπάρχει
υπόλοιπο. Το $\max{\;\cdot,\;0}$ εξασφαλίζει ότι λαμβάνουμε τουλάχιστον 0 (το αποτέλεσμα του
κλάσματος μπορεί να βγει αρνητικό για πολύ μικρά πακέτα με συγκεκριμένες παραμέτρους, οπότε
θεωρείται 0) \cite{SemtechModulationBasics}. Στο αποτέλεσμα προστίθεται σταθερά το 8, που αντιπροσωπεύει
έναν ελάχιστο αριθμό 8 συμβόλων πάντα, ακόμη και για πολύ μικρό \en{payload} - πρακτικά το
μικρότερο \en{payload} που μεταδίδεται καταλαμβάνει 8 σύμβολα (πέραν του \en{preamble}).

Αφότου υπολογιστεί το $N_{payload}$, μπορούμε να βρούμε τον χρόνο εκπομπής του \en{payload}:
\begin{equation}
T_{payload} = N_{payload} T_s
\end{equation}

Τέλος, ο συνολικός χρόνος στον αέρα του πακέτου είναι απλώς το άθροισμα: 
\begin{equation}
T_{packet} = T_{preamble} + T_{payload}
\end{equation}

Από τις παραπάνω σχέσεις, είναι φανερό ότι όσο αυξάνεται ο $SF$ ή/και ο $CR$, τόσο περισσότερα
σύμβολα απαιτούνται (μεγαλύτερο $N_{payload}$) και τόσο μεγαλύτερο είναι το $T_s$,
οδηγώντας σε μεγαλύτερο χρόνο εκπομπής. Αντίθετα, ένα μεγαλύτερο $BW$ μειώνει ανάλογα το
$T_s$ (π.χ. διπλασιασμός του $BW$ μισός χρόνος συμβόλου) και έτσι μειώνει τον \en{ToA}. Για να δώσουμε
μια αίσθηση, στον πίνακα 2.4 φαίνεται ο \en{ToA} ($ms$) για ένα σύντομο πακέτο τυπικών
διαστάσεων, υπό διάφορους $SF$, με σταθερά $BW=125 kHz$ και $CR=1 (4/5)$, σύμφωνα με τους
υπολογισμούς των παραπάνω εξισώσεων.

\begin{table}[ht]
  \centering
  \begin{tabular}{c c l}
    \toprule
    \en{Spreading Factor} & \en{Symbol Duration (ms)} & \en{Low Data Rate Opt.} \\
    \midrule
    7  &  41   &  \\
    8  &  72   &  \\
    9  &  144  &  \\
    10 &  289  &  \\
    11 &  578  & (\en{DE} = 1) \\
    12 &  991  & (\en{DE} = 1) \\
    \bottomrule
  \end{tabular}
  \caption{Διάρκεια συμβόλου \en{LoRa} σε \en{BW}=125 \en{kHz} για διάφορα \en{Spreading Factors}, με ενεργοποιημένη \en{Low Data Rate Optimization (DE)} όπου απαιτείται.}
  \label{tab:sf_symbol_duration}
\end{table}

Οι τιμές αυτές επιβεβαιώνουν ότι περίπου κάθε μονάδα αύξησης του $SF$ διπλασιάζει τον απαιτούμενο
χρόνο εκπομπής (για σταθερό εύρος ζώνης) \cite{RFWirelessLoRaAirtimeCalc}. Αυτό έχει σημαντικές επιπτώσεις
στην ενεργειακή κατανάλωση των κόμβων: ένας κόμβος που εκπέμπει με υψηλό $SF$ θα διατηρεί τον
πομπό του ανοικτό πολύ περισσότερη ώρα σε σχέση με έναν που εκπέμπει το ίδιο δεδομένο με
χαμηλότερο $SF$, ξοδεύοντας περισσότερη ενέργεια από την μπαταρία του. Για τον λόγο αυτό, η επιλογή
του χαμηλότερου δυνατού $SF$ που ικανοποιεί τις απαιτήσεις επικοινωνίας είναι κρίσιμη για την
παράταση της διάρκειας ζωής των συσκευών. Επιπλέον, σε κανονιστικά πλαίσια όπως η Ευρώπη, 
ισχύουν περιορισμοί \en{duty-cycle} στο \en{EU863-870} που 
διαφέρουν ανά υπο-ζώνη (π.χ.\ 0.1\%, 1\% ή 10\%). Όσο μεγαλώνει ο \en{ToA}, τόσο περιορίζεται ο πρακτικός 
ρυθμός αποστολών. Ενδεικτικά, αν ένα πακέτο διαρκεί 1 δευτερόλεπτο στον αέρα και το όριο της υπο-ζώνης είναι 1\%, 
η συσκευή δεν μπορεί να το εκπέμπει πάνω από 36 φορές (περίπου) ανά ώρα. Συνεπώς, η μείωση \en{ToA} βοηθά 
ταυτόχρονα την ενεργειακή απόδοση και τη συμμόρφωση με τους περιορισμούς εκπομπής.



%%%%   Υποενότητα 2.3.7: Αποδιαμόρφωση και αποκωδικοποίηση σήματος LoRa   %%%%


\subsection{Αποδιαμόρφωση και αποκωδικοποίηση σήματος \en{LoRa}}

Η λήψη και αποδιαμόρφωση (\en{demodulation}) του σήματος \en{LoRa} πραγματοποιείται με
τεχνικές που εκμεταλλεύονται τη δομή των \en{chirp}. Ο δέκτης, μόλις ανιχνεύσει το προοίμιο,
παράγει ένα τοπικό σήμα αναφοράς (π.χ. έναν \en{down-chirp} που καλύπτει την ίδια ζώνη
συχνοτήτων) και πολλαπλασιάζει το ληφθέν σήμα με το συζυγές του σήματος αναφοράς, μία διαδικασία
γνωστή ως «απο-τσιρπικοποίηση» (\en{de-chirping}). Με αυτόν τον τρόπο, ένας λαμβανόμενος \en{up-chirp} 
μετατρέπεται σε ένα σχεδόν σταθερής συχνότητας ημιτονικό σήμα στο πεδίο του χρόνου, του οποίου
η συχνότητα εξαρτάται από τη σχετική μετατόπιση (ολίσθηση) που είχε το \en{chirp} του συμβόλου.
Εφαρμόζοντας έναν γρήγορο μετασχηματισμό \en{Fourier (FFT)} στο απο-διαμορφωμένο σήμα, ο
δέκτης μπορεί να αποτυπώσει ένα διάγραμμα ισχύος ως προς τη συχνότητα, στο οποίο εμφανίζεται
μια χαρακτηριστική κορυφή. Η θέση της κορυφής αυτής στο φάσμα αντιστοιχεί στη δυαδική τιμή του
συμβόλου που μεταδόθηκε. Με άλλα λόγια, η αποδιαμόρφωση στο \en{LoRa} υλοποιείται με
έναν αποδιασπορέα συχνότητας και έναν μετασχηματισμό \en{FFT}, που επιτρέπουν την ανίχνευση
συσχέτισης του σήματος με κάθε πιθανή συμβολοσειρά \cite{SemtechModulationBasics,Tutorial2022CSS,RohdeSchwarz2018}. Η μέθοδος αυτή είναι 
αποδοτική υπολογιστικά και μπορεί να υλοποιηθεί με χαμηλής κατανάλωσης ηλεκτρονικά στον δέκτη,
σε αντίθεση με πιο περίπλοκα σχήματα διάχυσης (π.χ. \en{DSSS}).

Αφότου εξαχθούν οι διαδοχικές τιμές συμβόλων από το φυσικό επίπεδο, ακολουθεί η διαδικασία
αποκωδικοποίησης των \en{bits}. Σε φυσικό επίπεδο, το \en{LoRa} εφαρμόζει διορθωτικό κώδικα
προς διόρθωση σφαλμάτων \en{(Forward Error Correction)}. Συγκεκριμένα, χρησιμοποιείται ένας κώδικας
διεύρυνσης \en{(coding rate)} όπου για κάθε ομάδα 4 δεδομμένων \en{bits} προστίθενται $CR$ επιπλέον \en{bits}
ανίχνευσης/διόρθωσης σφαλμάτων. Αυτό αντιστοιχεί σε λόγο κώδικα 4/(4+$CR$) - π.χ. $CR=1$ δίνει
4/5 (20\% πλεονάζοντα \en{bits}), ενώ $CR=4$ δίνει 4/8=1/2 (50\% πλεονάζοντα). Ο δέκτης, γνωρίζοντας το
$CR$ από την επικεφαλίδα (ή προκαθορισμένα), εφαρμόζει τον αντίστροφο αλγόριθμο του κώδικα για
να ανιχνεύσει και διορθώσει τυχόν σφάλματα στα \en{bits} των συμβόλων. Έπειτα, ελέγχει την ακεραιότητα
της συνολικής ωφέλιμης ακολουθίας μέσω του \en{CRC} (εάν υπάρχει). Εφόσον το \en{CRC}
επαληθευτεί, τα αρχικά δεδομένα περνούν στο επάνω επίπεδο (π.χ. εφαρμογή). Σημειώνεται ότι η
ανοχή του \en{LoRa} σε ταυτόχρονες μεταδόσεις περιορίζεται όταν αυτές χρησιμοποιούν τον ίδιο \en{SF}
και κανάλι. Σε τέτοιες περιπτώσεις \en{(collision)}, ο δέκτης συνήθως θα συγχρονιστεί και θα
αποδιαμορφώσει μόνο το ισχυρότερο από τα πακέτα (φαινόμενο \en{capture effect}), εκτός αν τα σήματα
είναι επαρκώς χρονικά και φασματικά διαχωρισμένα. Παρ’ όλα αυτά, η χρήση διαφορετικών
\en{Spreading Factors} ή καναλιών συχνοτήτων από τις συσκευές αποτρέπει τις περισσότερες
συγκρούσεις, αξιοποιώντας πλήρως την ορθογωνιότητα και ευελιξία του πρωτοκόλλου.

% Ορίζεται ο παράγοντας εξάπλωσης (\en{Spreading Factor}, 
% \textbf{\en{SF}}), ο οποίος καθορίζει τον αριθμό των «\en{chips}» (μικρών χρονικών βημάτων σήματος) 
% που συνθέτουν κάθε σύμβολο. Συγκεκριμένα, ο αριθμός των \en{chips} ανά σύμβολο είναι $N = 2^{SF}$. 
% Αυτό σημαίνει ότι κάθε σύμβολο αποτελείται από $2^{SF}$ διακριτές μεταβολές συχνότητας 
% (\en{chips}). Έτσι, ένα σύμβολο μπορεί να αντιστοιχεί σε $2^{SF}$ διαφορετικές τιμές (από 0 έως 
% $2^{SF}-1$), κωδικοποιώντας συνολικά $SF$ \en{bits} πληροφορίας (αφού $2^{SF}$ διαφορετικοί 
% συνδυασμοί αντιστοιχούν σε $SF$ \en{bits}). Η διάρκεια κάθε συμβόλου (\en{symbol duration}) εξαρτάται 
% από τις παραμέτρους $SF$ και $BW$. Δεδομένου ότι το \en{LoRa} μεταδίδει τα \en{chips} με ρυθμό 
% ίσο με το εύρος ζώνης ($R_c = BW$, σε \en{chips} ανά δευτερόλεπτο), ένα σύμβολο που περιέχει 
% $2^{SF}$ \en{chips} θα έχει διάρκεια: 
% \begin{equation}
% T_{sym} = \frac{2^{SF}}{BW},
% \end{equation} όπου $T_{sym}$ μετράται σε δευτερόλεπτα. Συνεπώς, η διάρκεια συμβόλου 
% αυξάνεται εκθετικά με το $SF$ (αφού κάθε αύξηση του $SF$ κατά 1 διπλασιάζει τον αριθμό 
% \en{chips} και τον χρόνο μετάδοσης), ενώ μειώνεται αντιστρόφως ανάλογα με το $BW$. Για 
% παράδειγμα, με $SF=7$ και $BW=125kHz$ προκύπτει 
% $T_{sym} \approx 2^7/125000 = 0.001024s \approx 1.0~ms$, ενώ με $SF=12$ (ίδιο $BW$) 
% προκύπτει $T_{sym} \approx 2^{12}/125000 \approx 0.032768~s \approx 32.8~ms$, 
% δηλαδή πολύ μεγαλύτερη διάρκεια συμβόλου. Το αντίστροφο μέγεθος είναι ο 
% \textit{ρυθμός συμβόλων} $R_{sym}$ (σύμβολα ανά δευτερόλεπτο), ο οποίος δίνεται από: 
% \begin{equation}
% R_{sym} = \frac{1}{T_{sym}} = \frac{BW}{2^{SF}},
% \end{equation} δηλαδή $R_{sym}$ μειώνεται όσο αυξάνεται το $SF$, και αυξάνεται όσο 
% αυξάνεται το $BW$ (αντιστρόφως ανάλογη σχέση του $R_{sym}$ με $2^{SF}$ και ανάλογη 
% με το $BW$). Επειδή κάθε σύμβολο μπορεί να μεταφέρει $SF$ \en{bits} πληροφορίας (στην 
% απλούστερη περίπτωση χωρίς διορθωτική κωδικοποίηση), μπορούμε να ορίσουμε τον ωφέλιμο 
% \textit{ρυθμό μετάδοσης \en{bit}} $R_b$ (\en{bit} ανά δευτερόλεπτο) ως το γινόμενο του ρυθμού 
% συμβόλων επί τα \en{bits} ανά σύμβολο: \begin{equation}
% R_b = R_{sym} \cdot SF = SF \cdot \frac{BW}{2^{SF}},
% \end{equation} εκφρασμένο σε \en{bits/second}. Όπως αναμένεται, μεγαλύτερο $SF$ (ή μικρότερο $BW$) 
% οδηγεί σε χαμηλότερο $R_b$, ενώ μικρότερο $SF$ (ή μεγαλύτερο $BW$) αυξάνει το $R_b$. Στην 
% πράξη, η διαμόρφωση \en{LoRa} μπορεί να χρησιμοποιεί πρόσθετη διορθωτική κωδικοποίηση 
% (\en{Forward Error Correction, FEC}) με λόγο κωδικοποίησης (π.χ. 4/5, 4/6, 4/7, 4/8), γεγονός 
% που μειώνει τον καθαρό ρυθμό δεδομένων. Για παράδειγμα, με λόγο 4/5, μόνο το 80\% των 
% μεταδιδόμενων \en{bit} αντιστοιχεί σε ωφέλιμη πληροφορία, επομένως ο καθαρός ρυθμός \en{bit} γίνεται 
% $0.8, R_b$. Γενικά, αν ο λόγος κωδικοποίησης εκφράζεται ως $r_c = 4/(4+\delta)$ (όπου 
% $\delta=1,2,3,4$ για 4/5 έως 4/8 αντίστοιχα), ο καθαρός ρυθμός δεδομένων δίνεται από 
% $R_b^{(net)} = r_c \cdot R_b$ - δηλαδή μεγαλύτερη τιμή του $\delta$ (περισσότερα 
% πλεονάζοντα \en{bits FEC}) μειώνει ανάλογα το καθαρό \en{bitrate}, προσφέροντας όμως αυξημένη προστασία 
% στα σφάλματα. 

% Ο ρυθμός \en{chips} $R_c$ είναι μια σημαντική έννοια στη διαμόρφωση \en{CSS}. Όπως αναφέρθηκε, 
% αριθμητικά $R_c = BW$ (\en{chips} ανά δευτερόλεπτο), που σημαίνει ότι το σήμα εκπέμπει $R_c$ \en{chips} 
% κάθε δευτερόλεπτο. Η διάρκεια ενός \en{chip} ισούται με $T_{chip} = 1/R_c = 1/BW$ δευτερόλεπτα. 
% Για παράδειγμα, με $BW = 125kHz$, κάθε \en{chip} έχει διάρκεια $8\mu s$  περίπου. Κάθε \en{chip} αντιστοιχεί 
% σε μία μικρή μεταβολή στη συχνότητα του \en{chirp}. Μπορούμε να θεωρήσουμε ότι το διαθέσιμο φάσμα 
% χωρίζεται σε $2^{SF}$ διακριτά βήματα συχνότητας, καθένα πλάτους $\Delta f = BW/2^{SF}$. Σε 
% κάθε διαδοχικό \en{chip}, η συχνότητα του σήματος αυξάνεται κατά $\Delta f$, με αποτέλεσμα το σήμα 
% να σαρώνει ολόκληρη τη ζώνη συχνοτήτων στη διάρκεια ενός συμβόλου. Η αντιστοιχία $R_c = BW$ 
% και ο ορισμός του $SF$ εξασφαλίζουν ότι ένα σύμβολο διάρκειας $T_{sym}$ καλύπτει πράγματι όλο 
% το εύρος ζώνης, αποτελούμενο από $2^{SF}$ \en{chips}, σύμφωνα με τη σχέση $(BW \cdot T_{sym} = 2^{SF})$ 
% που προέκυψε παραπάνω. Ένα βασικό πλεονέκτημα της διαμόρφωσης \en{CSS} είναι το \textit{κέρδος λόγω 
% διασποράς} (\en{spreading gain}), το οποίο βελτιώνει την ευαισθησία και την αξιοπιστία της ζεύξης. 
% Ουσιαστικά, ο λόγος του ρυθμού \en{chips} προς τον ρυθμό συμβόλων ισούται με $R_c / R_{sym} = 2^{SF}$, 
% δηλαδή ισούται με τον αριθμό των \en{chips} ανά σύμβολο. Αυτός ο λόγος αντιπροσωπεύει τον παράγοντα 
% με τον οποίο ενισχύεται το σήμα έναντι του θορύβου μέσω της διασποράς. Σε λογαριθμική κλίμακα, 
% το κέρδος διασποράς μπορεί να προσεγγιστεί από τη σχέση: 
% \begin{equation}
% G_{sp} \approx 10 \log_{10}(2^{SF})  (dB),
% \end{equation} που εκφράζεται σε \en{decibel}. Για παράδειγμα, αύξηση του $SF$ κατά 1 (διπλασιασμός 
% των \en{chips} ανά σύμβολο) προσθέτει περίπου $10\log_{10}2 \approx 3dB$ επιπλέον επεξεργαστικού 
% κέρδους. Ένα σύστημα με $SF$ = 12 παρουσιάζει θεωρητικά κέρδος διασποράς $10\log_{10}(2^{12}) 
% \approx 36dB$ συγκριτικά με ένα σύστημα χωρίς διασπορά. Στην πράξη, αυτό σημαίνει ότι ο δέκτης 
% \en{LoRa} μπορεί να ανιχνεύσει και να αποκωδικοποιήσει σήματα ακόμα και όταν η ισχύς τους 
% βρίσκεται αρκετά κάτω από το επίπεδο του θερμικού θορύβου (αρνητικό \en{SNR}). Φυσικά, το αυξημένο 
% αυτό κέρδος συνοδεύεται από το αντίτιμο της μειωμένης ταχύτητας μετάδοσης, όπως φάνηκε παραπάνω, 
% αποτελώντας τη γνωστή ανταλλαγή ευαισθησίας-ρυθμού (\en{sensitivity vs data rate trade-off}) στη 
% διαμόρφωση \en{LoRa}.

% kkkkkkkkkkkkkkkkkkkkkkkkkkkkkkkkkkkkkkkkkkkkkkkkkkkkkkkkkkkkkkkkkkkkkkkkkkkkkkkkkkkkkkkk

% \subsection{\en{OLD: Chirp Spread Spectrum} και η υλοποίησή του στη \en{LoRa}}

% Η τεχνολογία \en{LoRa} χρησιμοποιεί μια ιδιόκτητη τεχνική διαμόρφωσης ευρέως φάσματος γνωστή ως 
% \en{Chirp Spread Spectrum (CSS)} \cite{SemtechModulationBasics}. Σε αυτήν τη διαμόρφωση, το 
% αρχικό σήμα διαμορφώνεται μέσω παλμών με γραμμικά μεταβαλλόμενη συχνότητα, που ονομάζονται 
% \en{chirps}. Ένα \en{chirp} είναι ουσιαστικά ένα σήμα του οποίου η συχνότητα μεταβάλλεται γραμμικά 
% με τον χρόνο μέσα σε ένα καθορισμένο εύρος ζώνης. Ένα \en{chirp} αυξανόμενης συχνότητας ονομάζεται 
% \en{up-chirp}, ενώ μειούμενης συχνότητας \en{down-chirp}.

% Κάθε σύμβολο στη διαμόρφωση \en{LoRa CSS} μεταδίδεται ως ένα \en{chirp}, που καλύπτει όλο το εύρος 
% ζώνης (π.χ., $125 kHz$). Η πληροφορία κωδικοποιείται ως κυκλική μετατόπιση στη φάση ή τη συχνότητα 
% εκκίνησης του \en{chirp}. Συγκεκριμένα, η τιμή του κάθε ψηφιακού συμβόλου καθορίζει το σημείο στο 
% οποίο ξεκινά το \en{chirp} μέσα στο διαθέσιμο φάσμα \cite{GhoslyaCSS2024}.

% Ορίζεται ο παράγοντας εξάπλωσης (\en{Spreading Factor, SF}), που καθορίζει τον αριθμό των \en{chips} 
% ανά σύμβολο ως $N = 2^{SF}$. Ένα \en{chip} αποτελεί τη μικρότερη χρονική μεταβολή της συχνότητας 
% εντός του εύρους ζώνης. Έτσι, κάθε σύμβολο αποτελείται από $2^{SF}$ \en{chips}, κωδικοποιώντας 
% συνολικά $SF$ \en{bits} πληροφορίας.

% Η διάρκεια του συμβόλου ($T_{sym}$) εξαρτάται από το εύρος ζώνης ($BW$) και το $SF$ ως εξής:
% \begin{equation}
% T_{sym} = \frac{2^{SF}}{BW},
% \end{equation}
% όπου $T_{sym}$ μετράται σε δευτερόλεπτα. Συνεπώς, η διάρκεια συμβόλου αυξάνεται εκθετικά με το $SF$ 
% και μειώνεται αντιστρόφως ανάλογα με το $BW$. Ο ρυθμός συμβόλων ($R_{sym}$) ορίζεται ως:
% \begin{equation}
% R_{sym} = \frac{BW}{2^{SF}},
% \end{equation}
% ενώ ο αντίστοιχος ρυθμός μετάδοσης bit ($R_b$) υπολογίζεται από τη σχέση:
% \begin{equation}
% R_b = SF \cdot \frac{BW}{2^{SF}}.
% \end{equation}

% Στην πράξη, το \en{LoRa} εφαρμόζει επίσης διορθωτική κωδικοποίηση (\en{Forward Error Correction, FEC}) 
% με λόγο κωδικοποίησης που μειώνει τον καθαρό ρυθμό μετάδοσης:
% \begin{equation}
% R_b^{(net)} = r_c \cdot R_b,
% \end{equation}
% όπου $r_c$ είναι ο λόγος κωδικοποίησης (π.χ., 4/5, 4/6, κλπ.).

% Ο ρυθμός \en{chips} ($R_c$) ισούται με το εύρος ζώνης:
% \begin{equation}
% R_c = BW,
% \end{equation}
% και κάθε \en{chip} έχει διάρκεια:
% \begin{equation}
% T_{chip} = \frac{1}{BW}.
% \end{equation}

% Η διαμόρφωση \en{CSS} παρέχει σημαντικό κέρδος διασποράς ($G_{sp}$), το οποίο εκφράζεται ως:
% \begin{equation}
% G_{sp} \approx 10\log_{10}(2^{SF}),(dB).
% \end{equation}
% Το κέρδος αυτό επιτρέπει στο \en{LoRa} να επιτύχει αξιόπιστη επικοινωνία με αρνητικούς λόγους σήματος 
% προς θόρυβο (\en{SNR}), αυξάνοντας σημαντικά την εμβέλεια και την ανθεκτικότητα στις παρεμβολές και 
% στο φαινόμενο \en{Doppler}.

% Η μετάδοση δεδομένων ξεκινά με ένα προοίμιο (\en{preamble}) από διαδοχικά \en{up-chirps}, 
% ακολουθούμενο από ειδικά \en{down-chirps} που βοηθούν στον συγχρονισμό. Ακολουθούν τα σύμβολα 
% δεδομένων, καθένα εκ των οποίων κωδικοποιείται μέσω διαφορετικών κυκλικών μετατοπίσεων, προσφέροντας 
% ανθεκτικότητα σε θόρυβο και παρεμβολές και επιτρέποντας αξιόπιστη επικοινωνία ακόμα και σε κινητές 
% εφαρμογές.

% -------------------------------------
% Ενότητα 2.4: Το Πρωτόκολλο LoRaWAN
% -------------------------------------


\section{Το Πρωτόκολλο \en{LoRaWAN}}

Μετά την ανάλυση της φυσικής στρώσης \en{LoRa} και των δικτύων \en{LPWAN} στις προηγούμενες 
ενότητες, στρεφόμαστε τώρα στο πρωτόκολλο \en{LoRaWAN}. Το \en{LoRaWAN} είναι ένα ανοικτό 
πρωτόκολλο δικτύου που αναπτύσσεται από τη συμμαχία \en{LoRa Alliance} και λειτουργεί πάνω 
από τη διαμόρφωση \en{LoRa}, καθορίζοντας πώς οι συσκευές επικοινωνούν σε επίπεδο \en{MAC} 
και δικτύου. Ενώ το \en{LoRa} εξασφαλίζει τη μετάδοση σε μεγάλες αποστάσεις με χαμηλή ισχύ, 
το \en{LoRaWAN} ορίζει την αρχιτεκτονική και τους κανόνες δικτύωσης ώστε εκατομμύρια 
τερματικές συσκευές να μπορούν να συνδεθούν αξιόπιστα μέσω μίας υποδομής μεγάλης κλίμακας. Το 
πρωτόκολλο αυτό έχει σχεδιαστεί ειδικά για τις απαιτήσεις του \en{IoT}: αμφίδρομη επικοινωνία 
με χαμηλό ρυθμό δεδομένων, ασφάλεια από άκρο σε άκρο, δυνατότητα κινητικότητας και στήριξη 
υπηρεσιών εντοπισμού θέσης \cite{loraalliance_about_lorawan}. Παρακάτω παρουσιάζονται αναλυτικά η αρχιτεκτονική του 
\en{LoRaWAN}, οι κατηγορίες λειτουργίας των συσκευών, οι τύποι μηνυμάτων και η μορφή του 
πλαισίου, οι μηχανισμοί ενεργοποίησης και ασφάλειας, καθώς και σημαντικές παραμέτροι 
όπως το \en{Adaptive Data Rate} και οι περιορισμοί εκπομπής, κάνοντας αναφορές στα επίσημα 
πρότυπα (\en{LoRa Alliance}) και πρόσφατη βιβλιογραφία όπου ενδείκνυται. Τέλος, παρουσιάζονται 
συνοπτικά στοιχεία του οικοσυστήματος \en{LoRaWAN} στην πράξη (π.χ. \en{The Things Network}) 
και συνδέσεις με εφαρμογές όπως η τηλεμέτρηση ενέργειας με έξυπνους μετρητές σε υποσταθμούς 
ηλεκτρικής ενέργειας.

\begin{Illustration}[!ht] 
  \centering
	\includegraphics[width=0.8\textwidth]{figures/LoRa-LoRaWAN_layers.png} 
  \caption{Τεχνολογική στοίβα των \en{LoRa} και \en{LoRaWAN}.}
  \label{figure2.9}
  \cite{semtech_lora_lorawan} 
\end{Illustration} 



%%%%   Υποενότητα 2.4.1: Αρχιτεκτονική δικτύου LoRaWAN   %%%%


\subsection{Αρχιτεκτονική δικτύου \en{LoRaWAN}}

Το δίκτυο \en{LoRaWAN} υλοποιείται σε τοπολογία τύπου \en{«star-of-stars»} (αστέρι των 
αστεριών), όπου οι πύλες (\en{gateways}) λειτουργούν ως διαμεσολαβητές μεταφέροντας 
ασύρματα μηνύματα μεταξύ των τερματικών συσκευών και ενός κεντρικού \en{Network Server}. 
Στην Εικόνα 2.10 παρακάτω απεικονίζεται μια τυπική αρχιτεκτονική \en{LoRaWAN}, με τις 
τερματικές συσκευές (\en{end devices}) να επικοινωνούν μέσω πολλαπλών πυλών με έναν κεντρικό 
\en{Network Server}, ενώ τα δεδομένα προωθούνται τελικά σε έναν \en{Application Server}, όπου 
βρίσκεται η τελική εφαρμογή του χρήστη. Το \en{LoRaWAN Network Server} (\en{LNS}) είναι υπεύθυνο για τον 
έλεγχο και συντονισμό του δικτύου, ενώ ένας ξεχωριστός \en{Join Server} μπορεί να συμμετέχει 
στη διαδικασία εισόδου νέων συσκευών στο δίκτυο, αποθηκεύοντας τα απαραίτητα κλειδιά και 
συνδράμοντας στον υπολογισμό των κλειδιών ασφαλείας κατά την ενεργοποίηση (περισσότερα για 
αυτό στο Υποτμήμα 2.4.4) \cite{ttn_lorawan}.

\begin{Illustration}[!ht] 
  \centering
	\includegraphics[width=1\textwidth]{figures/LoRaWAN_architecture.png} 
  \caption{Τυπική αρχιτεκτονική \en{LoRaWAN} δικτύου.}
  \label{figure2.10}
  \cite{ttn_lorawan}
\end{Illustration} 

\textbf{Τερματικές Συσκευές (\en{End Devices})}: Πρόκειται για αισθητήρες, μετρητές ή ενεργοποιητές 
που διαθέτουν πομποδέκτη \en{LoRa}. Είναι συνήθως συσκευές χαμηλής ισχύος, συχνά με μπαταρία, 
που στέλνουν δεδομένα (\en{uplinks}) ή λαμβάνουν εντολές (\en{downlinks}) ασύρματα. Κάθε 
τερματική συσκευή επικοινωνεί απευθείας με όποιες πύλες βρίσκονται στην εμβέλειά της, 
χρησιμοποιώντας την ασύρματη ζεύξη \en{LoRa} χωρίς ανάγκη συσχέτισης με συγκεκριμένη πύλη. 
Η μετάδοση είναι τύπου \en{ALOHA}, δηλαδή χωρίς χειραψία, και μπορεί να ακουστεί από πολλές 
πύλες ταυτόχρονα \cite{ttn_lorawan}. Η κάθε συσκευή αναγνωρίζεται στο δίκτυο από μια διεύθυνση συσκευής 
(\en{DevAddr}) μήκους \en{32-bit}, που εκχωρείται κατά την ενεργοποίηση.

\textbf{Πύλες (\en{Gateways})}: Οι πύλες λειτουργούν ως διαφανείς γέφυρες που μετατρέπουν τα ασύρματα 
πακέτα \en{LoRa} σε πακέτα \en{IP} και αντιστρόφως \cite{loraalliance_about_lorawan}. Μια πύλη \en{LoRaWAN} περιλαμβάνει 
δέκτη/πομπό \en{LoRa} (συχνά με δυνατότητα ταυτόχρονης λήψης σε πολλαπλά κανάλια συχνοτήτων) 
και συνδέεται μέσω δικτύου \en{IP} (\en{Ethernet}, \en{Wi-Fi}, ή κινητή σύνδεση \en{LTE/5G}) 
με τον \en{Network Server}. Δεν πραγματοποιεί τοπική αναμετάδοση ή δρομολόγηση, αντίθετα κάθε 
λαμβανόμενο πλαίσιο \en{LoRa} προωθείται αυτούσιο στον \en{Network Server}. Σε αντίθεση με 
τα δίκτυα κινητής, οι πύλες \en{LoRaWAN} δεν διαχειρίζονται συσχετίσεις σύνδεσης. Οποιαδήποτε 
πύλη που λαμβάνει ένα έγκυρο πακέτο από μια συσκευή θα το μεταδώσει προς το κεντρικό δίκτυο. 
Αυτό επιτρέπει πλεονασμό και ευρεία κάλυψη, καθώς ένα \en{uplink} μήνυμα μπορεί να ληφθεί από πολλές 
πύλες παράλληλα. Ο \en{Network Server} φροντίζει να απορρίψει τα διπλότυπα και να κρατήσει 
μόνο ένα αντίγραφο (συνήθως από την πύλη με το καλύτερο σήμα). Οι πύλες αποτελούν το μοναδικό 
σημείο εκπομπής \en{downlink} μηνυμάτων από το δίκτυο προς τις συσκευές \cite{ttn_lorawan}. Αξίζει να σημειωθεί 
ότι σε εφαρμογές διαχείρισης δικτύων ενέργειας ή βιομηχανικών εγκαταστάσεων, οι πύλες 
\en{LoRaWAN} μπορούν να τοποθετηθούν π.χ. σε υποσταθμούς ή κέντρα ελέγχου ώστε να συλλέγουν 
δεδομένα από πολλούς κατανεμημένους αισθητήρες (ενδεικτικά, μετρητές κατανάλωσης) στην 
περιοχή.

\textbf{\en{Network Server} (Διακομιστής Δικτύου)}: Ο \en{Network Server} (\en{NS}) είναι η «καρδιά» 
του δικτύου \en{LoRaWAN}. Πρόκειται για λογισμικό που τρέχει σε κεντρικό διακομιστή 
(\en{cloud} ή \en{on-premise}) και επιτελεί μια σειρά από κρίσιμες λειτουργίες δικτύου: \en{i}) 
Επικυρώνει την αυθεντικότητα των συσκευών και την ακεραιότητα των μηνυμάτων, ελέγχοντας τον 
\en{Message Integrity Code} (\en{MIC}) κάθε πλαισίου με τα κατάλληλα κλειδιά. \en{ii}) Κάνει 
απαλοιφή διπλοτύπων (\en{deduplication}) για \en{uplink} πακέτα που έλαβε ταυτόχρονα από 
πολλαπλές πύλες. \en{iii}) Καταχωρεί και διαχειρίζεται τις ενεργές συσκευές και τις διευθύνσεις 
τους (\en{DevAddr}), ελέγχοντας επίσης το εύρος των \en{frame counters} για αποτροπή 
επαναλήψεων (\en{replay attacks}). \en{iv}) Δρομολογεί τα εξερχόμενα \en{application payloads} 
προς τους αντίστοιχους \en{Application Servers} και αντίστροφα, λαμβάνοντας από αυτούς 
καθοδηγούμενα \en{downlink} μηνύματα για τις συσκευές. \en{v}) Επιλέγει την πλέον κατάλληλη πύλη 
για να μεταδώσει ένα \en{downlink} προς μια συσκευή (συνήθως την πύλη που είχε το καλύτερο 
σήμα στο τελευταίο \en{uplink} του εν λόγω κόμβου). \en{vi}) Αποστέλλει εντολές διαχείρισης 
σύνδεσης και πόρων, όπως τις εντολές \en{ADR} (\en{Adaptive Data Rate}) προς τις συσκευές, 
ρυθμίζοντας το ρυθμό δεδομένων ή την ισχύ εκπομπής τους για βελτιστοποίηση της ενεργειακής 
κατανάλωσης και της χωρητικότητας του δικτύου. \en{vii}) Συντονίζει τις διαδικασίες ενεργοποίησης 
συσκευών (\en{OTAA}), προωθώντας τα σχετικά μηνύματα \en{Join} προς τον κατάλληλο 
\en{Join Server} και διασφαλίζοντας την ορθή διανομή των κλειδιών ασφαλείας (βλ. 2.4.4). 
Συνολικά, ο \en{Network Server} υλοποιεί ολόκληρο το πρωτόκολλο \en{LoRaWAN} στη μεριά του 
δικτύου και ενεργεί ως το μόνο σημείο λήψης αποφάσεων για τη ροή των δεδομένων και τον 
έλεγχο των συσκευών \cite{ttn_lorawan}.

\begin{Illustration}[!ht] 
  \centering
	\includegraphics[width=1\textwidth]{figures/LoRaWAN_Network_Server_architecture.png} 
  \caption{Αρχιτεκτονική \en{LoRaWAN Network Server}.}
  \label{figure2.11}
  \cite{tti_homepage}
\end{Illustration} 

\textbf{\en{Application Server} (Διακομιστής Εφαρμογών)}: Ο \en{Application Server} (\en{AS}) είναι 
υπεύθυνος για την παραλαβή και επεξεργασία των δεδομένων εφαρμογής από τις συσκευές, καθώς 
και για τη δημιουργία τυχόν \en{downlink} μηνυμάτων σε επίπεδο εφαρμογής. Στο \en{LoRaWAN} 
η ασφάλεια είναι διαμοιρασμένη, έτσι ο \en{Application Server} διατηρεί το κλειδί εφαρμογής 
(\en{AppSKey}) για κάθε συσκευή, προκειμένου να αποκρυπτογραφεί τα δεδομένα που προωθεί ο 
\en{Network Server}. Οποιαδήποτε επιχειρησιακή λογική (π.χ. αποθήκευση μετρήσεων, ανάλυση 
δεδομένων, εμφάνιση σε πίνακες ελέγχου) υλοποιείται πάνω από τον \en{Application Server}. 
Σημειώνεται ότι μπορεί να υπάρχουν πολλαπλοί \en{Application Servers} σε ένα δίκτυο (π.χ. 
διαφορετικοί οργανισμοί να λαμβάνουν δεδομένα από τις δικές τους συσκευές) και ο 
\en{Network Server} φροντίζει να δρομολογεί σωστά τα πακέτα σε καθέναν από αυτούς (βάσει 
του \en{DevAddr}/εφαρμογής που αντιστοιχεί στη συσκευή) \cite{ttn_lorawan}.

\textbf{\en{Join Server}}: Ο \en{Join Server} είναι μια επιπρόσθετη οντότητα (διακομιστής) που 
εμφανίστηκε κυρίως μετά την έκδοση \en{LoRaWAN} 1.1. Αναλαμβάνει να διαχειρίζεται την 
ασφαλή ενεργοποίηση των συσκευών. Συγκεκριμένα, ο \en{Join Server} αποθηκεύει τα 
\en{Root Keys} των συσκευών (βασικά κλειδιά εγγραφής, όπως το \en{AppKey}/\en{NwkKey}) 
και συμμετέχει στη διαδικασία \en{Over-The-Air Activation} (\en{OTAA}) υπολογίζοντας και 
παρέχοντας τα προσωρινά κλειδιά συνεδρίας τόσο στον \en{Network Server} όσο και στον 
\en{Application Server} \cite{TrendMicro2021LoRaWANSecurity}. Με αυτόν τον διαχωρισμό, επιτυγχάνεται απομόνωση των πεδίων 
ασφαλείας: ο \en{Network Server} δεν χρειάζεται να γνωρίζει το κλειδί εφαρμογής της 
συσκευής, ενώ ο \en{Application Server} δεν γνωρίζει τα κλειδιά δικτύου, ενώ αμφότεροι 
λαμβάνουν μόνο τα κλειδιά συνεδρίας που τους αναλογούν από τον \en{Join Server}. Στις 
πρώτες εκδόσεις (1.0.\en{x}) ο ρόλος του \en{Join Server} είτε δεν υπήρχε (το \en{AppKey} 
ήταν γνωστό απευθείας στο \en{Network Server}) είτε μπορούσε να συμπίπτει με τον 
\en{Application Server}. Στο \en{LoRaWAN} 1.1 όμως καθορίζεται ρητά ξεχωριστός 
\en{Join Server}, βελτιώνοντας την ασφάλεια και υποστηρίζοντας επιπλέον λειτουργίες 
όπως η περιαγωγή μεταξύ δικτύων \cite{ttn_lorawan} \cite{Loukil2022AnalysisLoRaWAN}. Η διαδικασία \en{OTAA} με τη συμμετοχή \en{Join Server} 
περιγράφεται λεπτομερώς στην Ενότητα 2.4.4.

Τέλος, αξίζει να αναφέρουμε ότι η αρχιτεκτονική \en{end-to-end} του \en{LoRaWAN} δεν 
οριοθετεί αυστηρά το εμπορικό μοντέλο υλοποίησης, μιας και μπορούν να υπάρξουν δημόσια δίκτυα, 
ιδιωτικά δίκτυα ή κοινόχρηστες υποδομές. Το πρότυπο εγγυάται τη διαλειτουργικότητα: μια 
πιστοποιημένη συσκευή \en{LoRaWAN} μπορεί να λειτουργήσει σε οποιοδήποτε συμβατό δίκτυο. 
Για παράδειγμα, το \en{The Things Network} (\en{TTN}) αποτελεί ένα παγκόσμιο δημόσιο/\en{community} δίκτυο \en{LoRaWAN}, ενώ υπάρχουν πάροχοι που προσφέρουν εμπορικές υποδομές 
(\en{Orange}, \en{LORIOT}, κ.ά.), καθώς και δυνατότητα για εντελώς ιδιωτικά δίκτυα 
(π.χ. εγκατάσταση ενός \en{Network Server} και \en{gateways} αποκλειστικά για μια 
βιομηχανική εγκατάσταση ή ένα δίκτυο ενέργειας).





%%%%   Υποενότητα 2.4.2: Κλάσεις συσκευών και χρονισμοί επικοινωνίας   %%%%


\subsection{Κλάσεις συσκευών και χρονισμοί επικοινωνίας}



\section{Ηλεκτρικοί Υποσταθμοί και Ανάγκες Εποπτείας}
Οι ηλεκτρικοί υποσταθμοί αποτελούν κρίσιμα σημεία του ηλεκτρικού συστήματος μεταφοράς και διανομής ενέργειας. Ο ρόλος τους είναι η μετατροπή της τάσης από υψηλά επίπεδα μεταφοράς σε χαμηλότερα επίπεδα που είναι κατάλληλα για διανομή και τελική κατανάλωση. Οι υποσταθμοί μπορούν να είναι είτε πρωτεύοντες (μεταφοράς), είτε δευτερεύοντες (διανομής).

Η εποπτεία και διαχείριση των υποσταθμών περιλαμβάνει:
\begin{itemize}
  \item παρακολούθηση ηλεκτρικών παραμέτρων όπως ρεύμα, τάση, ισχύς και συχνότητα ανά φάση,
  \item ανίχνευση βλαβών ή ανομαλιών (π.χ. υπερφόρτιση, βυθίσεις τάσης),
  \item έλεγχο λειτουργικών μονάδων όπως διακόπτες ισχύος και προστατευτικά ρελέ,
  \item λήψη αποφάσεων σε πραγματικό χρόνο για την εξασφάλιση της αδιάλειπτης παροχής και της ασφάλειας του εξοπλισμού.
\end{itemize}

Παραδοσιακά, τέτοια εποπτεία γινόταν με ενσύρματες ή \en{SCADA} λύσεις υψηλού κόστους. Η ενσωμάτωση τεχνολογιών όπως το \en{LoRaWAN} επιτρέπει τη δημιουργία αποκεντρωμένων, χαμηλού κόστους και επεκτάσιμων λύσεων, κατάλληλων ακόμη και για μικρούς ή απομακρυσμένους υποσταθμούς.

\section{Συμπεράσματα}
Οι τεχνολογίες \en{LPWAN} και ιδιαίτερα το \en{LoRaWAN} παρέχουν μια αποτελεσματική λύση για τηλεμετρικές εφαρμογές σε περιβάλλοντα όπου απαιτείται χαμηλή κατανάλωση ισχύος και μεγάλη απόσταση μετάδοσης. Το θεωρητικό αυτό υπόβαθρο θεμελιώνει την επιλογή του \en{LoRaWAN} ως βασική τεχνολογία επικοινωνίας στο σύστημα παρακολούθησης και ελέγχου υποσταθμού που αναπτύχθηκε στο πλαίσιο της παρούσας διπλωματικής εργασίας.
