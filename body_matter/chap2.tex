\chapter{Τεχνολογίες \en{LPWAN} και το Πρωτόκολλο \en{LoRaWAN}}
\label{chap:lpwan}

\section{Εισαγωγή στα \en{LPWAN}}

Η διαρκώς αυξανόμενη ανάγκη για απομακρυσμένη και αποδοτική, ως προς την ενέργεια, επικοινωνία 
μεταξύ έξυπνων συσκευών και αισθητήρων έχει οδηγήσει στην εμφάνιση και εξέλιξη μιας νέας γενιάς 
ασύρματων τεχνολογιών, γνωστών ως \en{Low Power Wide Area Networks (LPWAN)}. Οι 
τεχνολογίες \en{LPWAN} επιτρέπουν την αποστολή μικρών σε ποσότητα δεδομένων σε μεγάλες
 αποστάσεις με εξαιρετικά χαμηλή κατανάλωση ενέργειας, καθιστώντας τις ιδανικές για 
 εφαρμογές \en{Internet of Things (IoT)}, όπου η διάρκεια ζωής της μπαταρίας και η 
 αξιοπιστία είναι κρίσιμοι παράγοντες.

Σε αντίθεση με τις τεχνολογίες \en{Wi-Fi} ή \en{Bluetooth}, οι οποίες είναι σχεδιασμένες 
για υψηλούς ρυθμούς μετάδοσης δεδομένων σε μικρές αποστάσεις, τα \en{LPWAN} είναι 
προσανατολισμένα στην υποστήριξη ενός μεγάλου αριθμού συσκευών, με δυνατότητα μετάδοσης 
δεδομένων σε αποστάσεις που υπερβαίνουν τα 10 χιλιόμετρα, σε ανοικτό πεδίο, και σε 
συχνότητες που βρίσκονται στο μη αδειοδοτημένο φάσμα (\en{unlicensed spectrum}). Το 
σημαντικότερο, μάλιστα, όφελος έναντι άλλων τεχνολογιών μετάδωσης πληροφορίας μεγάλου έυρους
(όπως το φάσμα κινητής τηλεφωνίας \en{3G, 4G ή 5G}), είναι η ελάχιστη ενέργεια που απαιτείται 
για την τροφοδοσία των αντίστοιχων συσκευών \cite{ICTexpress}.

Οι πιο διαδεδομένες τεχνολογίες \en{LPWAN} είναι οι εξής:

\begin{itemize}
    \item \textbf{\en{NB-IoT (Narrowband Internet of Things)}} – αποτελεί τεχνολογία 
    βασισμένη στο \en{LTE (Long-Term Evolution)}, η οποία λειτουργεί σε αδειοδοτημένο φάσμα και προσφέρει 
    αξιόπιστη κάλυψη εντός κτιρίων (\en{deep indoor penetration}). Υποστηρίζεται από 
    το πρότυπο \en{3GPP Release 13} και έχει σχεδιαστεί για εφαρμογές με ανάγκες μαζικής 
    συνδεσιμότητας και μικρού όγκου δεδομένων, όπως μετρητές νερού ή αερίου.
    
    \item \textbf{\en{LTE-M (LTE Cat-M1)}} – επίσης βασίζεται στο \en{LTE} και προσφέρει 
    υψηλότερους ρυθμούς μετάδοσης από το \en{NB-IoT}, διατηρώντας ωστόσο χαμηλή κατανάλωση. 
    Είναι κατάλληλο για φορητές εφαρμογές που απαιτούν αμφίδρομη επικοινωνία σε πραγματικό 
    χρόνο, όπως \en{asset tracking} και \en{wearables}.
    
    \item \textbf{\en{LoRa} και \en{LoRaWAN}} – πρόκειται για το πιο διαδεδομένο πρωτόκολλο 
    σε μη-αδειοδοτημένο φάσμα (π.χ. 868~\en{MHz} στην Ευρώπη), με κύρια πλεονεκτήματα την 
    ευκολία υλοποίησης, τη μεγάλη αυτονομία (έως και 10 έτη), τη χαμηλή κατανάλωση ισχύος 
    και την υψηλή ευελιξία ανάπτυξης μέσω ιδιωτικών ή δημόσιων δικτύων. Η τεχνολογία \en{LoRa} 
    αναπτύχθηκε αρχικά από τη γαλλική \en{Cycleo} και κατοχυρώθηκε από τη \en{Semtech}, 
    ενώ το \en{LoRaWAN} αναπτύσσεται και προτυποποιείται από τη \en{LoRa Alliance}.
\end{itemize}

Τα δίκτυα \en{LPWAN} ενσωματώνονται όλο και περισσότερο σε κρίσιμες υποδομές, όπως συστήματα 
παρακολούθησης ενέργειας, γεωργίας ακριβείας, έξυπνων πόλεων και βιομηχανικής αυτοματοποίησης, 
προσφέροντας λύσεις υψηλής κάλυψης, ανθεκτικότητας και χαμηλού κόστους εγκατάστασης.



\section{\en{LoRa} – Φυσικό Επίπεδο}
Το \en{LoRa} (\en{Long Range}) αποτελεί την τεχνολογία στο φυσικό επίπεδο (\en{PHY}) και βασίζεται στη μέθοδο διαμόρφωσης \en{Chirp Spread Spectrum (CSS)}. Αναπτύχθηκε αρχικά από τη γαλλική εταιρεία \en{Cycleo} και αργότερα αποκτήθηκε από τη \en{Semtech}, η οποία κατέχει τα αποκλειστικά πνευματικά δικαιώματα για τη χρήση της.

Το \en{LoRa} λειτουργεί στο \en{ISM} φάσμα (π.χ. 868 \en{MHz} για Ευρώπη) και προσφέρει εξαιρετική ανθεκτικότητα στο θόρυβο, υψηλό \en{link budget} και χαμηλό \en{bitrate}, καθιστώντας την κατάλληλη για χρήση σε αισθητήρες που λειτουργούν με μπαταρία για πολλά έτη.

\section{\en{LoRaWAN} – Επίπεδο \en{MAC}}
Το \en{LoRaWAN} είναι το πρωτόκολλο που λειτουργεί στο επίπεδο \en{MAC (Media Access Control)} και καθορίζει την αρχιτεκτονική του δικτύου. Βασίζεται σε αρχιτεκτονική \en{Star-of-Stars}, όπου οι τελικές συσκευές επικοινωνούν με \en{gateways} που συνδέονται στο διαδίκτυο και δρομολογούν τα δεδομένα προς έναν κεντρικό \en{Network Server}.

Οι συσκευές \en{LoRaWAN} ταξινομούνται σε τρεις κλάσεις:
\begin{itemize}
  \item \textbf{\en{Class A}} – βασική λειτουργία, με αμφίδρομη επικοινωνία μετά από \en{uplink}.
  \item \textbf{\en{Class B}} – συγχρονισμός με \en{beacon} για \en{downlink slots}.
  \item \textbf{\en{Class C}} – μόνιμη λήψη (μόνο όταν η συσκευή έχει συνεχή τροφοδοσία).
\end{itemize}

Επιπλέον, υποστηρίζονται δύο τρόποι ενεργοποίησης:
\begin{itemize}
  \item \en{OTAA (Over-The-Air Activation)}
  \item \en{ABP (Activation By Personalization)}
\end{itemize}

\section{Σύγκριση Τεχνολογιών \en{LPWAN}}
Η παρακάτω σύγκριση παρουσιάζει βασικά χαρακτηριστικά των \en{NB-IoT}, \en{LTE-M} και \en{LoRa}:

\begin{center}
\begin{tabular}{|c|c|c|c|}
\hline
\textbf{Παράμετρος} & \textbf{\en{NB-IoT}} & \textbf{\en{LTE-M}} & \textbf{\en{LoRa}} \\
\hline
Τυποποίηση & \en{3GPP} & \en{3GPP} & \en{LoRa Alliance} \\
Φάσμα & \en{Licensed} & \en{Licensed} & \en{Unlicensed (ISM)} \\
Εμβέλεια & 15 \en{km} & 10 \en{km} & 15 \en{km} \\
\en{Bitrate} & 200 \en{kbps} & 1 \en{Mbps} & 50 \en{kbps} \\
Αυτονομία μπαταρίας & 7.5 έτη & 1.5 έτη & 8.75 έτη \\
Αμφίδρομη επικοινωνία & Ναι & Ναι & Ναι \\
\hline
\end{tabular}
\end{center}

Κάθε τεχνολογία έχει τα πλεονεκτήματά της και η επιλογή εξαρτάται από την εφαρμογή. Το \en{LoRaWAN} υπερτερεί σε σενάρια όπου προέχει η ενεργειακή αυτονομία, η μεγάλη εμβέλεια και η χρήση μη-αδειοδοτημένου φάσματος.

\section{Ηλεκτρικοί Υποσταθμοί και Ανάγκες Εποπτείας}
Οι ηλεκτρικοί υποσταθμοί αποτελούν κρίσιμα σημεία του ηλεκτρικού συστήματος μεταφοράς και διανομής ενέργειας. Ο ρόλος τους είναι η μετατροπή της τάσης από υψηλά επίπεδα μεταφοράς σε χαμηλότερα επίπεδα που είναι κατάλληλα για διανομή και τελική κατανάλωση. Οι υποσταθμοί μπορούν να είναι είτε πρωτεύοντες (μεταφοράς), είτε δευτερεύοντες (διανομής).

Η εποπτεία και διαχείριση των υποσταθμών περιλαμβάνει:
\begin{itemize}
  \item παρακολούθηση ηλεκτρικών παραμέτρων όπως ρεύμα, τάση, ισχύς και συχνότητα ανά φάση,
  \item ανίχνευση βλαβών ή ανομαλιών (π.χ. υπερφόρτιση, βυθίσεις τάσης),
  \item έλεγχο λειτουργικών μονάδων όπως διακόπτες ισχύος και προστατευτικά ρελέ,
  \item λήψη αποφάσεων σε πραγματικό χρόνο για την εξασφάλιση της αδιάλειπτης παροχής και της ασφάλειας του εξοπλισμού.
\end{itemize}

Παραδοσιακά, τέτοια εποπτεία γινόταν με ενσύρματες ή \en{SCADA} λύσεις υψηλού κόστους. Η ενσωμάτωση τεχνολογιών όπως το \en{LoRaWAN} επιτρέπει τη δημιουργία αποκεντρωμένων, χαμηλού κόστους και επεκτάσιμων λύσεων, κατάλληλων ακόμη και για μικρούς ή απομακρυσμένους υποσταθμούς.

\section{Συμπεράσματα}
Οι τεχνολογίες \en{LPWAN} και ιδιαίτερα το \en{LoRaWAN} παρέχουν μια αποτελεσματική λύση για τηλεμετρικές εφαρμογές σε περιβάλλοντα όπου απαιτείται χαμηλή κατανάλωση ισχύος και μεγάλη απόσταση μετάδοσης. Το θεωρητικό αυτό υπόβαθρο θεμελιώνει την επιλογή του \en{LoRaWAN} ως βασική τεχνολογία επικοινωνίας στο σύστημα παρακολούθησης και ελέγχου υποσταθμού που αναπτύχθηκε στο πλαίσιο της παρούσας διπλωματικής εργασίας.
